\PassOptionsToPackage{table,dvipsnames,cmyk,svgnames,prologue}{xcolor}
\documentclass[final,border=10pt]{standalone}
%\documentclass[minion,final,a4paper,english,ngerman]{ttct}
%\usepackage[utf8]{inputenc}
%\usepackage[T1]{fontenc}
%\usepackage[lucidasmallscale]{lucidabr}
\usepackage%[osf]
{libertineotf}
\usepackage{microtype,dtk-logos}
\usepackage[english,ngerman]{babel}

\def\XInfofont{}

\usepackage{pstricks}
\usepackage{pst-node}
\def\PDF{PDF}\def\DVI{DVI}
%\usepackage{LaTeXRef}

\pagestyle{empty}
\begin{document}

\psset{shadow=true,shadowcolor=black!40}
\begin{pspicture}[showgrid=false](-1,-1.7)(12,9.3)
\rput(5,9){\rnode{A}{\psframebox[framearc=0.4,fillcolor=Goldenrod!80,fillstyle=solid]{\quad\textbf{\LaTeX\ Quelltext}\quad}}}
\rput(5,7.5){\rnode{B}{\psframebox[doubleline=true,fillcolor=Yellow!30,fillstyle=solid]{%
    \kern5mm\large\textbf{??\TeX-Compiler}\kern5mm}}}
\ncline[shadow=false]{->}{A}{B}
%
\rput(9.5,8.5){\rnode{C}{%
    \psframebox[framearc=0.4,fillcolor=Goldenrod!30,fillstyle=solid]{%
	\tabular{l}Klassendatei\\Zusatzpakete\endtabular}}}
\ncangle[shadow=false,angleA=-90,offsetB=-5pt,angleB=0]{->}{C}{B}
%
\rput(0.5,8.5){\rnode{D}{%
    \psframebox[framearc=0.4,fillcolor=Goldenrod!30,fillstyle=solid]{%
	\tabular{l}binäre Formatdatei\\Fonts (Metriken)\endtabular}}}
\ncangle[shadow=false,angleA=-90,angleB=180]{->}{D}{B}
%
\rput(4.0,6){\rnode{E}{\psframebox[framearc=0.4,fillcolor=Yellow!30,fillstyle=solid]{%
 	\rule[-8pt]{0pt}{20pt}\kern2pt\texttt{DVI}-Datei\kern2pt}}}
\ncangle[shadow=false,angleA=90,angleB=-90,offsetB=1cm]{<-}{E}{B}
%
\rput(7.25,6){\rnode{F}{\psframebox[framearc=0.4,fillcolor=Yellow!30,fillstyle=solid]{%
 	\rule[-8pt]{0pt}{20pt}\kern2pt Hilfsdateien\kern2pt}}}
%\ncline[offsetA=12pt,nodesepA=14pt]{->}{B}{F}
\psline[shadow=false]{->}(6.5,7.1)(6.5,6.47)
\ncangle[shadow=false,angleA=90,offsetA=-20pt,offsetB=5pt,angleB=0]{->}{F}{B}
%
\rput(10,6){\rnode{G}{\psframebox[fillcolor=Goldenrod!60,fillstyle=solid]{%
 	\rule[-8pt]{0pt}{20pt}\small\tabular{@{}l@{}}externe Programme\\[-2pt]
	Index, \\[-2pt] Bibliografie, \\[-2pt] Glossar)\endtabular}}}
\ncline[shadow=false]{->}{F}{G}
\ncangles[shadow=false,angleA=0,angleB=0]{->}{G}{B}
%
\rput(4.0,4.45){\rnode{H}{\psframebox[fillcolor=Goldenrod!80,fillstyle=solid]{%
 	\texttt{DVI}-Treiber\vphantom{j}}}}
\ncline[shadow=false]{->}{E}{H}
%
\rput(0,4.25){%
    \psframebox[framearc=0.4,fillcolor=Goldenrod!30,fillstyle=solid]{%
	\tabular{l}\rnode{I}{Fonts (Type~1,}\\%Bitmap,TrueType\\
	%\rnode{I1}{Type3, Bitmap}\\
	                                       \rnode{I1}{OpenType, \ldots)}\endtabular}}
\ncline[shadow=false,nodesepA=4mm]{->}{I}{H}
\ncline[shadow=false,nodesepA=4mm,nodesepB=-7.02cm]{->}{I1}{I1}\psdot(6,4.05)
%\ncline[shadow=false,offsetA=0.9mm]{->}{I}{H}
%\psline[shadow=false]{->}(1.35,4)(5.5,4)%
%
\pnode(4,3.85){dvipdf}\psdot(dvipdf)
%
\rput(4.0,3){\rnode{J}{\psframebox[framearc=0.4,fillcolor=Yellow!30,fillstyle=solid]{%
 	\rule[-8pt]{0pt}{20pt}\kern2pt\texttt{PS}-Datei\kern2pt}}}
\ncline[shadow=false]{->}{H}{J}
%
%\pnode(4.0,2.5){J}%\psdot(J)
\rput(4.0,1.5){\rnode{J1}{\psframebox[fillcolor=Goldenrod!80,fillstyle=solid]{%
 	\texttt{ghostscript}}}}
\ncline[shadow=false,angleA=180,angleB=90]{->}{J}{J1}
%
\rput(2.0,0){\rnode{xetex}{\psframebox[framearc=0.4,fillcolor=Yellow!30,fillstyle=solid]{%
 	\rule[-8pt]{0pt}{20pt}\kern2pt\texttt{PDF}-Datei\kern2pt}}}
\ncangle[shadow=false,angleA=180,angleB=90]{->}{dvipdf}{xetex}
%
\rput(4.0,0){\rnode{J2}{\psframebox[framearc=0.4,fillcolor=Yellow!30,fillstyle=solid]{%
 	\rule[-8pt]{0pt}{20pt}\kern2pt\texttt{PDF}-Datei\kern2pt}}}
\ncline[shadow=false]{->}{J1}{J2}
%
%
\rput(6,0){\rnode{K}{\psframebox[framearc=0.4,fillcolor=Yellow!30,fillstyle=solid]{%
 	\rule[-8pt]{0pt}{20pt}\kern2pt\texttt{PDF}-Datei\kern2pt}}}
\ncangle[shadow=false,angleA=90,angleB=-90,offsetB=-1cm]{<-}{K}{B}
%
\rput[lt](6.5,3){\parbox{5.5cm}{\raggedleft Der Aufbau eines \TeX-Systems mit pdf\TeX,
        \XeTeX\ oder Lua\TeX\ als Compiler,
	die eine Ausgabe im \PDF- oder im treiberunabhängigen \DVI-Format erlauben.}}
%\psline[fillcolor=black!50,fillstyle=solid,
%    linecolor=black!50,opacity=0.4,strokeopacity=0.4,shadow=false]%
%    (2,-0.75)(2,6.8)(5,6.8)(5,-0.75)
\rput[t](2,-0.7){\bfseries\tabular[t]{c}\XeLaTeX\endtabular}
\rput[t](4,-0.7){\bfseries\tabular[t]{c}\LaTeX\\ Lua\LaTeX\endtabular}
\rput[t](6,-0.7){\bfseries\tabular[t]{c}pdf\LaTeX\\ Lua\LaTeX\\\XeLaTeX\endtabular}
\psframe*[opacity=0.3,linecolor=black,shadow=false](2.5,1)(5.5,6.8)
\end{pspicture}



\end{document}

