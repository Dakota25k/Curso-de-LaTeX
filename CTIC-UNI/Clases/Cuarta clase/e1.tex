\documentclass[12pt,a4paper]{article}
\usepackage[utf8]{inputenc} %para tildes
\usepackage[T1]{fontenc}
\usepackage[spanish,es-sloppy]{babel}
\usepackage{here}
\usepackage{graphicx}
\usepackage[dvipsnames,usenames]{color}
%cambiar margenes
\usepackage[lmargin=2cm,rmargin=1.5cm,tmargin=2cm,bmargin=2cm]{geometry}
%Cambiar el espacio interlinear
\pagestyle{empty}%elimina la numeración de pags salvo la primera pagina

\linespread{1.5}%1.5 de latex no de word. Para word es 1.25
\title{Mi primera Portada}
\author{Ivonne\thanks{Beca}\\Lima \and Jorge\\Tacna}
\date{4 de junio de 2018}

\begin{document}
\maketitle	
\thispagestyle{empty} %elimina la numeración de la pag
\begin{abstract}
Para resumir las ideas presentadas en todo el documento. El objetivo, la hipótesis, la metodología y resultados. 
\end{abstract}

\renewcommand{\spanishabstractname}{Abstract}
\begin{abstract}
	Para resumir las ideas presentadas en todo el documento. El objetivo, la hipótesis, la metodología y resultados.
\end{abstract}

Tenemos varias referencia en las páginas \pageref{fig1} y en \pageref{fig2} de las figuras \ref{fig1} y \ref{fig2}.
	\begin{center}
			Universidad Nacional de Ingeniería
	\end{center}
	
	\noindent 3 movimientos que hace la Tierra (que no son ni rotación ni traslación) y que quizás no conocías \\%agregar una línea
	
	\centerline{Universidad Nacional de Ingeniería}%centra y con iguales distancias verticales.
	
	Seguramente lo habrás aprendido en la escuela primaria: la Tierra describe una órbita elíptica alrededor del Sol.\\[1cm]%agregar una línea + 2 cent
	
	Este recorrido, que se conoce como movimiento de traslación, le toma al planeta unos 365 días 	(más 5 horas, 45 minutos y 46 segundos).\newline%igual a doble \\
	\begin{flushright}
	El otro movimiento que \\te enseñaron es el de rotación: \\la Tierra gira en torno a su propio eje.El otro\\ movimiento que te enseñaron es el de 
	\end{flushright}
	
 30\%
	
	``estas son las comillas en latex''%comilla

\renewcommand{\spanishcontentsname}{Contenido de mi {Indice}}


\tableofcontents
	
\listoffigures	
\renewcommand{\spanishtablename}{Mis Tablitas}
\listoftables

\section[de letras]{Tipos de letras por defecto de letras por defectode letras por defectode letras por defectode letras por defectode letras por defectode letras por defectode}
	Universidad Nacional de Ingeniería
	
	\textrm{Universidad Nacional de Ingeniería}
	
	\textsf{Universidad Nacional de Ingeniería}
	
	\texttt{Universidad Nacional de Ingeniería}
	
	{\ttfamily Universidad Nacional de Ingeniería}
	
	\textit{Universidad Nacional de Ingeniería}
	
	\textsl{Universidad Nacional de Ingeniería}
	
	{\itshape Universidad Nacional de Ingeniería}
	
	\textbf{Universidad Nacional de Ingeniería}
	
	\underline{Universidad Nacional de Ingeniería}%falla cuando es más de 1 línea
	
	\underline{\textbf{Universidad Nacional de Ingeniería}}
	
\section{Tamaño de letras}\label{sec1}
	
	{\Huge Universidad Nacional de Ingeniería}
	
	{\LARGE Universidad Nacional de Ingeniería}
	
	\textcolor{Orchid}{Universidad Nacional de Ingeniería}
	
	{\color{Red}Universidad Nacional de Ingeniería}
	
\underline{text}
	\newpage
	Este giro sobre sí misma le toma aproximadamente un día (23 horas, 56 minutos 4,1 segundos, para ser exactos). 
	
	Sin embargo, estos no son los únicos movimientos que hace la Tierra.
	
	Te contamos —o recordamos— cuáles son los otros tres, también importantes, que ejecuta el planeta.\footnote{donde vivimos}
	\begin{figure}[H]
		\centering
		\includegraphics[width=3cm]{escudo}
		\caption{Este es escudoo de la Uni XXX09} \label{fig1}
	\end{figure}
	
	Movimiento de precesión de los equinoccios
	
	Este es el movimiento que describe el eje inclinado de la tierra de forma circular.
	
	Más concretamente, es el movimiento que hace el polo norte terrestre respecto al punto central de la elipse 
	que describe la Tierra en el movimiento de translación.
	
	Esta oscilación fue descrita por primera vez por el astrónomo, geógrafo y matemático griego Hiparco de Nicea 
	que vivió entre los años 190 a.C. y 120 a.C. y fue el tercer movimiento de la Tierra en ser detectado.
\section{Otras cosas}
	Este bamboleo cíclico en la orientación del eje de rotación de la Tierra demora alrededor de 25.780 años. 
	\clearpage
	
\vspace*{5cm}
HOla

\vspace{4cm}

\hspace*{3cm} Mundo \hspace{4cm} hoy \ \ \ \ día \quad lunes \qquad 4
	
	Su duración, no obstante, es relativamente imprecisa porque se ve influida por el movimiento y desplazamiento 
	de las placas tectónicas.
\subsection{Con color}
	¿Qué lo produce? Se genera por fundamentalmente por el momento de fuerza que ejerce el Sol sobre la Tierra. 
	
\subsubsection{Color y resaltado}	
	
	
	Movimiento de nutación
	
	Este movimiento se produce por una suerte de vibración del eje polar terrestre.
	
	Esto hace que, durante el movimiento de precesión de los equinoccios, los círculos que se describen no sean 
	perfectos sino irregulares.
	
	Es decir, el eje de la Tierra se inclina un poco más o un poco menos respecto a la circunferencia que describe 
	durante la precesión.
	
	El movimiento es cíclico y cada uno de los episodios dura algo más de 18 años y medio. Durante este tiempo, 
	la variación es de un máximo de 700 metros respecto a la posición inicial.
	
	La nutación fue descubierta por el astrónomo británico James Bradley en 1728.
	
	Varios años después hallaron la causa de este vaivén, cuando cálculos llevados a cabo por distintos científicos 
	concluyeron que era producto directo de la atracción gravitatoria de la Luna.
	
	Por lo visto en la sección \ref{sec1} cambiamos letras de la página \pageref{sec1}.
%\includegraphics{escudo}
\newpage
\includegraphics[scale=0.2]{escudo}
\includegraphics[width=4cm,height=2cm]{escudo}
%centrar el escudo
\begin{center}
	\includegraphics[scale=0.1]{escudo}
\end{center}

\begin{figure}[H]
	\centering
	\includegraphics[width=3cm]{escudo}
	\caption{Este es escudoo de la Uni XXX}\label{fig2}
\end{figure}

\begin{enumerate}
	\item One
	\item Direction
	\begin{enumerate}
		\item Harry
		\item George
		\begin{enumerate}
			\item London
		\end{enumerate}
		\item Max
	\end{enumerate}
	\item BTS
\end{enumerate}

\begin{itemize}
	\item Jhon
	\item Paul
	\begin{enumerate}
		\item Stella
		\item USA
	\end{enumerate}
	\item Yoko
\end{itemize}

\begin{description}
	\item[Casa] es el lugar donde vive una familia
	\item [Silla] es un objeto para sentarse\footnote{esto está en el pie de página}
\end{description}
\ \\[2cm]
\begin{tabular}{|l|c|r|}
	\hline
Mesa & Silla & Casa \\
\hline
mesa & silla & casa \\
\hline
MESA & SILLA & CASA\\
\hline
\end{tabular}

\renewcommand{\spanishtablename}{Tablita}
\begin{table}[H]
	\centering
	\caption{mi primera tabla}\label{tab1}
	\begin{tabular}{|l|c|r|}
		\hline
		Mesa & Silla & Casa \\
		\hline
		mesa & silla & casa \\
		\hline
		MESA & SILLA & CASA\\
		\hline
	\end{tabular}
	
\end{table}

\begin{table}[H]
	\centering
	\caption{mi primera tabla}\label{tab2}
	\begin{tabular}{|l|c|p{3cm}|}
		\hline
		\multicolumn{2}{|c|}{PALABRA} & Casa \\
		\hline
		mesa & silla & casa \\
		\cline{2-3}
		MESA & SILLA & CASA \newline CASA CASA CASA CASA CASA CASA\\
		\hline
	\end{tabular}
	
\end{table}

Vemos en la Tablas \ref{tab1} lo revisado en \cite[cap.7]{smi17}

\fbox{texto en cuadro}

\setlength{\fboxrule}{0.4pt}
\setlength{\fboxsep}{3pt}
\fbox{texto en cuadro de Boticelli}

\begin{minipage}{3cm}
		Es decir, el eje de la Tierra se inclina un poco más o un poco menos respecto a la circunferencia que describe 
	durante la precesión.
	
	El movimiento es cíclico y cada uno de los episodios dura algo más de 18 años y medio. Durante este tiempo, 
	la variación es de un máximo de 700 metros respecto a la posición inicial.
	
	La nutación fue descubierta por el astrónomo británico James Bradley en 1728.
\end{minipage}

\begin{thebibliography}{99}
	\bibitem{ve14} Venero B., Armando. 	``Análisis Matemático''. Gemar. 2014.
	\bibitem{smi17} Smith, John. \emph{Acerca del Agua}. Wiley, 2017.
\end{thebibliography}

\newpage 
Hello
\vfill mundo \hfill hoy

Nombre:\hrulefill 

Apellidos: \dotfill
\clearpage
	El movimiento es cíclico y cada uno de los episodios dura algo más de 18 años y medio. Durante este tiempo, 
la variación es de un máximo de 700 metros respecto a la posición inicial. No obstante $g(x)=x+z+y$ y $x+y$ es.
$$
x+y+z = g(x)
$$
\begin{equation}\label{ec1}
a+b+c=h(z)
\end{equation}

Viendo la equación (\ref{ec1}) aprendemos a sumar.

\end{document}