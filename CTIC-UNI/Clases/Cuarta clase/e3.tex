\documentclass{report}
\usepackage[utf8]{inputenc}

\newcommand*{\fd}{\Longrightarrow}

\newcommand*{\impo}{\bfseries\Large}

\newcommand{\negrita}[1]{\textbf{#1}}

\newcommand{\deri}[2]{\frac{\partial #1}{\partial #2}}



\usepackage{amsmath}

\DeclareMathOperator{\sen}{sen}

\DeclareMathOperator*{\algo}{algo}


\usepackage{amssymb}

\usepackage{amsthm}

\newtheorem{ej}{Ejemplo}
\newtheorem{ejr}[ej]{Ejercicio}
\theoremstyle{definition}
\newtheorem*{cor}{Corolario}

\swapnumbers
\newtheorem{teo}{Teorema}[chapter]

\theoremstyle{remark}
\newtheorem{prop}[teo]{Proposición}


\usepackage{url}

\usepackage{fancybox}

\usepackage{enumitem}
\setlist[enumerate,1]{label=\arabic*--+)}
\setlist[enumerate,2]{label=\Alph*\}}

\usepackage{mathptmx}

\usepackage{fancyhdr}
\pagestyle{fancy}
\lhead{John Smith}
\chead{}
\rhead{Tercera clase}
\lfoot{.}
\cfoot{.}
\rfoot{\thepage}
\renewcommand{\headrulewidth}{0pt}
\renewcommand{\footrulewidth}{2mm}


\usepackage[percent]{overpic}

\usepackage{pdflscape}

\usepackage{multicol}


%%\usepackage{hyperref}
\usepackage{apacite}

\begin{document}
	
\setlength{\columnsep}{1cm}
\setlength{\columnseprule}{1mm}
\begin{multicols}{3}
	Esta oscilación fue descrita por primera vez por el astrónomo, geógrafo y matemático griego Hiparco de Nicea 
que vivió entre los años 190 a.C. y 120 a.C. y fue el tercer movimiento de la Tierra en ser detectado

\columnbreak

	Esta oscilación fue descrita por primera vez por el astrónomo, geógrafo y matemático griego Hiparco de Nicea 
que vivió entre los años 190 a.C. y 120 a.C. y fue el tercer movimiento de la Tierra en ser detectado

	Esta oscilación fue descrita por primera vez por el astrónomo, geógrafo y matemático griego Hiparco de Nicea 
que vivió entre los años 190 a.C. y 120 a.C. y fue el tercer movimiento de la Tierra en ser detectado

	Esta oscilación fue descrita por primera vez por el astrónomo, geógrafo y matemático griego Hiparco de Nicea 
que vivió entre los años 190 a.C. y 120 a.C. y fue el tercer movimiento de la Tierra en ser detectado

	Esta oscilación fue descrita por primera vez por el astrónomo, geógrafo y matemático griego Hiparco de Nicea 
que vivió entre los años 190 a.C. y 120 a.C. y fue el tercer movimiento de la Tierra en ser detectado
\end{multicols}
	
	
	
	
	
	
\begin{overpic}[width=\linewidth]{cosplot}
\put(30,40){\sffamily\Large La función coseno}	
\put(75,30){\LARGE $\cos x$}
\end{overpic}	
	
	
\begin{landscape}
	Esta oscilación fue descrita por primera vez por el astrónomo, geógrafo y matemático griego Hiparco de Nicea 
	que vivió entre los años 190 a.C. y 120 a.C. y fue el tercer movimiento de la Tierra en ser detectado
	
	\begin{table}
		\centering
		\begin{tabular}{*{18}{c}}
			agua & agua & agua & agua & agua & agua & agua & agua & agua & agua & agua & agua & agua & agua & agua & agua & agua & agua  \\
			agua & agua & agua & agua & agua & agua & agua & agua & agua & agua & agua & agua & agua & agua & agua & agua & agua & agua  \\
			\hline
		\end{tabular}
	\caption{tabla muy ancha}
	\end{table}
\end{landscape}	
	
	
\pagenumbering{roman}
	
	
Esta oscilación fue descrita por primera vez por el astrónomo, geógrafo y matemático griego Hiparco de Nicea 
que vivió entre los años 190 a.C. y 120 a.C. y fue el tercer movimiento de la Tierra en ser detectado.
\begin{enumerate}[topsep=1cm,partopsep=0pt,parsep=0pt,itemsep=0pt]
	\item algo 
	\item hola
	\item Esta oscilación fue descrita por primera vez por el astrónomo, geógrafo y matemático griego Hiparco de Nicea 
	
	que vivió entre los años 190 a.C. y 120 a.C. y fue el tercer movimiento de la Tierra en ser detectado.
	
	\item mundo
	
\end{enumerate}

\noindent texto texto  texto texto texto texto texto texto texto texto

\begin{enumerate}[leftmargin=*]
	\item algo 
	\item hola
	\item Esta oscilación fue descrita por primera vez por el astrónomo, geógrafo y matemático griego Hiparco de Nicea 
	
	que vivió entre los años 190 a.C. y 120 a.C. y fue el tercer movimiento de la Tierra en ser detectado.
	
	\item mundo
	
\end{enumerate}

\begin{enumerate}[labelsep=*]
	\item algo 
	\item hola
	\item Esta oscilación fue descrita por primera vez por el astrónomo, geógrafo y matemático griego Hiparco de Nicea 
	
	que vivió entre los años 190 a.C. y 120 a.C. y fue el tercer movimiento de la Tierra en ser detectado.
	
	\item mundo
	
\end{enumerate}

\begin{enumerate}[label=\Roman*))]
	\item algo 
	\item hola
	\item Esta oscilación fue descrita por primera vez por el astrónomo, geógrafo y matemático griego Hiparco de Nicea 
	
	que vivió entre los años 190 a.C. y 120 a.C. y fue el tercer movimiento de la Tierra en ser detectado.
	
	\item mundo
	
\end{enumerate}

\begin{enumerate}[label=(\alph*)]
	\item algo 
	\item hola
	\item Esta oscilación fue descrita por primera vez por el astrónomo, geógrafo y matemático griego Hiparco de Nicea 
	
	que vivió entre los años 190 a.C. y 120 a.C. y fue el tercer movimiento de la Tierra en ser detectado.
	
	\item mundo
	
\end{enumerate}


\begin{enumerate}[label=\arabic*.-]
	\item algo 
	\item hola
	\item Esta oscilación fue descrita por primera vez por el astrónomo, geógrafo y matemático griego Hiparco de Nicea 
	
	que vivió entre los años 190 a.C. y 120 a.C. y fue el tercer movimiento de la Tierra en ser detectado.
	
	\item mundo
	
\end{enumerate}

\begin{enumerate}[start=1001,label=\Roman*)]
	\item algo 
	\item hola
	\item Esta oscilación fue descrita por primera vez por el astrónomo, geógrafo y matemático griego Hiparco de Nicea 
	
	que vivió entre los años 190 a.C. y 120 a.C. y fue el tercer movimiento de la Tierra en ser detectado.
	
	\item mundo
	
\end{enumerate}

texto texto texto texto

\begin{enumerate}[resume*]
	\item algo 
	\item hola
	\item Esta oscilación fue descrita por primera vez por el astrónomo, geógrafo y matemático griego Hiparco de Nicea 
	
	que vivió entre los años 190 a.C. y 120 a.C. y fue el tercer movimiento de la Tierra en ser detectado.
	
	\item mundo
	
\end{enumerate}

\begin{enumerate}
	\item algo 
	\item hola
	\item Esta oscilación fue descrita por primera vez por el astrónomo, geógrafo y matemático griego Hiparco de Nicea 
	
	que vivió entre los años 190 a.C. y 120 a.C. y fue el tercer movimiento de la Tierra en ser detectado.
	
	\item mundo
	
\end{enumerate}
	
	
\begin{enumerate}
	\item algo 
	\item hola
	\item Esta oscilación fue descrita por primera vez por el astrónomo, geógrafo y matemático griego Hiparco de Nicea 
	
	que vivió entre los años 190 a.C. y 120 a.C. y fue el tercer movimiento de la Tierra en ser detectado.
\begin{enumerate}
	\item algo 
	\item hola
	\item Esta oscilación fue descrita por primera vez por el astrónomo, geógrafo y matemático griego Hiparco de Nicea 
	
	que vivió entre los años 190 a.C. y 120 a.C. y fue el tercer movimiento de la Tierra en ser detectado.
	
	\item mundo
	
\end{enumerate}
	
	\item mundo
	
\end{enumerate}


\begin{enumerate}
	\item algo 
	\item hola
	\item Esta oscilación fue descrita por primera vez por el astrónomo, geógrafo y matemático griego Hiparco de Nicea 
	
	que vivió entre los años 190 a.C. y 120 a.C. y fue el tercer movimiento de la Tierra en ser detectado.
	
	\item mundo
	
\end{enumerate}
	
	
	
	
	\url{www.google.com}
	
\ \\[2cm]
	
\shadowbox{\parbox{6cm}{Movimiento de precesión de los equinoccios
	
	Este es el movimiento que describe el eje inclinado de la tierra de forma circular.
	
	Más concretamente, es el movimiento que hace el polo norte terrestre respecto al punto central de la elipse 
	que describe la Tierra en el movimiento de translación.}}

\doublebox{\parbox{6cm}{Movimiento de precesión de los equinoccios
		
		Este es el movimiento que describe el eje inclinado de la tierra de forma circular.
		
		Más concretamente, es el movimiento que hace el polo norte terrestre respecto al punto central de la elipse 
		que describe la Tierra en el movimiento de translación.}}
	
	
\cornersize*{13mm}
\ovalbox{\parbox{6cm}{Movimiento de precesión de los equinoccios
		
		Este es el movimiento que describe el eje inclinado de la tierra de forma circular.
		
		Más concretamente, es el movimiento que hace el polo norte terrestre respecto al punto central de la elipse 
		que describe la Tierra en el movimiento de translación.}}
	
	\Ovalbox{\parbox{6cm}{Movimiento de precesión de los equinoccios
			
			Este es el movimiento que describe el eje inclinado de la tierra de forma circular.
			
			Más concretamente, es el movimiento que hace el polo norte terrestre respecto al punto central de la elipse 
			que describe la Tierra en el movimiento de translación.}}
	
	
\chapter{mi teorema cero}
\pagenumbering{arabic}
\begin{teo}
	esto es un teorema
\end{teo}
\renewcommand{\qedsymbol}{$\heartsuit$}
\begin{proof}[Demostración]
esto es una prueba	 esto es una prueba esto es una prueba esto es una prueba esto es una prueba esto es una prueba
\end{proof} 
	
\begin{cor}
esto es un corolario	esto es un corolarioesto es un corolario esto es un corolario esto es un corolario esto es un corolario esto es un corolario esto es un corolario
\end{cor}

\begin{prop}
	esto es una proposición
\end{prop}

\begin{ej}
	esto es un ejemplo
\end{ej}

\begin{ej}
	otro ejemplo
\end{ej}

\begin{ejr}
	un ejercicio de aplicación
\end{ejr}	
	
	
	
	
	1em=ancho de la letra M
	
	1mu=1/12 em
	
$$
\sqrt[\uproot{6}\leftroot{8}\frac{n}{2}]{\frac{1}{\sum\limits_{i=1}^n}a_i}
$$	
	
$$
x+y\boxed{+y}
$$

$$
\overleftarrow{abc}\qquad\underleftrightarrow{xyz}
$$

$$
A\xleftarrow[r_0]{a+b+c}CC
$$

$$
\overset{\mathrm{def}}{=}\quad \overset{\circ}{A}
$$

$$
\underset{--}{B}--
$$

texto texto $\dfrac{a}{b}$ texto $\dbinom{n}{r}$ $\binom{r}{n}$

$$
\tfrac{a}{b}\quad\tbinom{r}{n}\binom{r}{n}
$$

$$
\Bigl(\sum_{i=1}^na_i\Bigr)
$$


%\DeclareMathOperator{\sen}{sen}

%\DeclareMathOperator*{\algo}{algo}
$$
\sen \alpha+\cos\beta
$$

$$
\algo_{x\downarrow0}\sqrt{x}=0
$$

$$
\sum_{\substack{i=1\\i\neq555}}^{n_0} a_i
$$

$$
\sum_{\begin{subarray}{l}i=1\\i\neq555\end{subarray}}^{n_0} a_i
$$

$$
\sideset{_a^b}{_c^d}\sum
$$

$$
\int\int f(x,y) d(x,y)\quad \iint f(\bar{x})d\bar{x}
$$

$$
\iiint f(z)dz\quad\iiiint f(h)dh\quad \oint f(r)dr
$$

$$
\idotsint_{R^n} f(x)dx
$$

$$
\mathbb{ABCDEFGHIJKLMNOPQRSTUVWXYZ}
$$

$$
\mathfrak{ABCDEFGHIJKLMNOPQRSTUVWXYZ}
$$




	
	
\newpage

\tableofcontents	
	
\setcounter{chapter}{5}	

\chapter{mi primer capítulo}
$$
\frac{\partial f}{\partial x}+\frac{\partial g}{\partial y}= ....
$$

$$
\deri{f}{x}+\deri{g}{y}+\deri{h}{z}
$$

texto

\begin{ej}
	esto es un ejemplo
\end{ej}

\begin{ej}
	otro ejemplo
\end{ej}

\begin{ejr}
	un ejercicio de aplicación
\end{ejr}	

{\bfseries\Large Algo}

\textbf{algo importante}  \negrita{IMPORTANTE}

$\Longrightarrow$

$\fd$


\section{sección primera}
Seguramente lo habrás aprendido en la escuela primaria: la Tierra describe $\fd$ una órbita elíptica alrededor del Sol.

Este recorrido, que se conoce como movimiento de traslación, le toma al planeta unos 365 días 
(más 5 horas, 45 minutos y 46 segundos). {\impo nota}

El otro movimiento que te enseñaron es el de rotación: la Tierra gira en torno a su propio eje.

Este giro sobre sí misma le $\fd$ toma aproximadamente un día (23 horas, 56 minutos 4,1 segundos, para ser exactos). 

Sin embargo, estos no son los únicos movimientos que hace la Tierra.

Te contamos —o recordamos— cuáles son los otros tres, también importantes, que ejecuta el planeta.


\renewcommand{\thesection}{\arabic{chapter}--\arabic{section}}
\addtocounter{section}{20}

\section{sección segunda}
	
Seguramente lo habrás aprendido en la escuela primaria: la Tierra describe una órbita elíptica alrededor del Sol.

Este recorrido, que se conoce como movimiento de traslación, le toma al planeta unos 365 días 
(más 5 horas, 45 minutos y 46 segundos).

El otro movimiento que te enseñaron es el de rotación: la Tierra gira en torno a su propio eje.

Este giro sobre sí misma le toma aproximadamente un día (23 horas, 56 minutos 4,1 segundos, para ser exactos). 

Sin embargo, estos no son los únicos movimientos que hace la Tierra.

Te contamos —o recordamos— cuáles son los otros tres, también importantes, que ejecuta el planeta.

Seguramente lo habrás aprendido en la escuela primaria: la Tierra describe una órbita elíptica alrededor del Sol.

Este recorrido, que se conoce como movimiento de traslación, le toma al planeta unos 365 días 
(más 5 horas, 45 minutos y 46 segundos).

El otro movimiento que te enseñaron es el de rotación: la Tierra gira en torno a su propio eje.

Este giro sobre sí misma le toma aproximadamente un día (23 horas, 56 minutos 4,1 segundos, para ser exactos). 

Sin embargo, estos no son los únicos movimientos que hace la Tierra.

Te contamos —o recordamos— cuáles son los otros tres, también importantes, que ejecuta el planeta.\\

\cite{ve14}

\cite[cap. 5]{ve14}

\cite<ver>[cap. 5]{ve14}

\citeNP{ve14}

\citeNP[cap. 5]{ve14}

\citeNP<ver>[cap. 5]{ve14}

\citeA{ve14}

\citeA[cap. 5]{ve14}

\citeA<ver>[cap. 5]{ve14}

\nocite{*}
\bibliographystyle{apacite}
\bibliography{biblio}
	
\end{document}