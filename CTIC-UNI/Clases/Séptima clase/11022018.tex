\documentclass[a4paper]{book}
\usepackage[utf8]{inputenc}
\usepackage[spanish]{babel}
\usepackage[x11names]{xcolor}
\usepackage{titlesec}
\usepackage{lipsum}

\setcounter{secnumdepth}{2}
\setcounter{tocdepth}{2}

\titleformat{name=\section}[hang]
{\color{green}\filleft\titlerule
\sffamily\Huge}
{\color{orange}Sección\space\thesection}
{6pt}
{}

\titlespacing*{\section}
{-1cm}
{0.5cm}
{0.5cm}
%\usepackage[titles]{tocloft}%con title hereda lo de titlesec
\usepackage{tocloft}
\setlength{\cftbeforetoctitleskip}{-2cm}
\setlength{\cftaftertoctitleskip}{5cm}
\renewcommand{\cfttoctitlefont}{
\ttfamily\hfil}% Se va al extremo derecho, hfill es mas debil que hfill
\renewcommand{\cftaftertoctitle}{\hfill}
\renewcommand{\cftdot}{\_}
%\renewcommand{\cftdotsep}{10}%Está en unidades matemáticas, 1era manera
\renewcommand{\cftdotsep}{\cftnodots}

\setlength{\cftbeforechapskip}{2cm}
\setlength{\cftchapindent}{0pt}
\setlength{\cftsecindent}{0pt}
\setlength{\cftsubsecindent}{1em}% 1em es una separacion relativamente pequeña.
\setlength{\cftsubsubsecindent}{2em}
\begin{document}
\pagenumbering{roman}
\tableofcontents
\newpage
\begingroup
\renewcommand*{\addvspace}[1]{}%lo redefine a 0, no existe
\listoffigures
\newpage
\listoftables
\endgroup

\listoffigures
\newpage
\listoftables

\chapter*{Introducción}
\addcontentsline{toc}{chapter}{Introducción}%la palabra que va.

\lipsum[2]
\chapter{Primer
cap}\pagenumbering{arabic}
Seguramente lo habrás aprendido en la escuela
\lipsum[1-3]

\subsection{subsección}
\subsection*{subsección}
\lipsum[1-8]

%rule debeben del coamndo \hbox
\addtocontents{toc}{\protect
\rule{6cm}{2pt}\par}

\addtocontents{toc}{\protect\centering
ANEXO\par}	%protect evita que se expanda

\appendix
\chapter{Primer
cap}\pagenumbering{arabic}
%\section{La primera sección (cap
%1)}
%\subsection{subsección primera
%[sección 1(cap 1)]}
\end{document}

Existe el paquete titletoc, también hay un paquete nuevo.

Tenemos un chapter sin numeración.

El archivo .toc es un archivo fino.

el comando par indica que pasa a la siguiente linea

cft, contenido figuras  tablas


{\cftXfont {\cftXpresnum SNUM\cftXaftersnum\hfil}%
	\cftXaftersnumb TITLE}%
{\cftXleader}{\cftXpagefont PAGE}\cftXafterpnum\par%%sin numeracio


{\cftXfont TITLE}{\cftXleader}{\cftXpagefont PAGE}%
\cftXafterpnum\par%con numeracion, los leader son los puntos