%frame es mas general
%slide
%diapositiva
\documentclass[utf8,spanish,
xcolor={x11names, table}]{beamer}
\usepackage{lipsum}
\usepackage{ragged2e}%Paquete para justificar
\uselanguage{spanish}
\languagepath{spanish}
%\spanishdatedel%, no compila porque es del paquete spanish. 
\usetheme{CambridgeUS}
%\usetheme{Warsaw}

\title{Mi primera exposición con beamer}
\subtitle{algo básico}
\author{Johm Smith}
\institute[UNI]{Universidad Nacional de Ingeniería}% primer slide de color negro, en power dot existe una opcion
\date{11-junio-2018}
% Para jerarquizar se coloca fuera de los frames.

\begin{document}

\maketitle
\begin{frame}{contenido de mi expo}
	\tableofcontents%beamer comenzo como algo simple, tableofcontents tiene mas opciones.
\end{frame}



\section{Comenzando beamer}
\subsection{Parte 1}
\subsection{Parte 2}

\section{Creando overlays}

\begin{frame}{mi primera diapo}{hoy lunes 11}
Hola mundo.\today
\end{frame}

\begin{frame}[t]
\today%se empieza a llenar desde la parte de arriba
\end{frame}

\begin{frame}[plain]
	\hspace*{-1cm}\includegraphics[width=128mm,height=96mm]{uni}
\end{frame}

\begin{frame}[fragile]
	\begin{verbatim}
	hola mundo hoy día
	\end{verbatim}
\end{frame}

\begin{frame}
	hola mundo \pause
	
	hola2 \pause
	
	hola3
\end{frame}

%No se estila mostrar la bibliografia
\begin{frame}{Bibliografía}
\begin{thebibliography}{9}%[allowframebreaks]
\bibitem{Kop04}
Kopka, Helmut; Daly, Patrick W.
\newblock {\em Guide to \LaTeX}.
4th ed.
\newblock Pearson Education,
Inc., 2004.
\end{thebibliography}
\end{frame}

\begin{frame}{Listas}
\begin{enumerate}[A)]%Cambio con respecto a texlive 2017.
	\item[1)] hola
	\item algo
	\item hoy
\end{enumerate}

\begin{itemize}
	\item hola2
	\item algo2
	\item hoy2
\end{itemize}

\begin{description}[MMMMMM] %hi%acabar temprano, esta metido en un caja cuyo ancho es 6m
	\item [casa] Movimiento de precesión de los equinoccios
	
	\item [mesa] Movimiento de precesión de los equinoccios Movimiento de precesión de los equinoccios
	
	\item [silla] hola
\end{description}
\end{frame}

\begin{frame}
	hola mundo \alert{hoy} día\newline%\\
	
	\begin{alertenv}
		esto es importante
	\end{alertenv}
\end{frame}

\begin{frame}
\justifying%justificar textos en beamer, por defecto se alinea a la derecha, se puede declarar arriba para justificar todas las diapoitivas
\lipsum[4]
\end{frame}

\begin{frame}
	\begin{block}{Mi ejemplo}
		esto es un bloque
	\end{block}

	\begin{exampleblock}{algo}
		esto es un bloque importante%estos bloques se resaltan
	\end{exampleblock}

	\begin{alertblock}{algo más}
	esto es un bloque de alerta
	\end{alertblock}
\end{frame}

\begin{frame}
	\begin{example}[de las cortezas cerebrales]
		esto es algo
	\end{example}
\end{frame}
\end{document}
el spanish no es del paquete babel, beamer lo usa como traducción.
% dar clic y muestra la acción copiar

%en la versión 2016, cambbie \begin{frame}\end{frame}
para no perder la caldad de la letra 128 mm x 96 mm, hay que tener en cuenta al insertar imágenes, en el estándar.

En la pc2 se debe reescalar, minipage.

Cascales

alowframebreaks

existen como cuatro coma

ndos para hacer overlides

\LaTeX usa el paqete amsthm, , con los nuevos ambientes, ya no es necesario crear nuevos paquetes.


Hay temas fuertes de matematicas aplicadas a la biologia, hay ecuaciones integrales, ecuaciones en derivadas parciales.