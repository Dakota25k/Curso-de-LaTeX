%%\documentclass[utf8,spanish, xcolor={x11names,table},14pt]{beamer}
\documentclass[utf8,spanish, xcolor={x11names,table},14pt, handout]{beamer}

\uselanguage{spanish}
\languagepath{spanish}

%\usetheme{Warsaw}
%\usetheme{CambridgeUS}
\usetheme{AnnArbor}
\usecolortheme{crane}
\useinnertheme{rectangles}
\useoutertheme[hooks]{tree}

\usefonttheme[onlymath]{serif}

\usepackage{ragged2e}


\title{Mi primera exposición con beamer}
\subtitle{algo básico}
\author{John Smith}
\institute[UNI]{Universidad Nacional de Ingeniería}
\date{11-junio-2018}

\begin{document}
\maketitle

\begin{frame}{contenido de mi expo}\transboxin
  \tableofcontents
\end{frame}

\section{Comenzando beamer}
\subsection{parte 1}
\begin{frame}{mi primera diapo}{hoy lunes 11}\transglitter
  hola mundo
  $$
  \int_a^b\sum_{i=1}^n f(a_i)\,\mathrm{d}x_i
  $$
\end{frame}

\begin{frame}[t]
  hola mundo
\end{frame}

\begin{frame}[b]
  hola mundo
\end{frame}

\begin{frame}[plain]
  \hspace*{-1cm}\includegraphics[width=128mm, height=96mm]{uni}
\end{frame}

\subsection{parte 2}
\begin{frame}[fragile]
  \begin{verbatim}
    hola    mundo hoy dia
  \end{verbatim}
\end{frame}


\section{creando overlays-básico}
\begin{frame}
  hola mundo \pause
  
  hola2 \pause
  
  hola3
\end{frame}

\begin{frame}{Bibliografía}\transblindsvertical
\begin{thebibliography}{9}
\bibitem{Kop04}
Kopka, Helmut; Daly, Patrick W.
\newblock {\em Guide to \LaTeX}. 4th ed.
\newblock Pearson Education, Inc., 2004.
\end{thebibliography}
\end{frame}

\begin{frame}{Listas}\transboxin
  \begin{enumerate}[1)]
    \item hola
    \item algo
    \item hoy
  \end{enumerate}
  
  \begin{itemize}
    \item hola2
    \item algo2
    \item hoy2
  \end{itemize}
  
  \begin{description}[MMMMMM]
    \item[casa] Movimiento de precesión de los equinoccios
    
    \item[mesa] Movimiento de precesión de los equinoccios
        Movimiento de precesión de los equinoccios
        
    \item[silla] hola
  \end{description}
\end{frame}

\begin{frame}\transglitter
  hola mundo \alert{hoy} día\newline
  
  \begin{alertenv}
    esto es importante
  \end{alertenv}
\end{frame}

\begin{frame}\transblindsvertical
\justifying
Seguramente lo habrás aprendido en la escuela primaria: la Tierra describe una órbita elíptica alrededor del Sol.
Este recorrido, que se conoce como movimiento de traslación, le toma al planeta unos 365 días
(más 5 horas, 45 minutos y 46 segundos).\newline

El otro movimiento que te enseñaron es el de rotación: la Tierra gira en torno a su propio eje.\newline

Este giro sobre sí misma le toma aproximadamente un día (23 horas, 56 minutos 4,1 segundos, para ser exactos).

Sin embargo, estos no son los únicos movimientos que hace la Tierra.

Te contamos —o recordamos— cuáles son los otros tres, también importantes, que ejecuta el planeta.
\end{frame}

\begin{frame}\transsplithorizontalin
  \begin{block}{}
    esto es un bloque
  \end{block}
  
  \begin{exampleblock}{Mi ejemplo}
    Para mis ejemplos
  \end{exampleblock}
  
  \begin{alertblock}{algo}
    esto es un bloque importante
  \end{alertblock}
  Te contamos —o recordamos— cuáles son los otros tres, también importantes, que ejecuta el planeta.
\end{frame}

\begin{frame}\transsplithorizontalin
  \begin{example}[de las cortezas cerebrales]
    esto es algo
  \end{example}
\end{frame}

\begin{frame}\transboxin
\begin{columns}[T]
\begin{column}{5cm}
  hola mundo hoy día
\end{column}
\begin{column}{5cm}
  Este es el movimiento que describe el eje inclinado de la tierra de forma circular.
\end{column}
\end{columns}
\end{frame}



\end{document}
