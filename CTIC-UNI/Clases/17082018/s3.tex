% arara: pdflatex
% arara: bibtex
% arara: pdflatex
% arara: pdflatex
\documentclass[a4paper]{report}

\usepackage[dvipsnames]{color}

\newcommand{\parcial}[2]{\frac{\partial #1}{\partial #2}}


%\usepackage[leqno,intlimits]{amsmath}
\usepackage{amsmath}
%Va después de amsmath
\numberwithin{equation}{section}

\DeclareMathOperator{\sen}{sen}
\DeclareMathOperator*{\algo}{sen}%con limites inferiores
\usepackage{showframe}

\usepackage{amssymb}
\usepackage{amsthm}

\theoremstyle{definition}

\newtheorem{defi}{Definición}[chapter]
\newtheorem{eje}[defi]{Ejemplo}
\swapnumbers
\theoremstyle{plain}
\newtheorem{teo}{Teorema}[section] % va el nombre del contador
\theoremstyle{remark}
\newtheorem{prop}[teo]{Proposición}% va a compartir el contador de teorema
\newtheorem*{sol}{Solución} % la versión asterisco no tiene contador

\renewcommand{\proofname}{\bfseries\upshape Demostración} %Interesante
\renewcommand{\qedsymbol}{$\bullet$}

\begin{document}
\chapter{Mi primer cap}

\section{Entornos}

\begin{defi}
	definición de algo
\end{defi}

\addtocounter{teo}{3}

\begin{teo}
	Esto es un teorema
\end{teo}

\begin{proof}
Es una demostración.
\end{proof}

\begin{prop}
Esto es una proposición

\begin{sol}
text
\end{sol}
\end{prop}



\chapter{MIS ENTORNOS TIPO TEOREMA}



\newpage

\begin{equation*}
a+b+c
\end{equation*}

\begin{multline}
a+b+c a+b+c a+b+c a+b+c a+b+c a+b+c \\
a+b+c a+b+c a+b+clasea+b+clasea+b+clasea+b\\
a+b+c a+b+c a+b+c a+b+c a+b+c 
\end{multline}

\begin{multline*}
a+b+c a+b+c a+b+c a+b+c a+b+c a+b+c \\
a+b+c a+b+c a+b+clasea+b+clasea+b+clasea+b\\
a+b+c a+b+c a+b+c a+b+c a+b+c 
\end{multline*}

\begin{equation}
\begin{split}
a+b+c+a+b+a+b+c+a=\alpha\\
+b+c+a+b+c+a+b=\theta% se usa en entornos numerados
\end{split}
\end{equation}

\begin{gather}
a+g+c+a+b+c+a+b+c+a+b+c+a+b\\
a+b=c\\
\sum a_i = f\\
l+q
\end{gather}

\begin{gather*}
a+g+c+a+b+c+a+b+c+a+b+c+a+b\\
a+b=c\\
\sum a_i = f\\
l+q
\end{gather*}

\begin{align}\label{eq:3}
{\left(a+b\right)}^{2}&=\left(a+b\right)\left(a+b\right)\\
											&=a^{2}+ab+ba+b^{2}\notag\\
											&=a^{2}+ab+ab+b^{2}\notag\\
											&=a^{2}+2ab+b^{2}\label{eq:4}
\end{align}

A partir de la ecuación~\ref{eq:3} y \eqref{eq:4}

\begin{equation}
a^{2}+b^{2}=c^{2}\tag{T. Pitágoras}
\end{equation}

\begin{equation}
a^{2}+b^{2}=c^{2}\tag*{T. Pitágoras}
\end{equation}

\begin{align*}
{\left(a+b\right)}^{2}&=\left(a+b\right)\left(a+b\right)\\
&=a^{2}+ab+ba+b^{2}\\
&=a^{2}+ab+ab+b^{2}\\
&=a^{2}+2ab+b^{2}
\end{align*}


\newpage

\begin{equation}
f(x)=\left\{\begin{gathered}
x+5+x^{4}, x\in A \\
\sqrt{x}+4, x\in A^{c}
\end{gathered}\right.
\end{equation}

\begin{equation}
f(x)=\left\{\begin{aligned}
x+5+x^{4}, &\quad x\in A \\
\sqrt{x}+4, &\quad x\in A^{c}
\end{aligned}\right.
\end{equation}

\[ f=g\mbox{ cuando }\dots h_{\mbox{mesa}} \]%mesa sale muy grande!!!!, pero amsmath tiene el comando \text{}

\[ f=g\text{ cuando }\dots h_{\text{mesa}} \]%\text{} generaliza al \mbox{}

\section{Despedida}

\begin{align}
{\left(a+b\right)}^{2}&=\left(a+b\right)\left(a+b\right)\\
&=a^{2}+ab+ba+b^{2}\notag\\
&=a^{2}+ab+ab+b^{2}\\
\intertext{Por lo tanto}
&=a^{2}+2ab+b^{2}
\end{align}

Por lo tanto

\begin{equation}
f(x)=\begin{cases}
x^{2}, & x\in A \\
x^{3}, & x\in B
\end{cases}
\end{equation}

texto texto texto texto texto texto texto texto \(\left(\begin{smallmatrix}
a & b \\
c & d
\end{smallmatrix}\right)\)%no mueve mucho el espacio interlineal

\[ \begin{matrix}
a & b & c \\
d & f & g \\
n & w & p 
\end{matrix} \]

\[ \begin{bmatrix}
a & b & c \\
d & f & g \\
n & w & p 
\end{bmatrix} \]

\[ \begin{pmatrix}
a & b & c \\
d & f & g \\
n & w & p 
\end{pmatrix} \]

\[ \begin{Bmatrix}
a & b & c \\
d & f & g \\
n & w & p 
\end{Bmatrix} \]

\[ \begin{vmatrix}
a & b & c \\
d & f & g \\
n & w & p 
\end{vmatrix} \]

\[ \begin{Vmatrix}
a & b & c \\
d & f & g \\
n & w & p 
\end{Vmatrix} \]

\[ \begin{Vmatrix}
a & b & c \\
d & f & g \\
\hdotsfor{3}
\end{Vmatrix} \]

\[ \sqrt[\leftroot{-10}n/2]{\frac{1}{\sum\limits_{i=1}^{n}}} \]%no es lo mismo usar displaystyle que limits

\[ \sqrt[\uproot{-10}n/2]{\frac{1}{\sum\limits_{i=1}^{n}}} \]

\[
\boxed{a+b+c=d}
\]

\[
\overleftarrow{a+b+c}\quad
\underleftarrow{p+r+s}
\]

\[
a+b+c\xrightarrow[\text{sumando}]{\quad \text{en} R}d%se acomoda el tamaño de la letra, se acomoda al ancho del contenido del documento.
\]

\[
a+b+c=\xleftarrow{\quad \text{en }R\quad}d
\]

\[
\overset{\circ}{A}\quad \underset{\sim}{B}%Poner símbolos encima de otro
\]

texto texto texto texto texto \(\dfrac{a}{b}\) texto texto 

\[
\tfrac{a}{b}+\tfrac{c}{d}
\]

texto texto \(\binom{c}{d}\) texto $\dbinom{c}{d}$

\[
\tbinom{c}{d}
\]

\[
\biggl(\sum_{i=1}^{n} a_i\bigg)%las versiones left o right de big son una generalización
\]


% \DeclareMathOperator{\sen}{sen}

\[
\sen\alpha
\]

\[
\algo_{i\in A} A_i
\]

\[
\sum_{\substack{n=1\\n\neq 3333}}^{\infty} \sen n\alpha \cos n\alpha
\]

\[
\sum_{\begin{subarray}{l}
	n=1\\
	n\neq 3333
	\end{subarray}} a_n
\]%ahora está alineado hacia la izquierda

\[
\sideset{_{a}^{b}}{_{c}^{d}}\sum%simbolos de Cristofell
\]%checkear este comando

\[
\sideset{_{a}^{b}}{_{c}^{d}}\int%simbolos de Cristofell
\]%checkear este comando

\[
\int\int f(x,y)d(x,y)
\]

\[
\iint f(x,y)d(x,y)
\]

\[
\iiint f(x,y)d(x,y)
\]

\[
\idotsint_{R^{n}} f(x,y)d(x,y)
\]


\newpage

\(\mathbb{NZQIRC}\)

\[
r\in\mathbb{R}
\]

\[
A\in \mathfrak{B}
\]

\newpage

% Paquete \usepackage{amsthm}% hay otros paquetes que se mejoraron



\newpage

\begin{align}
a+b&=c	& p&=q	\\
c&=\alpha & f+g&=h
\end{align}

\begin{align*}
a+b&=c	& p&=q	\\
c&=\alpha & f+g&=h
\end{align*}%Hay un entorno que te permite poner al límite permitido, es flalign o flushalign

\begin{flalign}
a+b&=c	& p&=q	\\
c&=\alpha & f+g&=h
\end{flalign}

\begin{flalign*}
a+b&=c	& p&=q	\\
c&=\alpha & f+g&=h
\end{flalign*}

\[ abc\quad \boldsymbol{abc}\quad\pmb{\alpha} \]

\begin{equation}
\sum_{i=1}^{n} a_i=\dots
\end{equation}

\[
a b
\]

\begin{equation}
\int_{a}^{b} g(y)dy
\end{equation}

\begin{defi}
Esto es una definición en nuestra tercera clase
	
\[
\int f(x) dx
\]
\end{defi}

texto texto texto

\begin{defi}
otra definición
\end{defi}
\[ \frac{\partial f}{\partial x}(x,y)+\frac{\partial f}{\partial y}(x,y) \]

\begin{eje}
otro ejemplo
\end{eje}

\[
\parcial{f}{x} + \parcial{f}{y}
\]

\cite{Baldeon2014}, \cite{Perez2010}
%\nocite{*}
\bibliographystyle{plain}
\bibliography{biblio3}

\end{document}
