\documentclass{report}
\usepackage[T1]{fontenc}
%\usepackage{luximono}
\usepackage[x11names]{xcolor}
\usepackage[utf8]{inputenc}
\usepackage{listings}
%\usepackage{listingsutf8}
%\lstset{inputencoding=utf8/latin1}%se puede declarar comandos globales a todo el código
\usepackage[linktocpage=true,breaklinks=true,colorlinks=true,urlcolor=-Orchid1,pdfstartview=FitH]{hyperref}
%el hidelinks oculta los link en el color, es decir, a blanco negro., es mejor poner, existe un paquete bookmarks
\renewcommand{\lstlistingname}{Programa}
\renewcommand{\lstlistlistingname}{Tabla de programas}
\begin{document}%%No podemos colocar letras con tilde//Cóomo comentar en verbatim.
texto texto \lstinline|\end{document}|%sus demilimatadores deben ser el mismo, pueden ser barras, guiones o llaves.
\chapter{Hola}
%Hemos mostrado comandos macros o comandos en \LaTeX
\tableofcontents
\url{www.google.com}
%%Debe de estar pegado al marco, sin líneas en blanco para que no haya inconvenientes
%teletipo monoespaciado, regla no firmada.
\section{básico}

\begin{lstlisting}
\huge
\textbf{ }
For
If
\LaTeX
\end{lstlisting}

\lstinputlisting[linerange={30-40},language=TeX,caption={copiado de CTAN}, label={pro1},captionpos=b]{picins.sty}

Usando el programa \ref{pro1} se aprende a  poner gráficos al lado de texto.

\lstlistoflistings
\hyperlink{fin}{final del documento}
\begin{lstlisting}[linewidth=8cm,breaklines=true]
#include<stdio.h>
int main(){
	printf("Hola mundo");
	return 0;
}
\end{lstlisting}

\begin{lstlisting}[xleftmargin=5cm]
#include<stdio.h>
int main(){
printf("Hola mundo");
return 0;
}
\end{lstlisting}

\begin{lstlisting}[frame=single,framerule=2pt,backgroundcolor=\color{yellow!40},rulecolor=\color{red}]
#include<stdio.h>
int main(){
printf("Hola mundo");
return 0;
}
\end{lstlisting}

\begin{lstlisting}[mathescape=true]
integral $\int_a^b$
\end{lstlisting}
%la a cambia modo texto con las barras
\begin{lstlisting}[escapechar=|]
hola mec|á|nica
\end{lstlisting}

%la tarea es en artículo.
\begin{lstlisting}[inputencoding=utf8,literate={á}{{\'a}}1{é}{{\'e}}1{ñ}{{\~n}}1{ú}{{\'u}}1]
mecánica té ñandú
\end{lstlisting}
%en powerdot se puede hacer reloj

\section{intermedio}

% Esta es una manera un poco tosca.
%en report la tabla de listings empieza en una nueva página

texto texto texto texto texto texto texto texto texto texto texto texto
%Se estila que los flotantes estén en la parte superior.

\begin{lstlisting}[firstline=2,lastline=4]
#include<stdio.h>
int main(){
printf("Hola mundo");
return 0;
}
\end{lstlisting}
% Modificar las lineas que se muestran.
%Índice de códigos

\begin{lstlisting}[linerange={2-3}]
#include<stdio.h>
int main(){
printf("Hola mundo");
return 0;
}
\end{lstlisting}
%anular los 8 primeros caracteres, para crear .dtx,
\begin{lstlisting}[gobble=8]
#include<stdio.h>
int main(){
printf("Hola mundo");
return 0;
}
\end{lstlisting}
\newpage

\begin{lstlisting}[showtabs=true, showspaces=true]
#include<stdio.h>
	int main(){
	printf("Hola mundo");
	return 0;
}
\end{lstlisting}

\begin{lstlisting}[numbers=left,numberstyle={\scriptsize\color{blue!20}}, numberblanklines=false, firstline=2,firstnumber=2]
#include<stdio.h>
	int main(){
	printf("Hola mundo");
	return 0;
}
\end{lstlisting}


\begin{lstlisting}[caption={Mi primer código pirateado}]
#include<stdio.h>
	int main(){
	printf("Hola mundo");
	return 0;
}
\end{lstlisting}

\section{avanzado}
%\lstinputlisting{picins.sty}
%Cuidado con codificación correcta.
\newpage

\lstinputlisting[linerange={20-35}, language=TeX]{picins.sty}

\newpage

\begin{verbatim}
hola	mundo

hoy día
\end{verbatim}
%También se pueden resaltar palabras que uno desee.
%Lo que se pone itálica son los comentarios.

\begin{lstlisting}[language=C, basicstyle=\ttfamily]
#include <stdio.h>
int main(){
	printf("Hola mundo");
	return 0;
}
\end{lstlisting}
\chapter{Aprendiendo a poner programas
	\texorpdfstring{$\int$}{integral}}

\href{http://www.uni.edu.pe}{Universidad} %universidad esconde 





\ \\[2cm]
%sanserif tiene solo medio y no boldface.
%se puede usar luxymono y se mantendría resaltado.
\hypertarget{fin}{My final}%está enfocado a screen y pantalla.
\end{document}
Con el paquete listings se pueden hacer cosas relativamente estándares, se puede poner colores o manipular para el uso de tildes.

p es para la siguiente pagina, b bottom, t top, h,  here.

lo que hace breaklines=true, con pagebackref aparece una bibliografia con bibtex.