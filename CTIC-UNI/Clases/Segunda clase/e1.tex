\documentclass[12pt,a4paper]{article}
\usepackage[utf8]{inputenc} % para mis tildes
\usepackage[T1]{fontenc}

\usepackage[lmargin=2cm,rmargin=1.5cm,tmargin=2cm,bmargin=2.5cm]{geometry}

\linespread{1.25}


\usepackage[dvipsnames,usenames]{color}

\usepackage[spanish,es-sloppy]{babel}


\pagestyle{empty}

\usepackage{graphicx}
\usepackage{here}


\title{Mi primer documento}
\author{yo mismo\thanks{a LIMA}\\UNI \and y el otro yo\\CTIC}
\date{hoy día}


\begin{document}

\maketitle

\begin{abstract}
 Esto es un resumen Esto es un resumen Esto es un resumen Esto es un resumen Esto es un resumen Esto es un resumen
\end{abstract}

\renewcommand{\spanishabstractname}{Abstract}

\begin{abstract}
	abstract
\end{abstract}

Tenemos el escudo de la UNI en la página \pageref{fig1} y en la página \pageref{fig2} como la Figura \ref{fig1} y \ref{fig2} respectivamente
	
\thispagestyle{empty}

Universidad Nacional de Ingeniería	
\begin{center}
	Universidad Nacional de Ingeniería
\end{center}	

\centerline{Universidad Nacional de Ingeniería}
Universidad Nacional de Ingeniería
	
	
\noindent Seguramente         lo habrás aprendido en la escuela primaria: la Tierra describe una órbita elíptica alrededor del Sol.\\[2cm]

\begin{flushright}
	Universidad Nacional de Ingeniería Universidad Nacional de Ingeniería Universidad Nacional de Ingeniería
\end{flushright}

\begin{flushright}
	El otro movimiento que\\ te enseñaron es el de rotación:\\ la Tierra gira en torno a su\\ propio eje.
\end{flushright}


\textbackslash  \{ \} \%

"estas no son comillas"

``son comillas''


\renewcommand{\spanishcontentsname}{Contenido de mi documento}

\tableofcontents

\listoffigures

\renewcommand{\spanishlisttablename}{Mis tablas}
\listoftables


\section{Tipos de letra por defecto}\label{sec1}

Universidad Nacional de Ingeniería

\textrm{Universidad Nacional de Ingeniería}

{\rmfamily Universidad Nacional de Ingeniería}

\textsf{Universidad Nacional de Ingeniería}

{\sffamily Universidad Nacional de Ingeniería}

\texttt{Universidad Nacional de Ingeniería}

{\ttfamily Universidad Nacional de Ingeniería}

\textup{Universidad Nacional de Ingeniería}

{\upshape Universidad Nacional de Ingeniería}

\textit{Universidad Nacional de Ingeniería}

{\itshape Universidad Nacional de Ingeniería} hola

{\itshape Universidad Nacional de Ingeniería\/} hola

\textsl{Universidad Nacional de Ingeniería}

{\slshape Universidad Nacional de Ingeniería}

\textsc{Universidad Nacional de Ingeniería}

{\scshape Universidad Nacional de Ingeniería}

\textmd{Universidad Nacional de Ingeniería}

{\mdseries Universidad Nacional de Ingeniería}

\textbf{Universidad Nacional de Ingeniería}

{\bfseries Universidad Nacional de Ingeniería}

\emph{Universidad Nacional de Ingeniería}

{\em Universidad Nacional de Ingeniería}

\underline{Universidad Nacional de Ingeniería}

{\Huge Universidad Nacional de Ingeniería}

\underline{\textbf{Universidad Nacional de Ingeniería}}

\underline{\bfseries Universidad Nacional de Ingeniería}

\section[tamaño]{tamaños de letra tamaños de letra tamaños de letra tamaños de letra tamaños de letra tamaños de letra tamaños de letra tamaños de letra tamaños de letra}

{\LARGE Universidad Nacional de Ingeniería}

{\large Universidad Nacional de Ingeniería}

{\tiny Universidad Nacional de Ingeniería}

%\usepackage[dvipsnames,usenames]{color}

\subsection{con color}

\textcolor{Orchid}{Universidad Nacional de Ingeniería}

{\color{Orchid} Universidad Nacional de Ingeniería}




\section{Otras cosas}
veamos otras cosas

\newpage
\noindent Este recorrido, que se conoce como movimiento de traslación, le toma al planeta unos 365 días 
(más 5 horas, 45 minutos y 46 segundos).\newline

El otro movimiento que te enseñaron es el de rotación: la Tierra gira en torno a su propio eje.

\newpage

Este giro sobre sí misma le toma aproximadamente un día (23 horas, 56 minutos 4,1 segundos, para ser exactos). 

\clearpage
\vspace*{4cm}
hola

\vspace{3cm}

\hspace*{3cm} mundo\hspace{4cm} hoy\ \ \ \  día\quad lunes\qquad 4

tierra\\
Sin embargo, estos no son los únicos movimientos que hace la Tierra.

Te contamos —o recordamos— cuáles son los otros tres, también importantes, que ejecuta el planeta.\footnote{donde vivimos}



\enlargethispage{5mm}

Movimiento de precesión de los equinoccios

Este es el movimiento que describe el eje inclinado de la tierra de forma circular.

Más concretamente, es el movimiento que hace el polo norte terrestre respecto al punto central de la elipse 
que describe la Tierra en el movimiento de traslación.

\begin{figure}[H]
	\centering
	\includegraphics[width=3cm]{uni}
	\caption{Esto es el escudo de la UNI 0000}\label{fig1}
\end{figure}

Esta oscilación fue descrita por primera vez por el astrónomo, geógrafo y matemático griego Hiparco de Nicea 
que vivió entre los años 190 a.C. y 120 a.C. y fue el tercer movimiento de la Tierra en ser detectado.

Este bamboleo cíclico en la orientación del eje de rotación de la Tierra demora alrededor de 25.780 años. 

Su duración, no obstante, es relativamente imprecisa porque se ve influida por el movimiento y desplazamiento 
de las placas tectónicas.

¿Qué lo produce? Se genera por fundamentalmente por el momento de fuerza que ejerce el Sol sobre la Tierra.

Movimiento de precesión de los equinoccios

Este es el movimiento que describe el eje inclinado de la tierra de forma circular.

Más concretamente, es el movimiento que hace el polo norte terrestre respecto al punto central de la elipse 
que describe la Tierra en el movimiento de traslación.

Esta oscilación fue descrita por primera vez por el astrónomo, geógrafo y matemático griego Hiparco de Nicea 
que vivió entre los años 190 a.C. y 120 a.C. y fue el tercer movimiento de la Tierra en ser detectado.

Este bamboleo cíclico en la orientación del eje de rotación de la Tierra demora alrededor de 25.780 años. 

Su duración, no obstante, es relativamente imprecisa porque se ve influida por el movimiento y desplazamiento 
de las placas tectónicas.

¿Qué lo produce? Se genera por fundamentalmente por el momento de fuerza que ejerce el Sol sobre la Tierra.

Movimiento de precesión de los equinoccios

Este es el movimiento que describe el eje inclinado de la tierra de forma circular.

Más concretamente, es el movimiento que hace el polo norte terrestre respecto al punto central de la elipse 
que describe la Tierra en el movimiento de traslación.

Esta oscilación fue descrita por primera vez por el astrónomo, geógrafo y matemático griego Hiparco de Nicea 
que vivió entre los años 190 a.C. y 120 a.C. y fue el tercer movimiento de la Tierra en ser detectado.

Este bamboleo cíclico en la orientación del eje de rotación de la Tierra demora alrededor de 25.780 años. 

Su duración, no obstante, es relativamente imprecisa porque se ve influida por el movimiento y desplazamiento 
de las placas tectónicas.

¿Qué lo produce? Se genera por fundamentalmente por el momento de fuerza que ejerce el Sol sobre la Tierra.	

Por lo visto en la Sección \ref{sec1} cambiamos letras de la página \pageref{sec1}

%\includegraphics{uni}

\newpage

\includegraphics[width=4cm]{uni}

\includegraphics[height=3cm]{uni}

\includegraphics[width=4cm,height=2cm]{uni}

\begin{center}
\includegraphics[scale=0.1]{uni}
\end{center}

\begin{figure}[H]
	\centering
	\includegraphics[width=3cm]{uni}
	\caption[Escudo UNI]{Esto es el escudo de la UNI Esto es el escudo de la UNI Esto es el escudo de la UNI Esto es el escudo de la UNI Esto es el escudo de la UNI Esto es el escudo de la UNI}\label{fig2}
\end{figure}

Esta oscilación fue descrita por primera vez por el astrónomo, geógrafo y matemático griego Hiparco de Nicea 

\begin{enumerate}
	\item hola
	\item mundo
	\begin{enumerate}
		\item texto1
		\item texto2
		\begin{enumerate}
			\item hoy
		\end{enumerate}
		\item texto3
	\end{enumerate}
	\item hoy
	\item día
\end{enumerate}

\begin{itemize}
	\item hola 
	\item mundo
	\begin{enumerate}
		\item abc
		\item cde
	\end{enumerate}	
\end{itemize}

\begin{description}
	\item[casa] es un lugar donde vivimos es un lugar donde vivimos es un lugar donde vivimos es un lugar donde vivimos es un lugar donde vivimos es un lugar donde vivimos
	
	es un lugar donde vivimos es un lugar donde vivimos es un lugar donde vivimos
	\item[silla] donde nos sentamos\footnote{esto está en el pie de página}
\end{description}

\ \\[2cm]

\begin{tabular}{lcr}
	Mesa & Silla & Casa \\
	mesaaaaaa & silla & casa \\
	MESAsss & SILLA & CASA
\end{tabular}

\ \\[2cm]


\begin{tabular}{|l|c|r|}
	Mesa & Silla & Casa \\
	mesaaaaaa & silla & casa \\
	MESAsss & SILLA & CASA
\end{tabular}

\ \\[2cm]


\begin{tabular}{|l|c|r|}
	\hline
	Mesa & Silla & Casa \\
	\hline
	mesaaaaaa & silla & casa \\
    \hline
	MESAsss & SILLA & CASA \\
	\hline
\end{tabular}


\renewcommand{\spanishtablename}{Tablita}

\begin{table}[H]
	\centering
\caption{esto es una tabla 11}
\begin{tabular}{|l|c|r|}
	\hline
	Mesa & Silla & Casa \\
	\hline
	mesaaaaaa & silla & casa \\
	\hline
	MESAsss & SILLA & CASA  CASA CASA CASA CASA\\
	\hline
\end{tabular}
\end{table}

\begin{table}[H]
	\centering
	\caption{esto es una tabla 222222}\label{tab1}
	\begin{tabular}{|l|c|p{3cm}||}
		\hline\hline
		\multicolumn{2}{|c|}{PALABRA} & Casa \\
		\hline
		mesaaaaaa & silla & casa \\
		\cline{2-3}
		MESAsss & SILLA & CASA\newline CASA CASA CASA CASA CASA\\
		\hline
	\end{tabular}
\end{table}

Vemos en la Tabla \ref{tab1} .... lo realizado en \cite[cap. 5]{smi17}

\rule{4cm}{2mm}

\rule{3cm}{3cm}

\fbox{esto es un texto}

\setlength{\fboxrule}{0.5mm}

\fbox{esto es un texto}

\setlength{\fboxsep}{1cm}

\fbox{esto es un texto}

\setlength{\fboxrule}{0.4pt}
\setlength{\fboxsep}{3pt}
\fbox{esto es un texto}


\fbox{\parbox{5cm}{Este bamboleo cíclico en la orientación del eje de rotación de la Tierra demora alrededor de 25.780 años. 

Su duración, no obstante, es relativamente imprecisa porque se ve influida por el movimiento y desplazamiento 
de las placas tectónicas.}}

\ \\[2cm]

\begin{minipage}{5cm}
Este bamboleo cíclico en la orientación del eje de rotación de la Tierra demora alrededor de 25.780 años. 

Su duración, no obstante, es relativamente imprecisa porque se ve influida por el movimiento y desplazamiento 
de las placas tectónicas.
\end{minipage}

texto1 texto2 \raisebox{2mm}{texto3} texto4 \raisebox{-3mm}{texto5} 


\begin{thebibliography}{99}
	\bibitem{ve14} Venero B., Armando. ``Análisis Matemático''. Gemar, 2014.
	
	\bibitem{smi17} Smith, John. \emph{Acerca del agua}. Algo, 2017
\end{thebibliography}

\newpage
hola

\vfill

mundo\hfill hoy

Nombre:\hrulefill

Apellidos:\dotfill

\clearpage


Su duración, no obstante xxx $g(x)=x+z+y$, $y^2$ y $x+y$ es relativamente imprecisa porque se ve influida por el movimiento y desplazamiento desplazamiento
$$
x+y+z=g(x)
$$

Su duración, no obstante xxx $g(x)=x+z+y$, $y^2$ y $x+y$ es relativamente imprecisa porque se ve influida por
$$
x+y+z=g(x)
$$

\begin{equation}\label{ec1}
a+b+c+d=g(x)+h(z)
\end{equation}

Viendo la Ecuación (\ref{ec1}) aprendemos a sumar

\end{document}