\documentclass{article}
\usepackage[spanish, es-sloppy]{babel}
\usepackage[dvipsnames, usenames]{color}
\usepackage{lipsum}
\usepackage{graphicx}
\usepackage{here}
\title{Mi primer documento}
\author{yo mismo\thanks{Lima}\\UNI\and y el otro\\CTIC}
\date{hoy día}
\renewcommand{\spanishlisttablename}{Mis tablas}
\renewcommand{\spanishabstractname}{Abstracto2}
%\renewcommand{\spanishabstractname}{Abstract}
\pagestyle{empty}

\begin{document}
%Leer el Carrasco.
\maketitle
\thispagestyle{empty}

\begin{abstract}
Primer abstracto.
\end{abstract}

\renewcommand{\abstractname}{Abstract}
\begin{abstract}
Segundo abstracto.
\end{abstract}
\renewcommand{\abstractname}{}

\begin{abstract}
Tercer abstracto.
\end{abstract}

\newpage
\vspace*{4cm}
Hola.

\vspace{2cm}

\hspace*{3cm} mundo \hspace{4cm} mundo 2. \ \ \ día.\quad lunes\qquad 4 de junio.

tierra\\ b\medskip a
\enlargethispage{2mm}
\lipsum[2-4]

%Hay dos paquetes más para etiquetas.
Dentro de la etiqueta debe haber palabras clave,  se etiqueta el número.
% Si no hay palabra se usa asterisco y si hay palabra no se asterisco.

\newpage
\begin{flushright}
El otro movimiento que\\ te enseñaron es el de la rotación\\ la Tierra gira en forma a su\\ propio eje. 
\end{flushright}

\textbackslash \{ \} \$
\section[Letra]{Tipos de letra por defecto}
"estas no son comillas"

``son comillas''


Universidad Nacional de Ingeniería
\section[Tamaño]{Tamaños de letra}\label{sec:2}

\textrm{Universidad Nacional de Ingeniería} % Es un comando con argumento.

{\rmfamily Universidad Nacional de Ingeniería} % Este es un comando sin argumento.

%Comparar la letra Serif del sin San Serif o palo seco.
\textsf{Universidad Nacional de Ingeniería}

{\sffamily Universidad Nacional de Ingeniería}

\texttt{Universidad Nacional de Ingeniería}

{\ttfamily Universidad Nacional de Ingeniería} % Cada letra está dentro de un cuadrado de un mismo ancho.

\textup{Universidad Nacional de Ingeniería}

{\upshape Texto}

\textit{Universidad Nacional de Ingeniería}

{\itshape Universidad Nacional de Ingeniería}

\textsl{Universidad Nacional de Ingeniería}

{\slshape Universidad Nacional de Ingeniería}

\textsc{Universidad Nacional de Ingeniería}

\tableofcontents

{\scshape Texto}

\textmd{Texto}

{\mdseries Texto}

\textbf{Texto}

{\bfseries Texto}

\emph{Texto}

{\em Texto}

\section{Otras cosas}

\underline{Universidad Nacional de Ingeniería}

{\Huge Universidad Nacional de Ingeniería}

{\LARGE Universidad Nacional de Ingeniería}

\section{Cosas extras}

{\large Universidad Nacional de Ingeniería}

{\tiny Universidad Nacional de Ingeniería}

\subsection{Con color}

\textcolor{Orchid}{Universidad Nacional de Ingeniería}

{\color{Peach} Universidad Nacional de Ingeniería}

% Es preferible que sea ordenado.

% Se recomienda usar cuando solo es menor a una línea.
%small caps

% Slan quiere decir oblicua.
\newpage


Por lo visto en la sección \ref{sec:2} cambiamos a letras por lo visto en la página \pageref{sec:2}.

\newpage

\includegraphics[width=4cm]{logo.png}%No es necesario colocar extensión.
\includegraphics[height=2cm]{logo}% El ancho o alto se mantiene en proporción con el anterior.
\includegraphics[width=4cm,height=2cm]{logo}

\includegraphics[scale=0.3]{logo}

\newpage

\begin{center}
\includegraphics[scale=0.1]{logo}
\end{center}

\begin{figure}[H]
	\centering
	\includegraphics[width=3cm]{logo}
	\caption[Escudo de la UNI]{Esto es el escudo de la UNI.}\label{fig:1}
\end{figure}

\listoffigures
\listoftables

Tenemos el escudo de la UNI \ref{fig:1} en la página \pageref{fig:1}.
% Figura y no figura
% Tabla y no tabla
\begin{enumerate}
	\item Hola
	\begin{enumerate}
		\item texto1
		\item texto2
		\begin{enumerate}
			\item Nivel 3
			\item Nivel 3 2
			\begin{enumerate}
				\item 1
				\item 2
%				\begin{enumerate}
%					\item a
%					\item b
%					\item c
%				\end{enumerate}
				\item 3
			\end{enumerate}
			\item Nivel 4
		\end{enumerate}
		\item texto3
	\end{enumerate}
	\item mundo
	\item hoy
	\item día
\end{enumerate}

\begin{itemize}
	\item a
	\item b
	\item c
	\begin{itemize}
		\item x
		\item y
		\begin{itemize}
			\item a
			\item c
			\begin{itemize}
				\item 5
				\item 6
%				\begin{itemize}
%					\item you
%				\end{itemize}
				\item 3
			\end{itemize}
			\item d
		\end{itemize}
		\item z
	\end{itemize}
	\item d
\end{itemize}

\begin{enumerate}
	\item A
	\begin{itemize}
		\item a
		\item a
		\begin{enumerate}
			\item x
			\begin{itemize}
				\item y
			\end{itemize}
			\item 
		\end{enumerate} 
	\end{itemize}
	\item 
\end{enumerate}
\newpage

\begin{description}
	\item[casa] Es un lugar donde vivimos.
	
	\lipsum[1]
	
	\item[silla] Donde nos sentamos.\footnote{esto es una nota de página}
	\lipsum[1]
\end{description}
También se puede referenciar los enumerate y los itemize, los description también se pueden, pero es necesario renombrar un comando.

%No usar parámetro opcional en enumerate ni itemize.
%\includegraphics{logouni.eps}
%variorref, cleverref.
% Fuente no es traducción de font.

En el libro de Cascal ver lo que ocurre con una tabla.

Es posible resetear el contador de , en reporte se resetea los contadores de nota de página.

Un tabular es un arreglo o "matriz", estas tablas, las filas\\[2cm]

\renewcommand{\spanishtablename}{Tablita}

\begin{tabular}{lcr}
	Mesa & Silla & Casa \\
	mesa & silla & casa \\
	MESA & SILLA & CASA
\end{tabular}

\ \\[2cm]
\begin{table}[H] %con H se evita que flote
\centering
\caption{Esto es una tabla}% Se estila que el caption esté en la parte superior.
\begin{tabular}{|l|c|r|}
	\hline
	Mesa & Silla & Casa \\
	\hline
	mesa & silla & casa \\
	\hline
	MESA & SILLA & CASA \\
	\hline
\end{tabular}
\end{table}

\begin{table}[H] %con H se evita que flote
	\centering
	\caption{Esto es una tabla n°2}\label{tab:2}% Se estila que el caption esté en la parte superior.
	\begin{tabular}{|l|c|r|p{3cm}||}
		\hline
		\multicolumn{2}{|c|}{palabra} & Casa \\
		\cline{2-3}
		mesa & silla & casa \\
		\hline\hline
		MESA & SILLA & CASA\newline \\
		\hline
	\end{tabular}
\end{table}
% Más adelante veremos el color.El tamaño del cuadro no es precisamente 3pt, sino que puede variar. 0.4pt
% No se acostumbra ahora usar rayas verticales.
Vemos en la Tabla \ref{tab:2} \ldots lo realizado en \cite{V14} o \cite{smi17}.
\begin{thebibliography}{99}
	\bibitem{V14} Venero B., Armando. ``Análisis matemático''. Gemar, 2014.
	\bibitem{smi17} Smith, John. \emph{Acerca del agua}. Algo, 2017.
\end{thebibliography}
\newpage

\rule{2cm}{3cm}

\rule{2cm}{4cm}

\fbox{Este es un texto.}%Esta separación se puede cambiar.

\setlength{\fboxrule}{0.5mm}

\fbox{Otro texto.}

\setlength{\fboxsep}{1cm}

\fbox{esto es un texto}

\setlength{\fboxrule}{0.6pt}% grosor
% La separacion con el texto es mas grande.
\setlength{\fboxsep}{2pt}% separacion
\fbox{Esto es un texto}

%\parbox{5cm}{\lipsum[1]}
%\ \\

\begin{minipage}{5cm}
\lipsum[1]
\end{minipage}
%Ver libro de Pascal
\newpage

texto1 texto2 texto3 \raisebox{2mm}{texto4} texto 5 \raisebox{-2mm}{texto5}

\newpage

hola
\vfill

se va a la ultima linea permitida

\newpage

holi
\hfill
al final de linea

Nombre \hrulefill

Apellidos \dotfill

\clearpage

\lipsum[1] $g(x)=x+y+z$ \lipsum[2] $y^{2}$ y $z=y$.

% El grifo es itálico, 

$$
x+y=g(x)
$$

Cuando llega a la mitad el porcentaje de espacio en modo matemático que cuando no supera el espacio.

El entorno equation sirve para numerar ecuaciones.

\begin{equation}\label{eq:1}
\int
\end{equation}

Mi ecuación es (\ref{eq:1}).%\eqref{eq:1}

\end{document}