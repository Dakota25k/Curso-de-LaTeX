% arara: pdflatex.exe
\documentclass[12pt,a4paer]{article}
\usepackage[utf8]{inputenc}
\usepackage[T1]{fontenc}
\usepackage[lmargin=2cm,rmargin=1.5cm,tmargin=2cm,bmargin=4cm]{geometry}
\usepackage{graphicx}
\usepackage[spanish,es-sloppy]{babel} % No usar así, tiene desventajas en el paquete tikz.
\spanishdatedel
%Con es-sloppy le estamos quitando muchas modifcaciones.
%dialectos
\usepackage{here}

\usepackage[dvipsnames,usenames]{color}

\renewcommand{\spanishcontentsname}{Contenido de mi artículo}
%\linespread{1.5}

\setlength{\parindent}{0pt} %
\setlength{\parskip}{14.4pt} % Espacio interlineal, un comando que guarda esa medida.

\title{Mi primer documento\thanks{Hecho en \LaTeX}}
% El \thanks se usa solo en este caso,l pues existe el comando \footnote{}

\author{Juan Pérez\\UNI\and John Smith\\CTIC}%los guarda en un comando

%\inst, cómo heredar del beamer.
\date{\today} 
%lo que se desea modificar y el valor que se cambia.
%\pagestyle{empty}%Para todo el documento

%\usepackage{lipsum}
% hasta acá es el preámbulo.
\begin{document}
\maketitle

\begin{abstract}
El resumen de mi documento
El resumen de mi documento
El resumen de mi documento
El resumen de mi documento\\
El resumen de mi documento\\
El resumen de mi documento\\
El resumen de mi documento
El resumen de mi documento
\end{abstract}
\begin{center}
	Universidad Nacional de Ingeniería
\end{center}
\centerline{
Universidad Nacional de Ingeniería
}
%\thispagestyle{empty}
\noindent 3 movimientos que hace la Tierra (que no son ni rotación ni traslación) 
y que quizás no conocías

Seguramente lo habrás aprendido en la escuela primaria: la Tierra describe una órbita elíptica alrededor del Sol.\\[2cm]

Este recorrido, que se conoce como movimiento de traslación, le toma al planeta unos 365 días 
(más 5 horas, 45 minutos y 46 segundos).

El otro movimiento que te enseñaron es el de rotación: la Tierra gira en torno a su propio eje.

Este giro sobre sí misma le toma aproximadamente un día (23 horas, 56 minutos 4,1 segundos, para ser exactos). 

Sin embargo, estos no son los únicos movimientos que hace la Tierra.

Te contamos —o recordamos— cuáles son los otros tres, también importantes, que ejecuta el planeta.

\enlargethispage*{15mm} % Hay con asterisco y sin asterisco.

\begin{flushright}
Se genera por fundamentalmente\\ por el momento de fuerza\\ que ejerce la Tierra.
\end{flushright}

Me debes \$ 20. Mi número \#5

El conjunto \{a,b,c\}

``hola''


"estas no son comiillas"


Letra romana recta media

Universidad

\textrm{Universidad}

{
\rmfamily Universidad
}

\textsf{Universidad}

{
\sffamily Universidad
}


\texttt{Universidad}

{
\ttfamily Universidad
}


\textup{Universidad}

{
\upshape Universidad
}

\textit{Universidad}

{
\itshape Universidad\/
}Nacional%Seperación

\textsl{Universidad}

{
\slshape Universidad\/ % Se parece a la letra romana
}


\textsc{Universidad} % No son letras mayúsculas

{\scshape Universidad}

\textmd{Universidad}

{
\mdseries Universidad % que hace punto medio de tamaño
}

\textbf{Universidad}

{
\bfseries Universidad
}

\emph{Universidad}

{
\em Univerisdad
}

\underline{Universidad}

%cuando lo quer se sunbraya es mas pequeño que una linea. paquete uline, eleline (lline)
%beamer por defecto es san serif.

\clearpage

{
\Huge Universidad
}

De la sección~\ref{sec:1} y de la página~\pageref{sec:1}


{
\Large Universidad
}

{
\small Universidad
}

{
\tiny Universidad
}

\newpage

\textcolor{YellowGreen}{Universidad}

{
\color{Red} Universidad
}

{
\colorbox{Yellow}{Universidad}
}

\tableofcontents

\section{Mi primera clase de \LaTeX}\label{sec:1}% Se etiqueta el contador de la sección

\subsection{Entrada}

Mi documento

\subsubsection[Salida]{Salida Salida Salida Salida Salida Salida Salida Salida Salida Salida Salida Salida Salida Salida}%hacerlo con un for

Movimiento
 de precesión de los equinoccios

Este es el movimiento que describe el eje inclinado de la tierra de forma circular.

Más concretamente, es el movimiento que hace el polo norte terrestre respecto al punto central de la elipse 
que describe la Tierra en el movimiento de translación.

Esta oscilación fue descrita por primera vez por el astrónomo, geógrafo y matemático griego Hiparco de Nicea 
que vivió entre los años 190 a.C. y 120 a.C. y fue el tercer movimiento de la Tierra en ser detectado.

Este bamboleo cíclico en la orientación del eje de rotación de la Tierra demora alrededor de 25.780 años. 

Su duración, no obstante, es relativamente imprecisa porque se ve influida por el movimiento y desplazamiento 
de las placas tectónicas.

¿Qué lo produce? Se genera por fundamentalmente por el momento de fuerza que ejerce el Sol sobre la Tierra. 




Movimiento de nutación

Este movimiento se produce por una suerte de vibración del eje polar terrestre.

Esto hace que, durante el movimiento de precesión de los equinoccios, los círculos que se describen no sean 
perfectos sino irregulares.

Es decir, el eje de la Tierra se inclina un poco más o un poco menos respecto a la circunferencia que describe 
durante la precesión.

El movimiento es cíclico y cada uno de los episodios dura algo más de 18 años y medio. Durante este tiempo, 
la variación es de un máximo de 700 metros respecto a la posición inicial.

La nutación fue descubierta por el astrónomo británico James Bradley en 1728.

Varios años después hallaron la causa de este vaivén, cuando cálculos llevados a cabo por distintos científicos 
concluyeron que era producto directo de la atracción gravitatoria de la Luna.


\vspace{4cm} % Funciona cuando hay texto antes de usarlo

Otro texto


\newpage

\vspace*{4cm} %Empezar pagina

texto\hspace{3cm} texto 2

\hspace*{5cm} texto, el asterisco se usa al empezar una nueva línea.\quad texto 3, quad es el espacio de una letra M texto4 \qquad text5, el enspace es la mitad de una quad, \enspace texto

\newpage

a

\

e

\

i


Bamboleo de Chandler

Esta otra irregularidad en la oscilación del eje terrestre fue descubierta en 1891 por el astrónomo estadounidense 
Seth Carlo Chandler y aún hoy sigue siendo un enigma: aunque hay muchas teorías, nadie ha logrado determinar su causa.

El llamado bamboleo de Chandler es un movimiento oscilatorio del eje de rotación de la Tierra.

Este movimiento puede hacer que la tierra se desplace hasta un máximo de 9 metros de la posición esperada en un 
momento en particular.

Su duración es de cerca de 433 días, lo que quiere decir que ese es el tiempo que demora completar una oscilación.

Algunas teorías apuntan a que puede ser provocado por cambios en la temperatura y la salinidad de los océanos así 
como por los cambios en la circulación de los mismos a causa del viento. Otras dicen que por cambios en el clima.




Tomado de:

http://www.bbc.com/mundo/noticias-41943066

Universidad Nacional de Ingeniería

3 movimientos que hace la Tierra (que no son ni rotación ni traslación) 
y que quizás no conocías


Seguramente lo habrás aprendido en la escuela primaria: la Tierra describe una órbita elíptica alrededor del Sol.

Este recorrido, que se conoce como movimiento de traslación, le toma al planeta unos 365 días 
(más 5 horas, 45 minutos y 46 segundos).

El otro movimiento que te enseñaron es el de rotación: la Tierra gira en torno a su propio eje.

Este giro sobre sí misma le toma aproximadamente un día (23 horas, 56 minutos 4,1 segundos, para ser exactos). 

Sin embargo, estos no son los únicos movimientos que hace la Tierra.

Te contamos —o recordamos— cuáles son los otros tres, también importantes, que ejecuta el planeta.





Movimiento de precesión de los equinoccios

Este es el movimiento que describe el eje inclinado de la tierra de forma circular.

Más concretamente, es el movimiento que hace el polo norte terrestre respecto al punto central de la elipse 
que describe la Tierra en el movimiento de translación.

Esta oscilación fue descrita por primera vez por el astrónomo, geógrafo y matemático griego Hiparco de Nicea 
que vivió entre los años 190 a.C. y 120 a.C. y fue el tercer movimiento de la Tierra en ser detectado.

Este bamboleo cíclico en la orientación del eje de rotación de la Tierra demora alrededor de 25.780 años. 

Su duración, no obstante, es relativamente imprecisa porque se ve influida por el movimiento y desplazamiento 
de las placas tectónicas.

¿Qué lo produce? Se genera por fundamentalmente por el momento de fuerza que ejerce el Sol sobre la Tierra. 




Movimiento de nutación

Este movimiento se produce por una suerte de vibración del eje polar terrestre.

Esto hace que, durante el movimiento de precesión de los equinoccios, los círculos que se describen no sean 
perfectos sino irregulares.

Es decir, el eje de la Tierra se inclina un poco más o un poco menos respecto a la circunferencia que describe 
durante la precesión.

El movimiento es cíclico y cada uno de los episodios dura algo más de 18 años y medio. Durante este tiempo, 
la variación es de un máximo de 700 metros respecto a la posición inicial.

La nutación fue descubierta por el astrónomo británico James Bradley en 1728.

Varios años después hallaron la causa de este vaivén, cuando cálculos llevados a cabo por distintos científicos 
concluyeron que era producto directo de la atracción gravitatoria de la Luna.


Bamboleo de Chandler

Esta otra irregularidad en la oscilación del eje terrestre fue descubierta en 1891 por el astrónomo estadounidense 
Seth Carlo Chandler y aún hoy sigue siendo un enigma: aunque hay muchas teorías, nadie ha logrado determinar su causa.

El llamado bamboleo de Chandler es un movimiento oscilatorio del eje de rotación de la Tierra.

Este movimiento puede hacer que la tierra se desplace hasta un máximo de 9 metros de la posición esperada en un 
momento en particular.

Su duración es de cerca de 433 días, lo que quiere decir que ese es el tiempo que demora completar una oscilación.

Algunas teorías apuntan a que puede ser provocado por cambios en la temperatura y la salinidad de los océanos así 
como por los cambios en la circulación de los mismos a causa del viento. Otras dicen que por cambios en el clima.

\listoffigures

\includegraphics[width=6cm]{logo}

\includegraphics[scale=0.05]{logo}



\begin{figure}[H]
\centering
\includegraphics[height=6cm]{logo}
\caption{El escudo de la UNI}\label{fig_1}
\end{figure}

\begin{figure}[H]
	\centering
	\includegraphics[scale=0.1]{logo}
	\caption[El escudo de la UNI al 10\%]{El escudo de la UNI al 10\% El escudo de la UNI al 10\% El escudo de la UNI al 10\% El escudo de la UNI al 10\% El escudo de la UNI al 10\% El escudo de la UNI al 10\% El escudo de la UNI al 10\%}
\end{figure}

Tomado de:
En la Figura~\ref{fig:1}
http://www.bbc.com/mundo/noticias-41943066

\end{document}