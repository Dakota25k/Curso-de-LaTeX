\documentclass[12pt,a4paper]{report}
\usepackage[lmargin=3cm, rmargin=2.5cm, tmargin=2.5cm, bmargin=2.5cm]{geometry}% Se usa para los márgenes.
\usepackage{graphicx}
\usepackage{showframe}%Se pueden poner notas al margen, todo el texto debe estar en el cuadrado del texto.
\usepackage[intlimits]{amsmath}
\numberwithin{equation}{section}
%\usepackage[intlimits, leqno]{amsmath}% esa opción tendrán los límites de integración arriba y abajo

%T1 para letra time, fontenc
% Con tikz se puede adornar la caráctula.
\begin{document}

\begin{titlepage}
\centering %Todo lo que esté en su entorno se centrará.
\bfseries % El alcance es por el entorno que lo contiene.

{\Large UNIVERSIDAD NACIONAL DE INGENIERÍA\\[2mm]

FACULTAD DE CIENCIAS\\[3mm]

ESCUELA PROFESIONAL DE MATEMÁTICA}\\[5mm]

\includegraphics[scale=0.15]{logo}\\[10mm]

presentado por:\\[1cm]

{\LARGE Juan Pérez}\\[5mm]

Asesor por:\\[1cm]

{\Large John Smith}\\[1cm]

UNI -- Perú\\[5mm]

2018
\end{titlepage}

Esto es hoy día

$$
x^2+y^2=z^2\qquad x^{12}+y^{12}=z^{12}
$$

$$
x_2+y_2=z_2\qquad x_{12}+y_{12}=z_{12}
$$

$$
\frac{a}{b}\qquad\frac{x^2+y^2}{z^2}
$$

texto texto2 $\frac{a}{b}$ texto texto texto texto texto texto texto texto % Se hace más pequeño por el espacio interlineal. No modificarlo. Está mal cambiarlo, tipográficamente hablando.

$$
\sqrt{a+b+c}\qquad \sqrt[n]{a+b+c}
$$

$$
\int f(x)dx \qquad \sum  a_i
% Está mal escrito. d se ve como variable, debe ser una romana recta y separada de la inicial.
$$

$$
\int\limits_{a}^{b} f(x)dx \qquad \sum_{i=1}^n a_i
$$

$$
\int_{a}^{b} f(x)dx \qquad \sum\nolimits_{i=1}^n a_i %Modo desplegado.
$$

% El comando limits mueve los límites de integración en la parte superior de la integral.

texto texto texto $\int_{a}^{b} f(x) dx \qquad \sum_{i=1}^n a_i$ texto texto 

$$
i=1,2,...,n (mal)\qquad i=1,2,\ldots,n% Incorrecto vs correcto.
$$

$$
c+d+e+\cdots+z \qquad \vdots\qquad \ddots% Con otro paquete se pueden obtener en la otra diagonal.
$$

$$
\alpha+\beta+\gamma+\varphi+\psi+\Omega%Lamda para un plano. Sigma distinto a la sumatoria.Se llama operador binario en LaTeX, en matemática no necesariamente. Puede generar un espacio extra en caso de solo usar, o de logica e y de logica.
$$

$$
a, b \in V,\qquad a\times b\qquad axb(mal)
$$

$$
f,g\in G ,\qquad f\circ g % Es importante dejar un espacio en blanco.
$$

$$
f,g \in G, f\colon A\to B, g:B\longrightarrow C,
\longrightarrow g\circ f:A\rightarrow C
$$

$$
cos\alpha (mal) \qquad \cos\alpha
$$

$$
\lim_{x\to\infty}\frac{1}{x}=0
$$

%ddot para segunda derivada, paquete vec ṕara flechas

$$
x,\hat{x},\tilde{x},\qquad x+\hat{x}=\tilde{x}
$$

%hat es el circunflejo en la parte superior.

$$
\hat{abc}\qquad \widehat{abc}
$$

$$
\tilde{abc}\qquad \widetilde{abc}
$$
% Podemos manipular los tamaños.

$$
(\frac{a}{b})\qquad \left(\frac{a}{b}\right)\qquad 
$$% Siempre el left y right van en pareja
% Se puede mejor incluso con left right
$$
\left(\frac{a}{b}\right.
$$% Para una regla de correspondencia, en vez de usar paréntsis se puede usar llaves.

$$
f=g \mbox{ cuando } f(x)=g(x) \forall x\in X
$$% El para todo está muy pegado.% mbox lo encierra y lo vuelve letra% para el comando text necesita un paquete.

$$
abc\quad\mathrm{abc}\quad\mathsf{abc}\quad\mathtt{abc}\quad\mathbf{abc}\quad\mathit{abc}\quad\mathcal{ABC}% No es itálica, la otra es romana recta, letra san serif, lo hace romana y lo vuelve negraita, mathcal solo funciona con mayúsculas.
$$

$$% Es como el tabular
\left(\begin{array}{ccc}%Faltan sus delimimitadores.
 a & b & c \\
 d & e & f \\
 g & h & i
\end{array}\right)_{3\times 3} \left(\begin{array}{c}
x \\
y \\
z
\end{array}\right)% Funciona en el modo matemático.
$$

$$
\left[\begin{array}{ccc}
a & b & c \\
d & e & f \\
g & h & i
\end{array}\right]
$$

%$$
%\left{\begin{array}{ccc}
%a & b & c \\
%d & e & f \\
%g & h & i
%\end{array}\right}
%$$

$$
\left\{\begin{array}{ccc}
a & b & c \\
d & e & f \\
g & h & i
\end{array}\right\}
$$

$$
\bar{xy}\qquad \overline{xyz}\quad\underbrace{abc} %overline crea una barra sobre su argumento
$$
underline funciona en matemática como en texto.

$$
\overbrace{3+3+3+3}^{12}\qquad\underbrace{3+3+3+3}_{12}
$$

$$
a+b+c\stackrel{\mathrm{def.}}{=}x
$$

%No usar eqnarray

NO USAR

\begin{eqnarray}% Es un arreglo de tres columnas
a + b + c	& = & b + c \\
			& = & f + g \\
			& = & x + y
\end{eqnarray}% Se alinea respecto a la segunda columna. Se pude manejar la dimensión.
La versión con asterisco hace lo mismo, pero le quita la numeración.

\begin{eqnarray*}% Es un arreglo de tres columnas
a + b + c	& = & b + c \\
& = & f + g \\
& = & x + y
\end{eqnarray*}% Se alinea respecto a la segunda columna. Se pude manejar la dimensión.

Operadores binarios en el sentido de \LaTeX{}

$$
a+b+c=\;\forall a,b\in A
$$

$$
a\; b\: c\,d\!e
$$% Son para afinar los operadores binarios.

texto texto texto texto $\displaystyle\int_a^b f(x)\,\mathrm{d}x$texto texto texto texto texto

Solo va la d y no la x, porque x es variable.
$$\textstyle\int_a^b f(x)\,\mathrm{d}x$$

estilo texto vs estilo desplegado.

$$
x^{x^x}\quad\scriptscriptstyle{abc}\qquad\scriptstyle{abc}%estilo desplegado, scriptstyle scriptscriptstyle
$$

$$
(\frac{a}{b})\quad\big(\frac{a}{b}\big)\quad\Big(\frac{a}{b}\Big)\quad\bigg(\frac{a}{b}\bigg)\quad\Bigg(\frac{a}{b}\Bigg)% Se prefiere el tercero y el cuarto según sea fraccion o integral .
$$
Hay cuatro tamaños disponibles

\begin{equation}
a+b+c=b+c
\end{equation}

\newpage

$$
abcd\quad \boldsymbol{abcd}\quad\mathbf{abc}
$$

$$
abcd\quad \pmb{abcd}% lo mueve a la fuerza. No es una negrita real. No usar
$$

Con \LaTeX{} estándar no existe \texttt{equation} con asterisco.

\begin{equation*}
a+b+c=de
\end{equation*}

\begin{multline}
a+b+c+d+e+g+g+f+s+afs++wetfg\\+w+gs+dfawsd+fsd+=
asg+we\\+w+qtgw+sfs+qg+w=sfg
\end{multline}

\begin{multline*}
a+b+c+d+e+g+g+f+s+afs++wetfg\\+w+gs+dfawsd+fsd+=
asg+we\\+w+qtgw+sfs+qg+w=sfg
\end{multline*}

\begin{equation}
\begin{split}
a+c+s+d=fasdf \\
asrfsdfad
\end{split}
\end{equation}
%starversion o asterisco

\begin{gather}
a+b+c+d+e+ +gsedf \\
f = s \\
a+gsdfas
\end{gather} %lo que se obtiene es centrar

\begin{gather*}
a+b+c+d+e+ +gsedf \\
f = s \\
a+gsdfas
\end{gather*}

\begin{align}
a+b+c & = b+c\\
& = f+g\\
& = x+y % se alinea con respecto al que alineo
\end{align}

\begin{align*}
a+b+c & = b+c & a &=b \\
& = f+g &	c&=d\\
& = x+y &	c&=a+b+d% se alinea con respecto al que alineo
\end{align*}

\begin{flalign}
a+b+c & = b+c\\
& = f+g\\
& = x+y
\end{flalign}


\begin{flalign*}
a+b+c & = b+c\\
& = f+g\\
& = x+y
\end{flalign*}

\begin{align}
\label{eq:1}a &=b000\\
c&=d000 \notag\\
\label{eq:2}c&=a+b+d000
\end{align}

Por las ecuaciones \eqref{eq:1} y \eqref{eq:2}

\begin{align}
a+b+c & = b+c & a&=b \tag{Teorema de Pitágoras}\\
& = f+g & c&=d\\
& = x+y & c&=a+b+d \tag*{T. Pitágoras}
\end{align}

En estas ecuaciones agarra todo el ancho de la línea, para explicar la ecuación es mejor detallarlo afuera del entorno.

$$
f(x)=\left\{\begin{aligned}
ab+d+c&= sdfr \\
a& f+g+ f\\
&=qe
\end{aligned}\right.
$$

$$
f(x)=\left\{\begin{gathered}
ab+d+c=sdfr \\
af+g+ f\\
=qe
\end{gathered}\right.
$$

$$
f(x)=\begin{cases}
x^2&, x\in X \\
x/2&, x\in X^C
\end{cases}
$$

$$
x_{\mbox{longitud}}\qquad x_{\text{longitud}}%sino usar mathrm y es una palabra grande, con \ŧext se hace el tamaño de un subindice.
$$

$$
f=g\text{ cuando }f(x)=g(x),\forall x\in A.
$$
%Trabajar con tcolorbox.

\begin{align}
a+b+c & = b+c\\
& = f+g \\ \intertext{Por lo tanto}%tengo que aprender a usarlo.
& = x+y
\end{align}
% No se debe escribir el por lo tanto dentro de la misma línea y dejar un espacio en blanco.

En gathered no se usa \&.

El eqref se usa cuando se etiqueta las ecuaciones.

Usando las Ecuaciones \eqref{eq:1} y \eqref{eq:2}.

\begin{enumerate}
\item algo
\end{enumerate}

Viendo (1)%Los correctores los destrozan.

\chapter{Mi clase del 5 de junio}

Esto es hoy día.

\begin{equation}
\alpha
\end{equation}

\chapter{Otro capítulo}

\section{Básico}

$$
f(x)=\left\{
x+y
\right\}
$$

texto texto texto, no hagan una matriz dentro de un texto $\left(\begin{smallmatrix} a & b\\ c & d \end{smallmatrix}\right)$ texto texto.

$$
\begin{matrix}
a & b & c \\
d & e & f \\
g & h & i
\end{matrix}
$$

$$
\begin{pmatrix}
a & b & c \\
d & e & f \\
g & h & i
\end{pmatrix}
$$

$$
\begin{bmatrix}
a & b & c \\
d & e & f \\
g & h & i
\end{bmatrix}
$$

$$
\begin{Bmatrix}
a & b & c \\
d & e & f \\
g & h & i
\end{Bmatrix}
$$

$$
\begin{vmatrix}
a & b & c \\
d & e & f \\
g & h & i
\end{vmatrix}
$$

$$
\begin{Vmatrix}
a & b & c \\
d & e & f \\
g & h & i
\end{Vmatrix}
$$% Como norma matricial.

$$
\begin{pmatrix}
a & b & c \\
d & e & f \\
\hdotsfor{3}
\end{pmatrix}
$$

\end{document}
cuando viene en pares hay que usar split
Mañana es la práctica se entrega hasta el medio día.

luego, por lo tanto, haciendo 1 y 2
babel, sen con babel eslopping no existe

En reporte la mayor jerarquía es reporte.