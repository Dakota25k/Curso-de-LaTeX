\documentclass[12pt,a4paper]{report}
\usepackage[utf8]{inputenc}
\usepackage[lmargin=3cm,rmargin=2.5cm,tmargin=2.5cm,bmargin=2.5cm]{geometry}

\usepackage{showframe}

\usepackage{graphicx}

\usepackage[intlimits]{amsmath}
%\usepackage[intlimits,leqno]{amsmath}

\numberwithin{equation}{section}

\begin{document}
\begin{titlepage}
\centering
\bfseries

	
{\Large UNIVERSIDAD NACIONAL DE INGENIERÍA\\[2mm]

FACULTAD DE CIENCIAS\\[3mm]

ESCUELA PROFESIONAL DE MATEMÁTICA}\\[5mm]


\includegraphics[scale=0.13]{uni}\\[10mm]	

presentado por:\\[1cm]

{\LARGE Juan Pérez}\\[5mm]

Asesor:\\[1cm]

{\Large John Smith}\\[1cm]

UNI-PERÚ\\[5mm]

2018
	
\end{titlepage}



\chapter{Mi clase del 5 de junio}

\section{básico}
Esto es hoy día

$$
x^2+y^2=z^2\qquad x^{12}+y^{12}=z^{12}
$$

$$
x_2+y_2=z_2\qquad x_{12}+y_{12}=z_{12}
$$

$$
\frac{a}{b}\qquad\frac{x^2+y^2}{z^2}
$$

texto texto2 $\frac{a}{b}$ texto texto texto texto texto texto texto texto texto texto texto texto texto texto texto texto texto texto texto texto texto texto texto texto texto texto texto texto

$$
\sqrt{a+b+c}\qquad \sqrt[n]{a+b+c}
$$

$$
\int f(x)dx \qquad \sum a_i
$$


$$
\int_{a}^{b} f(x)dx \qquad \sum_{i=1}^n a_i
$$

$$
\int\limits_{a}^{b} f(x)dx \qquad \sum_{i=1}^n a_i
$$

$$
\int_{a}^{b} f(x)dx \qquad \sum\nolimits_{i=1}^n a_i
$$

texto texto texto  $\int_{a}^{b} f(x)dx \qquad \sum_{i=1}^n a_i$ texto texto texto texto  texto texto  texto texto  texto texto  texto texto

$$
i=1,2,...,n (mal)\qquad i=1,2,\ldots,n
$$ 

$$
c+d+e+\cdots+z\qquad \vdots\qquad \ddots
$$

$$
\alpha+\beta+\gamma+\varphi+\psi+\Omega
$$

$$
a,b \in V,\qquad a\times b\qquad axb(mal)
$$

$$
f,g\in G ,\qquad f\circ g
$$

$$
f,g \in G, f:A\to B, g:B\longrightarrow C, \Longrightarrow g\circ f:A\rightarrow C
$$

$$
cos\alpha (mal) \qquad \cos\alpha 
$$

$$
\lim_{x\to\infty}\frac{1}{x}=0
$$

$$
x,\hat{x},\tilde{x},\qquad x+\hat{x}=\tilde{x}
$$

$$
\hat{abc}\qquad \widehat{abc}
$$

$$
\tilde{abc}\qquad \widetilde{abc}
$$

$$
(\frac{a}{b})\qquad \left(\frac{a}{b}\right)
$$

$$
\left(\frac{a}{b}\right.
$$

$$
f=g \mbox{ cuando } f(x)=g(x) \forall x\in X
$$

$$
abc\quad \mathrm{abc}\quad\mathsf{abc}\quad\mathtt{abc}\quad\mathbf{abc}\quad\mathit{abc}\quad\mathcal{ABC}
$$



$$
\left(\begin{array}{ccc}
 a & b & c \\
 d & e & f \\
 g & h & i
\end{array}\right)_{3\times 3} \left(\begin{array}{c}
x \\
y \\
z
\end{array}\right)
$$

$$
\left[\begin{array}{ccc}
a & b & c \\
d & e & f \\
g & h & i
\end{array}\right]
$$

$$
\bar{xy}\qquad \overline{xyz}\quad\underline{abc}
$$

$$
\overbrace{3+3+3+3}^{12}\qquad\underbrace{3+3+3+3}_{12}
$$

$$
a+b+c\stackrel{\mathrm{def.}}{=} x
$$

\begin{equation}
a+b+c=b+c
\end{equation}

NO USAR
\begin{eqnarray}
a+b+c & = & b+c \\
      & = & f+g \\
      & = & x+y
\end{eqnarray}

\begin{eqnarray*}
a+b+c & = & b+c \\
& = & f+g \\
& = & x+y
\end{eqnarray*}

$$
a+b=c\;\forall a,b\in A
$$

$$
a\; b\:c\,d\!e
$$

texto texto texto $\displaystyle\int_a^b f(x)\,\mathrm{d}x$ texto texto texto

$$\textstyle\int_a^b f(x)\,\mathrm{d}x$$

$$
x^{x^x} \quad \scriptscriptstyle{abc}\quad \scriptstyle{abc}
$$

$$
(\frac{a}{b})\quad \big(\frac{a}{b}\big)\quad \Big(\frac{a}{b}\Big)\quad \bigg(\frac{a}{b}\bigg)\quad \Bigg(\frac{a}{b}\Bigg)
$$



\newpage


\section{del paquete amsmath}
$$
abcd\quad \boldsymbol{abcd}\quad\mathbf{abc}
$$

$$
abcd\quad \pmb{abcd}
$$

\begin{equation*}
a+b+c=de
\end{equation*}

\begin{multline}
a+b+c+d+e+g+g+f+s+afs++wetg\\+w+gs+dfawsd+fsd+=asg+we\\+w+qtgw+sfs+qg+w=sfg
\end{multline}

\begin{multline*}
a+b+c+d+e+g+g+f+s+afs++wetg\\+w+gs+dfawsd+fsd+=asg+we\\+w+qtgw+sfs+qg+w=sfg
\end{multline*}

\begin{equation}
\begin{split}
a+c+s+d=fasdf \\
asrfsdfasd
\end{split}
\end{equation}

\begin{gather}
a+b+c+d+e +gsedf \\
f=s \\
a+gsdfas
\end{gather}

\begin{gather*}
a+b+c+d+e +gsedf \\
f=s \\
a+gsdfas
\end{gather*}

\begin{align*}
a+b+c & =  b+c \\
      & =  f+g \\
      & =  x+y
\end{align*}

\begin{align}
a+b+c & =  b+c \\
& =  f+g \\
& =  x+y
\end{align}


\begin{align}
a+b+c & =  b+c & a&=b \\
& =  f+g  &   c&=d\\
& =  x+y  &    c&=a+b+d
\end{align}

\begin{flalign}
a+b+c & =  b+c & a&=b \\
& =  f+g  &   c&=d\\
& =  x+y  &    c&=a+b+d
\end{flalign}

\begin{flalign*}
a+b+c & =  b+c & a&=b \\
& =  f+g  &   c&=d\\
& =  x+y  &    c&=a+b+d
\end{flalign*}


\begin{align}
\label{ec1} a&=b000 \\
c&=d0000 \notag\\
\label{ec2} c&=a+b+d000
\end{align}

Por las Ecuaciones \ref{ec1} y (\ref{ec2}) y 

\begin{align}
a+b+c & =  b+c & a&=b \tag{T. Pitágoras}\\
& =  f+g  &   c&=d\\
& =  x+y  &    c&=a+b+d \tag*{T. Pitágoras}
\end{align}

\chapter{Otro capítulo}

$$
f(x)=\left\{\begin{aligned}
ab+d+c&= sdfr \\
     a&=f+g+ f\\
      &=qe
\end{aligned}\right.
$$

$$
f(x)=\left\{\begin{gathered}
ab+d+c= sdfr \\
a=f+g+ f\\
=qe
\end{gathered}\right.
$$

$$
f(x)=\begin{cases}
x^2&, x\in X \\
x/2&, x\in X^C
\end{cases}
$$

$$
x_{\mbox{longitud}}\qquad x_{\text{longitud}}
$$

$$
f=g\text{ cuando }f(x)=g(x),\forall x\in A
$$


\begin{align}
a+b+c & =  b+c \\
& =  f+g \\ \intertext{Por lo tanto}
& =  x+y
\end{align}

Usando las Ecuaciones \eqref{ec1} y \eqref{ec2}

\begin{enumerate}
\item algo	
\end{enumerate}

Viendo (1)


texto texto $(\begin{smallmatrix}a & b \\ c & d\end{smallmatrix})$ texto texto texto texto texto texto


$$
\begin{matrix}
a & b  & c \\
d & e & f \\
g & h & i
\end{matrix}
$$

$$
\begin{pmatrix}
a & b  & c \\
d & e & f \\
g & h & i
\end{pmatrix}
$$

$$
\begin{bmatrix}
a & b  & c \\
d & e & f \\
g & h & i
\end{bmatrix}
$$

$$
\begin{Bmatrix}
a & b  & c \\
d & e & f \\
g & h & i
\end{Bmatrix}
$$

$$
\begin{vmatrix}
a & b  & c \\
d & e & f \\
g & h & i
\end{vmatrix}
$$

$$
\begin{Vmatrix}
a & b  & c \\
d & e & f \\
g & h & i
\end{Vmatrix}
$$

$$
\begin{pmatrix}
a & b  & c \\
d & e & f \\
\hdotsfor{3}
\end{pmatrix}
$$


\end{document}