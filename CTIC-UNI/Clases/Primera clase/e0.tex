\documentclass[12pt,a4paper]{article}

\usepackage[utf8]{inputenc}
\linespread{1.25}
\usepackage[lmargin=2cm,rmargin=2.5cm,tmargin=4cm,bmargin=3cm]{geometry}


\setlength{\parindent}{5mm}

\begin{document}
	
\begin{center}
	Universidad
\end{center}
	
\noindent Universidad        Nacional de Ingeniería	




\noindent 3 movimientos que hace la Tierra (que no son ni rotación ni traslación) 
y que quizás no conocías\\


Seguramente lo habrás aprendido en la escuela primaria: la Tierra describe una órbita elíptica alrededor del Sol.\\[3cm]

Este recorrido, que se conoce como movimiento de traslación, le toma al planeta unos 365 días 
(más 5 horas, 45 minutos y 46 segundos).

El otro movimiento que te enseñaron es el de rotación: la Tierra gira en torno a su propio eje.

Este giro sobre sí misma le toma aproximadamente un día (23 horas, 56 minutos 4,1 segundos, para ser exactos). 

Sin embargo, estos no son los únicos movimientos que hace la Tierra.

Te contamos —o recordamos— cuáles son los otros tres, también importantes, que ejecuta el planeta.

Movimiento de precesión de los equinoccios

Este es el movimiento que describe el eje inclinado de la tierra de forma circular.

Más concretamente, es el movimiento que hace el polo norte terrestre respecto al punto central de la elipse 
que describe la Tierra en el movimiento de translación.

Esta oscilación fue descrita por primera vez por el astrónomo, geógrafo y matemático griego Hiparco de Nicea 
que vivió entre los años 190 a.C. y 120 a.C. y fue el tercer movimiento de la Tierra en ser detectado.

Este bamboleo cíclico en la orientación del eje de rotación de la Tierra demora alrededor de 25.780 años. 

Su duración, no obstante, es relativamente imprecisa porque se ve influida por el movimiento y desplazamiento 
de las placas tectónicas.

Más concretamente, es el movimiento que hace el polo norte terrestre respecto al punto central de la elipse 
que describe la Tierra en el movimiento de translación.

Esta oscilación fue descrita por primera vez por el astrónomo, geógrafo y matemático griego Hiparco de Nicea 
que vivió entre los años 190 a.C. y 120 a.C. y fue el tercer movimiento de la Tierra en ser detectado.

Este bamboleo cíclico en la orientación del eje de rotación de la Tierra demora alrededor de 25.780 años. 

Más concretamente, es el movimiento que hace el polo norte terrestre respecto al punto central de la elipse 
que describe la Tierra en el movimiento de translación.

Esta oscilación fue descrita por primera vez por el astrónomo, geógrafo y matemático griego Hiparco de Nicea 
que vivió entre los años 190 a.C. y 120 a.C. y fue el tercer movimiento de la Tierra en ser detectado.

Este bamboleo cíclico en la orientación del eje de rotación de la Tierra demora alrededor de 25.780 años. 

Más concretamente, es el movimiento que hace el polo norte terrestre respecto al punto central de la elipse 
que describe la Tierra en el movimiento de translación.

Esta oscilación fue descrita por primera vez por el astrónomo, geógrafo y matemático griego Hiparco de Nicea 
que vivió entre los años 190 a.C. y 120 a.C. y fue el tercer movimiento de la Tierra en ser detectado.

Este bamboleo cíclico en la orientación del eje de rotación de la Tierra demora alrededor de 25.780 años. 

Más concretamente, es el movimiento que hace el polo norte terrestre respecto al punto central de la elipse 
que describe la Tierra en el movimiento de translación.

Esta oscilación fue descrita por primera vez por el astrónomo, geógrafo y matemático griego Hiparco de Nicea 
que vivió entre los años 190 a.C. y 120 a.C. y fue el tercer movimiento de la Tierra en ser detectado.

Este bamboleo cíclico en la orientación del eje de rotación de la Tierra demora alrededor de 25.780 años. 





¿Qué lo produce? Se genera por fundamentalmente por el momento de fuerza que ejerce el Sol sobre la Tierra. 






\end{document}