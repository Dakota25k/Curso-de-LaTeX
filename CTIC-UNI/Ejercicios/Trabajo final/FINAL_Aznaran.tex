\documentclass[12pt, a4paper, openright,twoside]{report}
\usepackage[T1]{fontenc}

\renewcommand{\rmdefault}{ptm}
%\usepackage{times}	% Otra manera y anulando \renewcommand{\rmdefault}{ptm}.

\title{\bf An Example of Report Class}  % Supply information
\author{for \LaTeX\ Class}              %   for the title page.
\date{\today}                           %   Use current date.

\begin{document}
%\maketitle                              % Print title page.
Hola
Una \emph{función elemental} de una variable $x$ es una función que se puede obtener de las funciones racionales en $x$ al unir repetidamente un número finito de anidados logaritmos, exponenciales y números o funciones algebraicas.

Desde $\sqrt{-1}$ es elemental, las funciones trigonométricas y sus inversas son elementales (cuando se reescriben usando exponenciales y logaritmos complejos), así como todas las funciones ``usuales''del cálculo. Por ejemplo,

\begin{equation}
2+1=3	
\end{equation}
\end{document}