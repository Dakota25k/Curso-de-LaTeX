\documentclass[12pt,a4paper]{report}
\usepackage[utf8]{inputenc}
\usepackage[spanish,es-sloppy]{babel}

\usepackage[lmargin=3cm,rmargin=2.5cm,tmargin=2.5cm,bmargin=2cm]{geometry}
\linespread{1.5}
\setlength{\parindent}{0pt}
\usepackage{mathptmx}
\usepackage{graphicx}

%\usepackage[x11names]{xcolor}
\usepackage[svgnames,table]{xcolor}
\usepackage{afterpage}

\usepackage{tikz}
\usepackage{listings}

\usetikzlibrary{shapes.geometric}
\usetikzlibrary{arrows.meta}
\usetikzlibrary{decorations.pathreplacing}
\newcommand*{\tikzgrid}[2]{\draw[help lines](0,0)grid[step=0.2,lightgray,ultra thin](#1,#2);\draw[help lines](0,0)grid[gray](#1,#2);\foreach\x in{0,1,...,#1}\node[below]at(\x,0){\scriptsize\x};\foreach\y in{1,2,...,#2}\node[left]at(0,\y){\scriptsize\y};}

\usepackage{fancyhdr}
\pagestyle{fancy}
\lhead{\colorbox{black}{\textcolor{white}{Tesis para optar el título}}}
\chead{}
\rhead{}
\lfoot{Juan Pérez}
\cfoot{}
\rfoot{\thepage}
\renewcommand{\headrulewidth}{0pt}
\renewcommand{\footrulewidth}{1pt}

\usepackage{fancybox}
\usepackage{longtable}
\usepackage{ragged2e}

\setcounter{secnumdepth}{2}
\setcounter{tocdepth}{3}

\usepackage{titlesec}
\titleformat{name=\chapter}
[display]
{\color{olive}\centering\LARGE\sffamily}
{Capítulo\space \arabic{chapter}}
{0mm}
{}
\titlespacing*{\chapter}
{0pt}
{-2cm}
{1cm}

\titleformat{name=\section}
[hang]
{\color{orange}\titlerule\RaggedLeft\bfseries\Large}
{\thesection\space}
{0cm}
{}
\titlespacing{\section}
{0pt}
{2cm}
{0cm}


\newcommand{\blankpage}{
	\newpage
	\thispagestyle{empty}
	\mbox{}
	\newpage
}
\usepackage{picins}
\usepackage{here}
\usepackage{hhline}
\usepackage{multirow}
\usepackage{pdflscape}
\newcommand{\salto}{\\[-3mm]}
\newcommand{\salti}{\\[-1.5mm]}
\usepackage{array}
\usepackage{listings}

%\usepackage{tocloft}
\usepackage[titles]{tocloft}
\setlength{\cftbeforetoctitleskip}{5cm}
\setlength{\cftaftertoctitleskip}{2cm}
\renewcommand{\cfttoctitlefont} {\hfil\LARGE\bfseries}
\renewcommand{\cftaftertoctitle}{\hfill}

\renewcommand{\cftdot}{-}
\renewcommand{\cftdotsep}{20}
\renewcommand{\cftdotsep}{\cftnodots}

\setlength{\cftsecindent}{0pt}
\setlength{\cftsubsecindent}{0pt}
\setlength{\cftsubsubsecindent}{0pt}

\renewcommand{\lstlistingname}{Código}
\usepackage[hidelinks,linktocpage=true,colorlinks =true,linkcolor=red]{hyperref}
\usepackage{apacite}


\begin{document}
		
	
\begin{titlepage}


	\begin{center}
		{\Large UNIVERSIDAD NACIONAL DE INGENIERÍA\\[-0.4cm]
		FACULTAD DE HUMANIDADES\\[-0.4cm]
		ESCUELA PROFESIONAL DE FILOSOFÍA\\[1cm]
		\includegraphics[width=3cm]{escudo}\\[0.8cm]
		Tesis para Optar\\[-0.4cm]
		el Título Profesional de\\
		LICENCIADO en FILOSOFÍA}
		
		\begin{tikzpicture}[very thick]
		%\tikzgrid{17}{3}
		\draw[line width=5mm,black](0,1)--(0,3);
		\draw[line width=5mm,black](0,2)--(15,2); 
		\end{tikzpicture}
		
		\scalebox{0.9}{Título:}\\[0.5cm]
		{\LARGE{Aprendiendo \LaTeX \ para mis trabajos}}\\[1cm]
		
		\scalebox{0.9}{Presentado por:}\\[0.5cm]
		{\Large\textbf{Juan Pérez}}\\[1cm]
		\scalebox{0.9}{Asesor}\\
		\scalebox{0.9}{Dr. John Smith}
		\vfill 
		LIMA - PERÚ\\
		2018
			
	\end{center}
	
\end{titlepage}

\blankpage

\pagenumbering{roman}
\addtocounter{page}{1}	

\ \\[16cm]
\begin{longtable}[r]{l}
	\parbox{6.2cm}{\textbf{Dedicatoria}\\
		Dedicado  a  los  que  alguna  vez  me
		ayudaron en comprender los conocimientos básicos durante mis estudios
		de pregrado.}
\end{longtable}

\tableofcontents
\listoffigures
\listoftables


\newpage
\pagenumbering{arabic}
\addtocounter{page}{5}
\chapter{Juegos de Mesa}
\section{Los expertos en juegos de mesa usan mejor el cerebro}
\ \\[0.5cm]
\piccaption[Los científicos estudiaron a expertos en shogi, el ajedrez japonés.]{Los científicos estudiaron a expertos en shogi, el ajedrez japonés.}
\parpic(8.85cm,4.85cm)[rs][br]{\includegraphics[page=8,trim=10cm 16.4cm 2.7cm 9cm,clip]{pc2a}}\label{f1}
Una nueva investigación descubrió
que los expertos en juegos de me-
sa, como el ajedrez, utilizan una
región del cerebro que el resto no
solemos usar.\\ El estudio, publicado en Science,
llevó a cabo escáneres cerebrales de jugadores, tanto profesionales como aficionados, del juego japonés shogi, también llamado ajedrez japonés debido a su similitud.\\Los investigadores del Instituto de Ciencia Cerebral Riken, en Japón, descubrieron que las jugadas intuitivas que llevan a cabo estos jugadores no son naturales, sino que surgen del entrenamiento cerebral.

\footnote{Ver Figura\ref{f1}}

\newpage
Los profesionales del shogi entrenan hasta por 10 años, tres o cuatro horas al día, para lograr la habilidad que se requiere para jugar a ese nivel.

\color{DarkGrey}
\begin{longtable}[r]{r}
\textcolor{ForestGreen}{\Huge{“}} \textcolor{DarkGrey}El hallazgo fue una sorpresa,\\
porque al volverse expertos\\
los maestros shogi comienzan\\
a usar todas las regiones del\\
cerebro \textcolor{ForestGreen}{\Huge{”}}\\
\textcolor{magenta}{\scalebox{0.6}{Prof. Keiji Tanaka}}
\end{longtable}
\color{black}
Estos individuos son capaces de llevar a cabo decisiones “intuitivas” muy rápidas sobre la jugada o combinación de jugadas que harán en el tablero para lograr el mejor resultado.

Los científicos reclutaron a jugadores profesionales miembros de la Asociación Japonesa de Shogi.

También participó en el estudio un grupo de jugadores aficionados.

A 17 de los profesionales se les presentó un juego de shogi que ya estaba en progreso y se les dieron dos segundos para elegir la mejor jugada siguiente, de entre cuatro jugadas.

Según los investigadores, los escáneres cerebrales de estos jugadores mostraron una activación significativa en el área del núcleo caudado mientras llevaban a cabo sus jugadas rápidas.

Durante mucho tiempo se ha pensado que esa región del cerebro es responsable del control de los movimientos corporales voluntarios. Pero estudios más recientes lo han vinculado al aprendizaje y la memoria.

\subsection{Actividad cerebral}

Cuando se les pidió a los jugadores aficionados que eligieran rápidamente su mejor jugada siguiente, no se observó activación significativa en el núcleo caudado.

Esta actividad cerebral sólo se vio en los jugadores profesionales que llevaban a cabo decisiones muy rápidas sobre la siguiente mejor jugada.
\newpage
\begin{center}
	\includegraphics[page=10,trim=4cm 23cm 4cm 2.5cm,clip]{pc2a}
	\caption{\small Quienes juegan profesionalmente utilizan partes del cerebro que otros no usan.}
\end{center}

Además, se encontró que los profesionales no usaban esa área del cerebro cuando se les daba un tiempo mayor a los ocho segundos para que pensaran estratégicamente sobre las siguientes jugadas que debían realizar.

\subsection{¿Con cuántas copas da positivo?}	
	\begin{table}[H]
		\centering
		\caption{Copas para alcanzar 0,25 mg/l}\label{t1}
		\vspace{1.8cm}
		\begin{tabular}{|c|c|c|c|c|c|c|}
			\hline
			\multirow{3}{*}{Tipos de bebida} &\multicolumn{2}{|c|}{Estimación para}&\multicolumn{4}{|c|}{Percepción de la población}\\
			\cline{4-7}
			&\multicolumn{2}{|c|}{alcanzar 0,25}&\multicolumn{2}{|c|}{Medias}&\multicolumn{2}{|c|}{Medianas}\\
			\cline{2-7}
			& Hombre &Mujer&Hombre&Mujer&Hombre&Mujer\\
			\hline
			Vasos de cerveza& 1,5 & 1,0&4,0&2,0&3,8&2,0\\
			\hline
			Copas de vino& 2,0 & 1,5 &3,8&2,0&3,8&2,0\\
			\hline
		\end{tabular}
	\end{table}	
	
	Extraido de:
	
	\textcolor{red}{\url{https://www.vinetur.com/2015100621273/con-cuantas-copas-da-positivo.html}}
	
\section{Una tabla}	

La siguiente tabla (Ver (Voss, 2011)) tiene las siguientes entradas.

\begin{enumerate}
	\item ab
	\item cd
	\item ef
	\item ghi
	\item j
	\item k
	\item lm
\end{enumerate}

\begin{table}[H]
	\centering
	\caption{La siguiente tabla es muy simple.}\label{t2}
	\vspace{1.8cm}
	\begin{tabular}{|c|c|l||}
	\hline
	\multicolumn{3}{|c|}{Una tabla simple}\\
	\hline
	ab&cd&ef\\
	\cline{1-2}
	\multicolumn{2}{|c|}{ghi}&\\
	\hline
	\multicolumn{1}{|r|}{j}&k&lm\\
	\hline
	\end{tabular}
\end{table}

\chapter{Otras cosas}
\section{Rol de la Educación en la prevención de la TB}
\vspace{8.5cm}
\begin{center}
\cornersize*{100mm}
\setlength{\fboxsep}{5mm}
\Ovalbox{\parbox{11.5cm}{El sistema educativo en la etapa de la formación del educando, tiene
		la finalidad clara de trasmitir a todos los actores de la comunidad
		educativa los elementos básicos de la cultura. Formarles para asumir
		sus deberes y ejercer sus derechos y prepararles para la incorporación
		a la vida activa}}
\end{center}
\newpage
\begin{landscape}
	\begin{center}
		\footnotesize
		\sffamily
		
		\begin{tikzpicture}[very thick]
						
				%\tikzgrid{20}{13}
				\useasboundingbox(0,0)rectangle(20,13);
				\node[text width=16cm,align=center,fill=DarkBlue!7,rounded corners=5mm]at(9.6,6.4){\vspace{12cm}};
				\node[text width=7cm,align=center,fill=DarkOrange]at(9,1){\textcolor{white}{\bfseries{Retroalimentación}}};
				\draw[line width=3mm,MidnightBlue](3.2,2.4)--(3.2,1.6)--(14.8,1.6)--(14.8,2.4);
				\draw[fill=MidnightBlue,draw=MidnightBlue,] (2.9,2.4) -- (3.5,2.4) -- (3.2,2.7) -- cycle;
				\draw[fill=MidnightBlue,draw=MidnightBlue,] (14.5,2.4) -- (15.1,2.4) -- (14.8,2.7) -- cycle;
				\node[text width=2cm,align=center,fill=DarkOrange,rounded corners]at(3.6,11.6){\textcolor{white}{\bfseries\sffamily{Abogacía}}};
				\node[text width=2.5cm,align=center,fill=DarkOrange,rounded corners]at(7,11.6){\textcolor{white}{\bfseries\sffamily{Movilización\\[-3mm]social\\[-3mm](Capacitación)}}};
				\node[text width=3cm,align=center,fill=DarkOrange,rounded corners]at(10.9,11.6){\textcolor{white}{\bfseries\sffamily{Comunicación e\\[-3mm]información}}};
				\node[text width=3.5cm,align=center,fill=Green!70,rounded corners]at(15.4,11.6){\textcolor{white}{\bfseries\sffamily{Acompañamiento}}};
				\draw[line width=3mm,MidnightBlue](4.9,11.6)--(5.2,11.6);
				\draw[fill=MidnightBlue,draw=MidnightBlue,] (5.2,11.8) -- (5.4,11.6) -- (5.2,11.4) -- cycle;
				\draw[line width=3mm,MidnightBlue](8.5,11.6)--(8.8,11.6);
				\draw[fill=MidnightBlue,draw=MidnightBlue,] (8.8,11.8) -- (9,11.6) -- (8.8,11.4) -- cycle;
				\draw[line width=3mm,MidnightBlue](12.7,11.6)--(13,11.6);
				\draw[fill=MidnightBlue,draw=MidnightBlue,] (13,11.8) -- (13.2,11.6) -- (13,11.4) -- cycle;
				\node[text width=2.7cm,align=center,fill=NavyBlue,rounded corners]at(3.4,10){\textcolor{white}{Reuniones de\\[-3mm]sensibilización}};
				\draw[arrows={-Straight Barb[scale=0.7,width=1mm,blue]},ultra thick,color=blue](2.09,9.6)--(2.09,8.4)--(2.4,8.4);
				\draw[arrows={-Straight Barb[scale=0.7,width=1mm,blue]},ultra thick,color=blue](2.09,8.4)--(2.09,6.8)--(2.4,6.8);
				\node[text width=2.7cm,align=center,fill=NavyBlue,rounded corners]at(7.2,9.8){\textcolor{white}{Capacitación a\\[-3mm]docentes\\[-3mm]universitarios}};
				\draw[arrows={-Straight Barb[scale=0.7,width=1mm,blue]},ultra thick,color=blue](6,9.24)--(6,8.4)--(6.3,8.4);
				\draw[arrows={-Straight Barb[scale=0.7,width=1mm,blue]},ultra thick,color=blue](6,8.4)--(6,7.25)--(6.3,7.25);		
				\node[text width=2.7cm,align=center,fill=NavyBlue,rounded corners]at(10.9,10){\textcolor{white}{Acciones\\[-3mm]promocionales}};
				\draw[arrows={-Straight Barb[scale=0.7,width=1mm,blue]},ultra thick,color=blue](9.8,9.56)--(9.8,8.6)--(10.1,8.6);
				\draw[arrows={-Straight Barb[scale=0.7,width=1mm,blue]},ultra thick,color=blue](9.8,8.6)--(9.8,7.15)--(10.1,7.15);
				\node[draw=DarkTurquoise,text width=2.6cm,fill=white,rounded corners]at(3.8,8.4){\textcolor{blue}{Reuniones con\\[-3mm]autoridades\\[-3mm]universitarias}};
				\node[draw=DarkTurquoise,text width=2.6cm,fill=white,rounded corners]at(3.8,6.5){\textcolor{blue}{Reuniones con\\[-3mm]responsables de\\[-3mm]Servicio de Salud\\[-3mm]Universitario y\\[-3mm]Bienestar Social}};
				\node[draw=DarkTurquoise,text width=2.2cm,fill=white,rounded corners]at(7.6,8.5){\textcolor{blue}{Módulo de\\[-3mm]capacitación}};
				\node[draw=DarkTurquoise,text width=2.2cm,fill=white,rounded corners]at(7.6,7.3){\textcolor{blue}{Talleres de\\[-3mm]capacitación}};
				\node[draw=DarkTurquoise,text width=2.5cm,fill=white,rounded corners]at(11.5,8.6){\textcolor{blue}{Implementación\\[-3mm]de iniciativas\\[-3mm]saludables}};
				\node[draw=DarkTurquoise,text width=2.5cm,fill=white,rounded corners]at(11.5,7.2){\textcolor{blue}{Consumo\\[-3mm]universitario}};
				\node[text width=2.9cm,align=center,fill=NavyBlue,rounded corners]at(7.4,6){\textcolor{white}{Capacitación a\\[-3mm]personal de Salud y\\[-3mm]de Bienestar Social}};
				\draw[arrows={-Straight Barb[scale=0.7,width=1mm,blue]},ultra thick,color=blue](6.1,4.6)--(6.1,3.4)--(6.4,3.4);
				\draw[arrows={-Straight Barb[scale=0.7,width=1mm,blue]},ultra thick,color=blue](6.1,5.45)--(6.1,4.6)--(6.4,4.6);
				\node[text width=2.7cm,align=center,fill=NavyBlue,rounded corners]at(11.1,6){\textcolor{white}{Réplicas a\\[-3mm]estudiantes}};
				\draw[arrows={-Straight Barb[scale=0.7,width=1mm,blue]},ultra thick,color=blue](9.9,5.6)--(9.9,4.7)--(10.2,4.7);
				\node[draw=DarkTurquoise,text width=2.4cm,fill=white,align=center,rounded corners]at(7.8,4.6){\textcolor{blue}{Talleres sobre\\[-3mm]guías nacionales}};
				\node[draw=DarkTurquoise,text width=2.4cm,fill=white,align=center,rounded corners]at(7.8,3.2){\textcolor{blue}{Talleres sobre\\[-3mm]metodologías de\\[-3mm]capacitación\\[-3mm]docente}};
				\draw [dotted,ultra thick,color=LightSeaGreen,arrows={-Straight Barb[scale=1,width=2.5mm,LightSeaGreen]}] (6.4,3) -- (5.65,3)--(5.65,6);
				\draw [ultra thick,color=LightSeaGreen,arrows={-Straight Barb[scale=1,width=2.5mm,LightSeaGreen]}] (5.84,6.2) -- (5.4,6.2)--(5.4,9.75)--(5.71,9.75);
				\draw [ultra thick,color=LightSeaGreen,arrows={-Straight Barb[scale=1,width=2.5mm,LightSeaGreen]}] (8.66,9.75) -- (9.15,9.75)--(9.15,6)--(9.64,6);
				\node[draw=DarkTurquoise,text width=2.5cm,fill=white,align=center,rounded corners]at(11.6,4.6){\textcolor{blue}{Charlas realizado\\[-3mm]por docentes}};
				\node[text width=2.7cm,align=center,fill=NavyBlue,rounded corners]at(11.2,3.2){\textcolor{white}{Material de\\[-3mm]comunicación\\[-3mm]producidos}};
				\draw [dotted,ultra thick,color=LightSeaGreen,arrows={-Straight Barb[scale=1,width=2.5mm,LightSeaGreen]}] (12.67,3.2) -- (13.1,3.2)--(13.1,6)--(12.6,6);
				\draw [dotted,ultra thick,color=LightSeaGreen,arrows={-Straight Barb[scale=1,width=2.5mm,LightSeaGreen]}] (13.1,6) -- (13.1,10.1)--(12.4,10.1);
				\node[text width=3.5cm,align=center,fill=Green!70,rounded corners]at(15.4,8.2){\textcolor{white}{\bfseries\sffamily{Recojo de reportes\\[-3mm]de atención}}};
				\node[text width=3.5cm,align=center,fill=Green!70,rounded corners]at(15.4,5){\textcolor{white}{\bfseries{Visitas de\\[-3mm]acompañamiento y\\[-3mm]supervisón}}};
				\draw[ultra thick,color=olive](13.35,2.55)--(13.35,6.3)--(13.95,6.5)--(13.35,6.7)--(13.35,10.45);
				\draw[ultra thick,color=olive](13.1,10.5)--(13.3,10.5);
				\draw[ultra thick,color=olive](13.1,2.5)--(13.3,2.5);
				\draw[ultra thick,color=olive] (13.35,10.44) arc (0:90:0.06);
				\draw[ultra thick,color=olive] (13.35,2.56) arc (0:-90:0.06);
				
		\end{tikzpicture}
		
		
	\end{center}
		\caption{ruta metodológica de la propuesta de intervención}
\end{landscape}

\begin{landscape}	
	\begin{table}[H]
		\centering
		\caption{Población universitaria por año censal y universidad en Lima y Callao}\label{t3}
		\vspace{1.8cm}
		\linespread{1.5}
		
		\small
		\sffamily
		\begin{tabular}{|c|c|c|c|c|c|c|c|c|c|c|}
			\hline
			\rowcolor{ForestGreen!77}&\multicolumn{4}{|c|}{\bfseries\small\textcolor{white}{Año 1996}}&&\multicolumn{4}{|c|}{\bfseries\small\textcolor{white}{Año 2010}}&\\
			\hhline{|>{\arrayrulecolor{ForestGreen!77}}->{\arrayrulecolor{black}}---->{\arrayrulecolor{ForestGreen!77}}->{\arrayrulecolor{black}}---->{\arrayrulecolor{ForestGreen!77}}->{\arrayrulecolor{black}}|}
			\rowcolor{ForestGreen!77}&\multicolumn{2}{|c|}{\small\textcolor{white}{Alumnos}}&&&&\multicolumn{2}{|c|}{\small\textcolor{white}{Alumnos}}&&&\\
			\hhline{|>{\arrayrulecolor{ForestGreen!77}}->{\arrayrulecolor{black}}|-->{\arrayrulecolor{ForestGreen!77}}--->{\arrayrulecolor{black}}-->{\arrayrulecolor{ForestGreen!77}}--->{\arrayrulecolor{black}}|}
			\rowcolor{ForestGreen!77}&&&&&&&&&&\\
			\hhline{|>{\arrayrulecolor{ForestGreen!77}}----------->{\arrayrulecolor{black}}|}
			\rowcolor{ForestGreen!77}\multirow{-4}{*}{\bfseries\small\textcolor{white}{Universidad}}& \multirow{-2}{*}{\parbox{1.3cm}{\centering\small\textcolor{white}{ Pre\salto grado}}} &\multirow{-2}{*}{\parbox{1.3cm}{\centering\small\textcolor{white}{ Post\salto grado}}}& \multirow{-3}{*}{\small{\parbox{1.8cm}{\centering \textcolor{white}{Docente\salto universita-\salto rio}}}}&\multirow{-3}{*}{\small{\parbox{2cm}{\centering\textcolor{white}{ Personal\salto administra-\salto tivo y de\salto servicio}}}}&\multirow{-4}{*}{\bfseries\small\textcolor{white}{Total}}&\multirow{-2}{*}{\parbox{1.3cm}{\centering\small\textcolor{white}{ Pre\salto grado}}} &\multirow{-2}{*}{\parbox{1.3cm}{\centering\small \textcolor{white}{Post\salto grado}}}&\multirow{-3}{*}{\small{\parbox{1.8cm}{\centering \textcolor{white}{Docente\salto universita-\salto rio}}}}&\multirow{-3}{*}{\small{\parbox{2cm}{\centering\textcolor{white}{ Personal\salto administra-\salto tivo y de\salto servicio}}}}&\multirow{-4}{*}{\bfseries\small\textcolor{white}{Total}}\\
			\hline
			\multicolumn{1}{|m{4cm}|}{\raggedright\footnotesize Universidad Nacional\salti
				Agraria La Molina}&3,225 &268 &453 &498 &4,434&4,903& 976& 445& 802& 7,126\\
			\hline
			\rowcolor{OrangeRed!80}\multicolumn{1}{|m{4cm}|}{\raggedright\footnotesize Universidad Nacional de\salti Ingeniería} &6953& 277& 940& 598&8768 &11034& 1 068& 1210& 1 489 &14801\\
			\hline
			\multicolumn{1}{|m{4cm}|}{\raggedright\footnotesize Universidad Nacional de\salti Educación Enrique\salti Guzmán y Valle}&6388 &147& 458&344& 7337& 9178& 2 144& 727& 561& 12610\\
			\hline
			\multicolumn{1}{|m{4cm}|}{\raggedright\footnotesize Universidad Nacional del\salti
				Callao}&8066&-&530 &216& 8812& 13584& 391& 632& 497& 15104\\
			\hline
			\multicolumn{1}{|m{4cm}|}{\raggedright\footnotesize Universidad Nacional\salti Federico Villarreal}& 16173&992& 1492& 862& 19519& 23105 &2 447 &1936& 1072& 28560\\
			\hline
			\multicolumn{1}{|m{4cm}|}{\raggedright\footnotesize Universidad Nacional\salti
				Mayor de San Marcos}&21341& 3077& 2677& 1 267& 28362& 28645& 3 477& 2711& 2786 &37619\\
			\hline
			\cellcolor{ForestGreen!77}\bfseries\small{\textcolor{white}{TOTAL}}&\bfseries62,146 &\bfseries4,751&\bfseries 6,550&\bfseries 3,785&\cellcolor{ForestGreen!77}&\bfseries 90,449 &\bfseries 10,503&\bfseries 7,661&\bfseries 7,207&\cellcolor{ForestGreen!77}\\
			\hline
		\end{tabular}
	\end{table}
\end{landscape}
\chapter{Un tipo de letra}
%\lstinputlisting[breaklines]{hge.mf}
\lstinputlisting[breaklines,rulecolor=\color{blue},language=TeX, basicstyle=\ttfamily,caption={Un tipo de letra}, backgroundcolor=\color{PaleGreen!50},frame=single,linerange={30-36},numbers=left, numberstyle={\scriptsize}]{hge.mf}


	
\end{document}
