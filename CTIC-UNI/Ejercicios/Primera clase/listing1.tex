%
% listing1.tex -- .
%
% Copyright © 2018 Oromion <caznaranl@uni.pe>
%
% This program is free software: you can redistribute it and/or modify
% it under the terms of the GNU General Public License as published by
% the Free Software Foundation, either version 3 of the License, or
% (at your option) any later version.
%
% This program is distributed in the hope that it will be useful,
% but WITHOUT ANY WARRANTY; without even the implied warranty of
% MERCHANTABILITY or FITNESS FOR A PARTICULAR PURPOSE.  See the
% GNU General Public License for more details.
%
% You should have received a copy of the GNU General Public License
%s along with this program.  If not, see <http://www.gnu.org/licenses/>.
% 
% Please contact caznaranl@uni.pe to report any problems or bugs.
%
\documentclass[12pt,a4paper]{article}
\usepackage[T1]{fontenc}
\usepackage[spanish,es-sloppy]{babel}
\usepackage[dvipsnames, usenames]{color}
\usepackage[lmargin=3.5cm,rmargin=4cm,tmargin=4.5cm,bmargin=4.5cm]{geometry}

\title{7 razones por las cuales la Tierra no sólo gira, sino que también tiembla y se tambalea}
\author{Nic Fleming\thanks{BBC Earth}}
\date{20 enero 2017}

%\pagestyle{empty}

\linespread{1.5}
\setlength{\parindent}{5mm}

\begin{document}

\maketitle

\tableofcontents

\newpage

\renewcommand{\spanishabstractname}{Resumen}

{\flushleft\begin{abstract}
Artículo copiado de

\texttt{http://www.bbc.com/mundo/vert-earth-38323434}

en papel A4, tamaño de letra 12pt y espacio y medio interlineal.
\end{abstract}}

{\bfseries\normalsize 
La Tierra parece sólida e inmutable bajo nuestros pies la mayor parte del tiempo, pero eso es tan solo una ilusión que nació de nuestra perspectiva limitada.
}

En realidad la Tierra está lejos de ser estable.

Bajo nosotros, enormes pedazos están constantemente moviéndose para crear valles, empujándose para formar montañas o desplazándose para dar lugar a ríos y océanos.

{\bfseries\sffamily
El suelo siempre está cambiando, estirándose y tambaleándose.
}

Y nuestra creciente comprensión de estos fenómenos está dando lugar a un mejor conocimiento sobre cómo funciona nuestro planeta.

{\Large
Estas son siete razones que explican el movimiento de la Tierra bajo nuestros pies.
}

\section{La presión}

Un globo terráqueo de escritorio es una esfera perfecta que gira sobre un eje fijo. Sin embargo, \textbf{la Tierra no es esférica} y la masa dentro de ella está distribuida de manera desigual y propensa a moverse.\\[5mm]

Como resultado, el eje sobre el que gira, y los polos rotacionales del norte y el sur se mueven.

Debido a que el eje de rotación es diferente al de simetría, \textcolor{blue}{\large la Tierra se tambalea al girar}.

\section{El agua}

\textbf{Las estaciones son la segunda gran influencia en el bamboleo de la Tierra}, pues causan variaciones geográficas en a partir de la cantidad de lluvia, nieve y humedad.

Los científicos identificaron los polos usando las posiciones de las estrellas desde 1899, y satélites desde la década de 1970.

Pero incluso después de eliminar el impacto del bamboleo de Chandler y las oscilaciones estacionales, los polos rotacionaless norte y sur todavía se mueven con respecto a la corteza terrestre.\\[2mm]

Antes del año 2000, \textcolor{red}{el eje de rotación de la Tierra estaba desviándose hacia Canadá unos pocos centímetros cada año. Pero mediciones posteriores mostraron que se desviaba hacia las islas británicas}.

\section{Los meandros artificiales}

Otros cambios que afectan a las oscilaciones de la Tierra se deben a la acción humana.

En un estudio en 2009, Felix Landerer, también del JPL, calculó que si los niveles de óxido de carbono se duplicaban entre 2000 y 2100, los océanos se calentarían y expandirían de tal forma que el polo norte se movería unos 1,5 centímetros por año hacia Alaska y Hawái durante el próximo siglo.

\ldots

\section{Los movimientos sísmicos}

No sólo grandes cantidades de agua y hielo afectan a la rotación de la Tierra si se mueven. Las rocas tienen el mismo efecto, si son lo suficientemente grandes.

\ldots

\section{Las tormentas}

Un terremoto desencadena ondas sísmicas que transportan su energía hacia el interior de la Tierra.

Existen dos tipos: las ``ondas p'' -que aprietan y expanden el material por el que pasan- y las ``ondas s'', más lentas y que hacen oscilar la roca de lado a lado.

\ldots

\section{La influencia de la Luna}

Estudios recientes sugieren que \textcolor{magenta}{\large los grandes terremotos son más probables cuando hay luna llena o nueva}.

Eso podría deberse a que el Sol, la Luna y la Tierra están alineados, lo cual aumenta la fuerza gravitacional que actúa sobre nuestro planeta.

En un estudio publicado en septiembre de este año, Satoshi Ide, de la Universidad de Tokio, y sus colegas analizaron las mareas en las dos semanas previas a grandes terremotos durante las últimas dos décadas.

De los 12 terremotos más grandes, nueve ocurrieron cerca de lunas llenas o nuevas. No encontraron esa relación en terremotos más pequeños.

Ide concluyó que esa fuerza gravitacional añadida podría aumentar las fuerzas que actúan sobre las placas tectónicas y provocar pequeños cambios.

Aunque suena razonable, algunos científicos se muestran escépticos porque el estudio sólo se basó en 12 terremotos.

\section{El sol}

Todavía es más controversial la idea de que las vibraciones originadas dentro del Sol puedan explicar las oscilaciones de la Tierra.

Cuando los gases se mueven dentro del Sol, producen dos tipos de ondas: las generadas por cambios en la presión se llaman ``modos p'' y las que se forman cuando el material denso es empujado por la gravedad se llaman ``modos g''.

\ldots
\end{document}