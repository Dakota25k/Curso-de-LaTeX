\documentclass{12pt,a4paper}{report}% TODO revisar parámtros
\usepackage[utf8]{inputenc} % En caso de no ser TeXLive 2018
\usepackage[spanish]{babel}
\usepackage{amssymb,amsthm}
\usepackage{enumitem}
\usepackage[dvipsnames, usenames]{color}
\title{Algo de matemática y economía}
\author{\textcolor{red}{Juan Pérez}\thanks{Gracias a CTIC}}
\date{junio, 6 del 2018}

\begin{document}

\tableofcontents
\listoffigures
\listoftables

\section{Espacios métricos}

\subsection{Nociones básicas}

\subsubsection{Espacios métricos. Definición y ejemplos}

Sea $X$ un conjunto no vacío. Una función $d:X\times X\rightarrow\mathbb{R}_{+}$ que satisface las siguientes propiedades es llamada una \textbf{función distancia} sobre $X$: Para cualquier $x,y,z\in X$,

\begin{enumerate}
	\item $d(x,y)=0$ si y sólo si $x=y$
	\item $d(x,y)-d(y,x)$
	\item $d(x,y)\leq\leq d*(x,z)+d(z,y)$.
\end{enumerate}

Si $d$ es una función distancia sobre $X$, decimos que $(X,d)$ es un \textbf{espacio métrico}, y nos referimos a los elementos de $X$ como \textbf{puntos} en $(X,d)$. Si $d$ satisface%
para cualquier $x\in X$, entonces decimos que $d$ es una \textbf{semimétrica} sobre $X$, y $(X,d)$ es un \textbf{espacio semimétrico}

Sea $X$ un conjunto no vacío. Una maner trivial de hacer $X$ un espacio métrico es usar la métrica

Para cualquier $n\in\mathbb{N}, a_i, b_i\in\mathbb{R}, i=1,\ldots,n$ y cualquier $1\leq p<\infty$,

$$
^{}\leq+
$$

\section{Equilibrio}

Ahora enunciaremos y probaremos el mayor resultado de esta introducción, que bajo las hipótesis introducidas arriba, la Ley de Walras y Continuidad, existe un equilibrio en la economía. Para hacer esto, necesitamos una pieza adicional de estructura matemática, el Teorema del Punto fijo de Brouwer.

Sean $f(\cdot)$ una función continua, $f:P\rightarrow P$. Entonces existe $x^{\star}\in P$ tal que $f(x^{\star})=x^{\star}$.

Consideremos que se satisface la Ley de Walras y Continuidad. Entonces existe $p^{\star}\in P$ tal que $p^{\star}$ es un equilibrio.

La prueba del teorema es el análisis matemático de una historia. Supongamos precios asignados por un subastador. Él llama un vector de precios $p$, mientras otros estarán en exceso de demanda. El subastador entonces hace lo obvio. Él alza el precio de los bienes en exceso de demanda y reduce el precio de los bienes en exceso de oferta. Pero no se puede mucho cualquier de los cambios; los precios deben mantenerse en el simplex. ¿Cómo debe asegurarse de mantener los precios en el simplex? Primero, los precios deben permanecer no negativos. Cuando él reduce un precio, debe asegurarse no reducirlo bajo cero. Cuando el aumenta los precios, debería estar seguro que el nuevo precio permanece en el simplex. ¿Cómo puede hacer esto? Él modifica los nuevos precios tal que sumen uno. Además del Teorema del Punto Fijo de Brouwer%todo figura repetida

Sea $T$
\section{Una tabla}

\begin{table}
\centering
\begin{tabular}{rcl}
\multicolumn{3}{|c|}{Una tabla simple} \\
abab & cded & ef \\
\multicolumn{2}{c}{ghi} & \\
j & k & lm
\end{tabular}
\end{table}

\end{document}