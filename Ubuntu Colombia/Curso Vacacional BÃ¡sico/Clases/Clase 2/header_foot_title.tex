\documentclass[10pt]{article}%{book}
\usepackage[utf8x]{inputenc}
\usepackage[spanish]{babel}
\spanishdatedel
\usepackage{amsmath}
\usepackage{amsfonts}
\usepackage{amssymb}
\usepackage[demo]{graphicx}
\usepackage{lipsum}
\usepackage{lscape}
\usepackage{tikz}
\usepackage{fancyhdr} %Nos permite configurar el contenido de los pies de ṕágina.
\lhead{Autor}
\rhead{pág. \thepage}
\chead{\includegraphics[scale=.01]{demo}}
\lfoot{\thesection}
\cfoot{}
\rfoot{\tikz \draw[magenta, opacity = 50] (0,0) circle[radius = 5pt];}
\renewcommand{\headrulewidth}{0pt} %Se elimina la línea del encabezado.
\renewcommand{\footrulewidth}{.5pt}
\usepackage[left = 2.5cm, right = 1cm, top = 3cm, bottom =1cm, foot = 5mm, head =1.5cm, width = 10cm]{geometry}%includeheadfoot

\author{Undg. Carlos Aznarán Laos}
\title{Lección V2.2}

\begin{document}
\maketitle

%\begin{abstract}
%	\lipsum[1-7]
%\end{abstract}

%\begin{landscape}
%	\lipsum[3]
%\end{landscape}

%	\lipsum[5]
%\section{Sección uno}
%\lipsum[1]
\begin{titlepage}
	\thispagestyle{empty} %No afecta a l apágina de título encabezado o pie de página.
	\newgeometry{hmargin ={1.5cm}, vmargin = {2cm}, nomarginpar, ignorefoot, ignorehead}
	\center
	\Huge{Título del libro}	\linebreak
	\Large{Subtítulo del libro}
	
	\vspace{5cm}
	
	\Large{Universidad Nacional de Ingeniería}	\linebreak
	\small{Facultad de Ciencias}
	
	\vspace{5mm}
	
	\includegraphics[scale=.15]{demo}
	
	\vspace{5cm}
	
	\begin{minipage}{.4\textwidth}
		Autor: Carlos Aznarán
		\small{Prácticando con \LaTeX{}}
	\end{minipage}

	\hfill
	\begin{minipage}{.4\textwidth}
		\flushright
		Jurado: Johnny Valverde	\linebreak
		\small{Montoro}
	\end{minipage}

	\vspace{2cm}
	
	\begin{tikzpicture}
		\draw[fill = magenta, opacity = .5] (0, .5) circle[radius =1];
		\draw[fill = cyan, opacity =.5](.5, 0) circle[radius =1];
		\draw[fill = yellow, opacity =.5](-.5,0) circle[radius =1];
	\end{tikzpicture}
	Diciembre 13 de 2017
\end{titlepage}

%\section{Capítulo uno}
\pagestyle{fancy}
\section{Nueva sección}
\lipsum[1-10]
\section{Segunda sección}
\lipsum[5-10]
\end{document}
Tarea: Hacer una carátula.

Ver tomando el control de LaTeX.

Llamar con myarticle.

Ornaments

Tarea: Hacer una página de título.

Caminos y nodos mañana.