%% -------------------------------------------------------------------------
%% 2017 por Fausto M. Lagos S. <piratax007@protonmail.ch>
%% 
%% Este trabajo puede ser distribuido o modificado bajo los
%% términos y condiciones de la LaTeX Project Public License (LPPL) v1.3C, 
%% o cualquier versión posterior. La última versión de esta licencia
%% puede verse en:
%% http://www.latex-project.org/lppl.txt
%% 
%% -------------------------------------------------------------------------
%% Usted es libre de usarlo, modificarlo o distribuirlo siempre que se
%% respeten los términos de la licencia y se reconozca al autor original
%% -------------------------------------------------------------------------
%% Estos comandos permiten dibujar rápidamente un intervalo de cualquier
%% tipo en recta real.
%% -------------------------------------------------------------------------
%% Paquetes necesarios
\usepackage{tikz}
\usetikzlibrary{babel, shapes}
\usepackage{amsmath}
\usepackage{amsfonts}
\usepackage{amssymb}
%% -------------------------------------------------------------------------

%----------------------------------------------------
% Parámetros de entrada
% #1 cota inferior
% #2 cota superior
% #3 posición vertical del intervalo
% #4 fill o fill = none o <-, tipo de cota
% #5 fill o fill = none o ->, tipo de cota
% #6 nonInf o inf, tipo de intervalo
% #7 color
\newcommand{\interval}[7]{
    \IfEqCase {#6}{
        {nonInf}{
        \node [circle, draw, #4, line width = 1.5pt, color = #7, inner sep = 0pt, minimum size = 5pt] (ci) at (#1, #3) {};
        \node [circle, draw, #5, line width = 1.5pt, color = #7, inner sep = 0pt, minimum size = 5pt] (cs) at (#2, #3) {};
        \draw [line width = 1.5pt, color = #7] (ci) -- (cs);
        \draw (ci) -- (#1, -.2);
        \node (tag) at (#1, -.4) {#1};
        \draw (cs) -- (#2, -.2);
        \node (tag) at (#2, -.4) {#2};
        }
        {inf}{
        	\IfEqCase {#4}{
            	{<-}{
			        \node [circle, draw, #5, line width = 1.5pt, color = #7, inner sep = 0pt, minimum size = 5pt] (cs) at (#2, #3) {};
                    \draw [#4, line width = 1.5pt, color = #7] (#1 - 1.5, #3) -- (cs);
                    \draw (cs) -- (#2, -.2);
                    \node (tag) at (#2, -.4) {#2};
                }
                {->}{
                	\node [circle, draw, #5, line width = 1.5pt, color = #7, inner sep = 0pt, minimum size = 5pt] (ci) at (#1, #3) {};
                    \draw [->, line width = 1.5pt, color = #7] (ci) -- (#2 + 1.5, #3);
                    \draw (ci) -- (#1, -.2);
                    \node (tag) at (#1, -.4) {#1};
                }
            }
        }
        }[\PackageError{tree}{Undefined option to intervals: #6}{}]
}
%----------------------------------------------------
% Recta Real
\newcommand{\reaLine}[2]{
    \node (li) at (#1 - 1.5, 0) {};
    \node (ls) at (#2 + 1.5, 0) {$\mathbb{R}$};
	\draw [<->] (li) -- (ls);
}