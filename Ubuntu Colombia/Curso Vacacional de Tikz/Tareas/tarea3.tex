\documentclass[tikz, preview=true, border=2mm]{standalone}
\renewcommand*\familydefault{\sfdefault}
\usepackage[utf8x]{inputenc}
\usepackage{tikz}
\usetikzlibrary{mindmap,trees,shadows}
\begin{document}
\makeatletter
\def\tikz@compute@circle@radii@b{%
  \pgf@process{\pgfpointtransformed{\pgfpointanchor{\tikztostart}{center}}}%
  \pgf@xa=\pgf@x%
  \pgf@process{\pgfpointtransformed{\pgfpointanchor{\tikztostart}{west}}}%
  \advance\pgf@xa by-\pgf@x%
  \pgf@xa=1.6\pgf@xa\relax%
  \pgfkeys{/pgf/decoration/start radius/.expanded=\the\pgf@xa}%
  \pgf@process{\pgfpointtransformed{\pgfpointanchor{\tikztotarget}{center}}}%
  \pgf@xa=\pgf@x%
  \pgf@process{\pgfpointtransformed{\pgfpointanchor{\tikztotarget}{west}}}%
  \advance\pgf@xa by-\pgf@x%
  \pgf@xa=1.6\pgf@xa\relax%
  \pgfkeys{/pgf/decoration/end radius/.expanded=\the\pgf@xa}%
}

\def\tikz@compute@segmentamplitude@b{%
  \pgf@x=\pgfkeysvalueof{/pgf/decoration/start radius}\relax%
  \ifdim\pgf@x>\pgfkeysvalueof{/pgf/decoration/end radius}\relax%
    \pgf@x=\pgfkeysvalueof{/pgf/decoration/end radius}\relax%
  \fi%
  \pgf@x=.35\pgf@x\relax%
  \edef\pgfdecorationsegmentamplitude{\the\pgf@x}%
}
\tikzoption{thick bar concept color}{%
  \let\tikz@old@concept@color=\tikz@concept@color%
  \let\tikz@old@compute@circle@radii=\tikz@compute@circle@radii%
  \let\tikz@compute@circle@radii=\tikz@compute@circle@radii@b%
  \let\tikz@old@compute@segmentamplitude=\tikz@compute@segmentamplitude%
  \let\tikz@compute@segmentamplitude=\tikz@compute@segmentamplitude@b%
  \def\tikz@edge@to@parent@path{
    (\tikzparentnode)
    to[circle connection bar switch color=from (\tikz@old@concept@color) to (#1)]
    (\tikzchildnode)}
  \def\tikz@concept@color{#1}%
}
 \tikzoption{standard bar concept color}{%
   \let\tikz@old@concept@color=\tikz@concept@color%
   \let\tikz@compute@circle@radii=\tikz@old@compute@circle@radii%
   \let\tikz@compute@segmentamplitude=\tikz@old@compute@segmentamplitude%
   \def\tikz@edge@to@parent@path{
     (\tikzparentnode)
     to[circle connection bar switch color=from (\tikz@old@concept@color) to (#1)]
     (\tikzchildnode)}
   \def\tikz@concept@color{#1}%
}
\makeatother

\begin{tikzpicture}

% Define experience colors
\colorlet{acolor}{blue!50}
\colorlet{bcolor}{red!75}
\colorlet{ccolor}{orange!80}
\colorlet{dcolor}{teal!70!green}
\colorlet{ecolor}{violet!75}
\colorlet{fcolor}{green!75!black}

\begin{scope}[mindmap,
every node/.style={concept, circular drop shadow, minimum size=0pt,execute at begin node=\hskip0pt, font=\bfseries},
primary/.append style={
    concept color=black, fill=white, line width=1.8ex, text=black, font=\huge\scshape\bfseries,},
level 1 concept/.append style={font=\bfseries},
text=white,
types/.style={thick bar concept color=blue!80!black},
father/.style={standard bar concept color=acolor},
mother/.style={standard bar concept color=bcolor},
brothers/.style={standard bar concept color=ccolor},
university/.style={standard bar concept color=acolor},
group1/.style={standard bar concept color=bcolor},
group2/.style={standard bar concept color=ccolor},
group3/.style={standard bar concept color=dcolor},
group4/.style={standard bar concept color=ecolor},
group5/.style={standard bar concept color=fcolor},
grow cyclic,
level 1/.append style={level distance=6.2cm,sibling angle=360/4},
level 2/.append style={level distance=3cm,sibling angle=80},
level 3/.append style={level distance=3cm,sibling angle=55},
]
\node [primary] (mathematician) {Heisuke\\Hironaka}[rotate=90] % root
    child [types] { node {Familia}
        child [father] { node {\small Papá} }
        child [mother] { node {\small Mamá} }
        child [brothers] { node {\small Hermanos} }
    }
    child [types] { node {\\Matematicas}
        child [group2] { node {\\ Educación }
        	child [group5] { node {\small \\Universidad de\\Kyoto} }
        	child [group5] { node {\small \\Universidad de\\Hiroshima} }
        	child [group5] { node {\small \\Universidad de\\Harvard} }
    	 }
        child [group2] { node {\small Matemáticos\\influyentes} 
        	child [group5] {node {\small Oscar\\Zariski} }
        	child [group5] {node {\small Yasuo\\ Akizuki} }
        	child [group5] {node {\small Alexander\\Grothendieck} }
    	}
    }
	child [types] { node {Música}
		child [group3] { node {\small Piano} }
	}
    child [types] { node {Distinciones}
	child [group4] { node {\small Medalla Fields\\(1970)} 
		child [group5] { node {\small \\Resolución de \\singularidades\\(1964)} }
	}
	child [group4] { node {\small Decano de la \\Universidad\\de\\Yamaguchi\\(1996)} }
	child [group4] { node {\small Ph.D en\\matemáticas\\Harvard\\(1960)} }
	child [group4] { node {\small Profesor del\\Instituto Clay\\(1983)} }
}
;
\end{scope}

\begin{scope}[xshift=-4.5cm, yshift=-10.5cm,every node/.style={text=black}]
\matrix[row sep=0pt,column sep=1mm, nodes={anchor=west}] {
    \fill [acolor] (0,.25ex) circle (1ex); & \node{Dirigente de una fabrica textil, presenció la bomba de Hiroshima.};\\
    \fill [bcolor] (0,.25ex) circle (1ex); & \node{Tercera esposa del papá de Heisuke.};\\
    \fill [ccolor] (0,.25ex) circle (1ex); & \node{Tuvo hasta diez hermanos.};\\
    \fill [fcolor] (0,.25ex) circle (1ex); & \node{Fue influenciado por el matemático húngaro Oscar Zariski};\\
   	\fill [dcolor] (0,.25ex) circle (1ex); & \node{Por muy poco iba a dedicar su vida al piano, pero decidió ser matemático.};\\
   	\fill [fcolor] (0,.25ex) circle (1ex); & \node{Probó que las singularidades algebracias admiten resolución en cuerpos de característica cero.};\\
    };
\end{scope}
\end{tikzpicture}

\end{document}
\colorlet{acolor}{blue!50}
\colorlet{bcolor}{red!75}
\colorlet{ccolor}{orange!80}
\colorlet{dcolor}{teal!70!green}
\colorlet{ecolor}{violet!75}