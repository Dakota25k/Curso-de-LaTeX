\documentclass[a4paper,12pt]{article}
\usepackage{amsmath,amsthm,amssymb}
\usepackage[utf8]{inputenc}
\usepackage[spanish]{babel}
\usepackage[dvipsnames,svgnames]{xcolor}
\usepackage{enumerate} % Paquete para enumeraciones
\usepackage{tikz}
\usetikzlibrary{babel} % Evita conflictos con tikz y spanish
% opening
\title{Segunda clase de \LaTeX}
\author{Grupo Estudiantil de Matemática}
\date{\today}
\usepackage{graphicx} % Se emplea para insertar imágenes
\theoremstyle{theorem}
\newtheorem{teo}{Teorema}[section]
\theoremstyle{definition}
\newtheorem{lema}{Lema}[section]
\newtheorem{defi}{Definición}[section]
\setlength\parindent{0pt}
\pagecolor{BlueGreen}
\color{Sepia}
\begin{document}
	\maketitle
	Algún texto texto mucho texto.
	\begin{teo}
		Sea \( f_i \colon \mathbb{R} \to \mathbb{R}^n \) una función vectorial de variable real.
	\end{teo}
	\begin{lema} 
		Este lema no tiene sentido \( x^2+y^2=1 \). Luego tenemos 
		\[ f(x)=\begin{cases} 5 & \mbox{ si } x<5 \\ 3 & \mbox{ si } x\ge5  \end{cases} \]
	\end{lema}
	
	Algunas fracciones: 
	
	El frac es para textos \( \frac{5}{4} \) vs \( \dfrac{5}{4} \) en línea y se ajusta automáticamente si es centrada o en línea.
	
	\[ \frac{5}{4} \mbox{ comparada con } \dfrac{5}{4}  \]

	\section{Los paréntesis}
	
	\[   \left(  \int^\int  \right)  \]
	  \[ \left\lbrace \iint \right\rbrace \] 
	  
	 \[ \left. \int \right\rbrace ~~~ \left\lbrace \int \right. \]
	 
	 
	\[  f(x)= \begin{cases}
		 	x-1 & \mbox{ si } x<1 \\
		 	x+1 & \mbox{ si } x>1 
	       \end{cases} \]
	       
	       
	       \begin{itemize}
	       	\item Cálculo Vectorial % control + shift + i 
	       	\item Análisis Numérico
	       	\item Análisis Complejo
	       	\item Análisis de Fourier
	       \end{itemize}
	       
\section{Matrices}

\[   \begin{pmatrix}
2 & 3 & 4 \\ 
1 & 2 & 3
\end{pmatrix}   \]

\[   \begin{bmatrix}
2 & 3 & 4 \\ 
1 & 2 & 3
\end{bmatrix}   \]

\[   \begin{pmatrix}
2  \\  3 \\ 4 
\end{pmatrix}   \]

\begin{tabular}{||l|c|r||}
	\hline Nombre & Edad & Carrera \\ 
	\hline Carlos & 21 & N2 \\ 
	\hline Junior & 19 & N2 \\ 
	\hline Cynthia & Muy joven & Ingeniería Física \\ 
	\hline Jaime & Muy joven & Maravilloso \\ 
	\hline\hline
\end{tabular} 
 
\begin{center}
	\includegraphics{a}
\end{center}
\( {\int }_{2}^{3}xdx \)%nota es posible que te genere error ya que la ubicación del archivo con el que fue creado no está en tu pc. Para solucionar este error, clic en asitentes, luego insertar gráfico, seleccionar el archivo y clic en OK. ¡Gracias!

\section{Tikz}
\begin{tikzpicture}
	\draw[color=green, dotted] (0,0) -- (5,0);
	\draw[draw=red, fill= blue, dashed] (5,3) circle(3);
	\draw (1,2) node[left] { \( \mbox{ Este es un texto} \)};
	                % right  Drecha 
	                % above  Arriba 
	                % below  Abajo
	
	\draw (3,6) node[left] { \( \int xdx \)};	
\end{tikzpicture}

\end{document}