\documentclass[12pt,a4paper]{article}
\usepackage[utf8]{inputenc}
\usepackage[spanish]{babel}
\usepackage{amsmath}
\usepackage{amsfonts}
\usepackage{amssymb}
\usepackage{graphicx}
\usepackage{amsfonts}
\usepackage[dvipsnames,svgnames]{xcolor}
\usepackage{fancyhdr}
\lhead{Curso de \LaTeX}
\chead{Apuntes de clase}
\rhead{Clase 1}
\lfoot{\textit{Grupo Estudiantil de Matemática}}
\cfoot{\thepage}
\rfoot{\today}
\pagestyle{fancy}
\usepackage[left=2cm,right=2cm,top=2cm,bottom=3cm]{geometry}
\usepackage{lscape}
\usepackage{amsthm}
\pagecolor{GreenYellow}
\color{Sepia}
\title{Apuntes de clases de \LaTeX Verano 2017}
\author{Carlos Alonso Aznarán Laos}
\date{\today}
\title{Apuntes de clases de \LaTeX}

\begin{document}
\maketitle
\section{Introducción al curso}
\subsection{Expositor}
\noindent Estudiante de matemática: Luis Felipe Villavicencio López\\E-mail: lvillavicenciol@uni.pe
\subsection{Organizador}
\noindent Estudiante de matemática: Carlos Alonso Aznarán Laos\\E-mail:
caznaranl@uni.pe
\subsection{Horario}
\noindent Sábado 28 de febrero: 10:00 A.M. (Primera Clase)\\
Miércoles 1 de febrero: 4:00 P.M a 6:00 P.M. (Segunda clase)
\subsection{Lugar}
\noindent Sala del Centro de Estudiantes de la Facultad de Ciencias. (Primera clase)\\
Sala de cómputo N$^{\circ} 1$ Tercer piso Facultad de Ciencias (Segunda clase).
\pagebreak

\begin{center}
\section*{Primera clase}
\noindent La extensión con la cual se guarda todo archivo en \LaTeX\\ se llama
\texttt{.tex} y genera archivos \texttt{.pdf} por lo general.
\end{center}
\begin{verbatim}
\documentclass{article} % también puede estar dentro de las llaves book,
letter, report, beamer
\end{verbatim}
\flushleft{\texttt{$\backslash$usepackage[optional]\{obligatorio\}} \Bigg \} Preámbulo}\\
\flushleft{\texttt{$\backslash$begin\ldots} \Bigg \} Cuerpo}
\begin{verbatim}
\end{document}
\end{verbatim}
Para escribir en el teclado $\backslash$ se puede realizar usando la combinación de teclas \texttt{Alt + 92} o sino \texttt{Alt Gr + tecla ubicada al lado derecho del 0}.
\linebreak
Para compilar basta con pulsar la tecla \texttt{F5}.
\linebreak
Para realizar un comentario, se escribe primero \texttt{\%} y se usa con el propósito de hacer el código fuente más fácil de entender con vistas a su mantenimiento o reutilización.
\linebreak
Estos son algunos de los paquetes más populares que se usan en  \LaTeX :
\linebreak

$\backslash$usepackage\{amsmath\}$  \rightarrow  \int \iint \oint \sum$\\
$\backslash$usepackage\{amssym\}$  \rightarrow N^{\circ}$\\
$\backslash$usepackage\{amsthm\}$  \rightarrow Teoremas, pruebas$\\
Otros paquetes: Tikz, asymptote, graphicxs. Todos ellos paquetes de dibujos e inclusión de gráficos.
\linebreak
\begin{verbatim}
\usepackage[spanish]{babel}
\usepackage[T1]{fontenc}
\usepackage[utf8]{inputenc}
\noindent omitir la indentación o sangría en una línea
\setlenght\parindent{0} sirve para borrar la indentación de todo el documento.
\end{verbatim}
Pulsar las teclas \texttt{Ctrl + e} para crear un \\$\backslash begin\{\}$ \\ $\backslash end\{\}$\\
Para crear un documento nuevo:
\begin{verbatim}
\documentclass[12pt,a4paper]{article}
\usepackage[spanish]{babel}
\usepackage[T1]{fontenc}
\usepackage[utf8]{inputenc}
\institute
\author{Yo\\Tú}
\title{Título normal}
\date{\today}
\begin{document}
\maketitle
\end{document}
\end{verbatim}

\end{document}