\documentclass[handout]{beamer}
\usepackage[spanish]{babel}
\usepackage[utf8]{inputenc}
\usepackage{amsmath,amsthm,amssymb,enumerate}
\author{Grupo Estudiantil de Matemática}
\title{Primera presentación en Beamer}
\institute{\LARGE{Universidad Nacional de Ingeniería\\Facultad de Ciencias}}
\date{8 de febrero del 2017} % Si deseas mostrar la fecha actual, entonces escribe \date{\today}
\newcommand{\ds}{\displaystyle}
\newcommand{\re}{\mathbb{R}}
\usetheme{Antibes}
\begin{document}

	\begin{frame}
		\maketitle
	\end{frame}

	\begin{frame}{Esto es un título}{Esto es un subtitulo}
	Esto es un frame, aquí podemos escribir fórmulas como \(\ds\int_S f(x,y)dA \).\\
	También podemos crear tablas o ecuaciones centradas.
	\[1+2+3+4+5+6+\cdots+n = {\color{blue}\frac{n(n+1)}{2}}\]
	{\color{red} Texto a ser coloreado de rojo} 
	\begin{center}
		
		\begin{tabular}{||c||c||}
			\hline 1 & 2 \\ \hline
			\hline 3 & 4 \\ \hline
			\hline 
		\end{tabular}
		
	\end{center}
	
\end{frame}

\begin{frame}{Tipos de Cajas}
	
	\begin{exampleblock}{Exampleblock}
	Texto o fórmulas en \LaTeX o imágenes o cuadros.
	\[f(x)\subset \re\]
\end{exampleblock}
	
\begin{alertblock}{Alertblock}
	Esta caja es de color rojo.
	
	\begin{center}
		\begin{tabular}{||c||c||}
			\hline 1 & 2 \\ \hline						
		\end{tabular} 
	\end{center}
	
\end{alertblock}

\begin{block}{Block}
	Caja pequeña de color azul.
\end{block}	
\end{frame}
	\begin{frame}{El poder de \LaTeX}
	A continuación aparecerá una caja. \pause 
	\begin{alertblock}{¡Apareció la caja!}
		Matemáticas
		\begin{itemize}
			\item Primera ítem \pause
			\item Segunda ítem  \pause
			\item Una ecuación \pause \[x^0+y^0=2,~\forall x,y\in\re\setminus\{0\}\]
		\end{itemize}
		
	\end{alertblock}
	
\end{frame}

\begin{frame}
	\[1+2+3+4+5+6= \pause(1+2+3)+\pause(4+5+6) = \pause6 + \pause 15= \pause 21\]
	
	\pause Para eliminar los velos cambiar\\
	 $\backslash\!\!$ documentclass[handout] por  $\backslash\!\!$ documentclass[beamer]\\\Large{¡Gracias por tu atención!}
\end{frame}
\end{document}