\documentclass[12pt,a4paper]{article}
\usepackage[utf8]{inputenc}
\usepackage[spanish]{babel}
\usepackage{amsmath}
\usepackage{amsfonts}
\usepackage{amssymb}
\usepackage{graphicx}
\usepackage[left=2cm,right=2cm,top=2cm,bottom=2cm]{geometry}
\author{Carlos Alonso Aznarán Laos}
\title{Curso de \LaTeX \\ Práctica  N$ ^{\circ}$2}
\begin{document}
\maketitle
\noindent Usando el comando Por ejemplo, $N=\{1, 2, 3, 4, \ldots \}$ es un conjunto infinito.\\
Usando el comando Por ejemplo, \(N=\{1, 2, 3, 4, \ldots \}\) es un conjunto infinito.\\
Usando el comando Por ejemplo,  \begin{math}N=\{1, 2, 3, 4, \ldots \} \end{math} es un conjunto infinito.
\\
$H_0 = \mu_{con\: programa}= 11$ (programa no tuvo efecto)\\
$a^{m}\cdot a^{n} = a^{m+n}$\\
$\left(\frac{a}{b}\right)^{n}=\frac{{a}^{n}} {{b}^{n}}$, donde: $(b\neq0)$\\
$\sqrt[n]{ab} = \sqrt[n]{a}\cdot \sqrt[n]{b}$\\
La propiedad $\sqrt[n]{\frac{a}{b}} = \frac{\sqrt[n]{a}}{\sqrt[n]{b}}$, donde: $(b\neq0)$\\
La propiedad $\displaystyle\sqrt[n]{\frac{a}{b}} = \frac{\sqrt[n]{a}}{\sqrt[n]{b}}$, donde: $(b\neq0)$\\
‘‘Nuestro trabajo es, por lo tanto, mostrar que la integral... \begin{displaymath} \mathnormal{\int_{0}^{1} \,\, (\widehat{f} (\alpha))^{3}e(\alpha n) d \alpha} \end{displaymath} es distinta de cero." Helfgott, 2013: 714.

\end{document}