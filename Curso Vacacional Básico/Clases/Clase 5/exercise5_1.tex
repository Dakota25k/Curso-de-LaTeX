\documentclass{article}
\usepackage[utf8x]{inputenc}
\usepackage[spanish]{babel}
\spanishdatedel
\usepackage{graphicx}
\usepackage{amsmath}
\usepackage{lipsum}
\usepackage{vmargin}
\setpapersize{A4}
\setmargins{2.5cm}{1.5cm}{16.5cm}{23.42cm}{10pt}{1cm}{0pt}{1cm}

\usepackage{xcolor}

\definecolor{myGreen}{HTML}{36A736}

\definecolor{myBlue}{HTML}{02528F}

\definecolor{my}{HTML}{FF4312}

\title{Nuestro título}
\author{Oromion}
\date{\today}

\renewcommand{\thesection}{\alph{section}}
\newcounter{obs}
\newcommand{\obs}{\stepcounter{obs}{\bf Observación \theobs:}}

\begin{document}

\maketitle
\renewcommand{\contentsname}{Tabla de contenido}
\tableofcontents

\section{Grafo ponderado} \label{ponderado}

\lipsum[1-2]

\obs{} Esta es una primera observación

\section{Grafo nulo} \label{nulo}

\lipsum[1-3]
Aquí nos vamos a referir a la sección \ref{ponderado} actual con número \pageref{ponderado}.

\begin{equation} \label{media}
	\bar{x}=\frac{1}{n}\sum_{i=1}^{n}a_i
\end{equation}
La ecuación de la media tiene enumeración \eqref{media}.
%minuto 40
\noindent
La sección \thesection aparece en la página \thepage .

\obs{} Esta es una observación importante \LaTeX{} este es el comando

\subsection{Generación de contadores}

\lipsum[1]

\subsubsection{Referencias cruzadas}

\lipsum[3]

\end{document}