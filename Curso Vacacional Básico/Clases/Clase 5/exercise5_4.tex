\documentclass[10pt,a4paper]{article}
\usepackage[utf8x]{inputenc}
\usepackage[spanish]{babel}
\spanishdatedel
\usepackage{amsmath,amsfonts,amssymb,amsthm}
\usepackage{graphicx}
\usepackage{hyperref}
\usepackage{lipsum}
\usepackage{vmargin}
\setpapersize{A4}
\setmargins{2.5cm}
{1.5cm}
{16.5cm}
{23.42cm}
{10pt}
{1cm}
{0pt}
{1cm}

%\usepackage{xcolor}

\definecolor{myGreen}{HTML}{36A736}

\definecolor{myBlue}{HTML}{02528F}

\definecolor{my}{HTML}{FF4312}


\title{Ejemplo lección 5}
\author{Oromion}
\date{\today}

%\renewcommand{\thesection}{\roman{section}}
\renewcommand{\thesection}{\alph{section}}
\newcounter{obs} % Aumenta de uno en uno.
\newcommand{\obs}{\stepcounter{obs}{\bf Observación \theobs:}} % Primero aumenta en uno y después imprime.
\begin{document}
\maketitle
\renewcommand{\contentsname}{Tabla de contenidos}
\tableofcontents

\section{Grafo ponderado} \label{ponderado}

\lipsum[1-2]

\obs{} Esta es una primera observación.

\section{Grafo nulo} \label{nulo}

\lipsum[1-3]


Aquí nos vamos a referir a la sección \ref{ponderado} %\thesection \;que
que aparece  en la página \pageref{ponderado}%\thepage

\begin{equation} \label{media}
	\bar{x} = \frac{1}{n}\sum_{i=1}^{n}a_i
\end{equation}

\obs{} Esta es una observación importante.

\subsection{Generación de contadores}

\lipsum[1]

\subsection{Referencias cruzadas}

\lipsum[1]

La ecuación de la media \eqref{media} que aparece en la página \pageref*{media}.


Un enlace dinámico al canal de Youtube del profesor Juan Espejo utilizando \texttt{href} \href{https://www.youtube.com/channel/UCmQYbAWGwSTMJAk5Q_orLbg}{Espejo}

%Enlace a documento externo \href{}{}
\begin{align*}
	x = a \tag{A1} \label{A1} \\
	y = b \tag{A2} \label{A2} \\
\end{align*}
\end{document}
Un contador es un ambiente o un comando.

%Los parágrafos no se enumeran como 1.1.1.1.1.1.1.1