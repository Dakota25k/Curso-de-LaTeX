%Tarea de clase 6
\documentclass[12pt]{beamer}
\setbeamercolor{title}{bg = magenta!50!black, fg= white}
\usepackage[utf8]{inputenc}
\usepackage[spanish]{babel}
\spanishdatedel
\uselanguage{Spanish} % Teoremas en idioma que elijas.
\languagepath{Spanish}
\input{code.tex}
\usepackage{cprotect} % Allow \cprotect for display code with \verb|code|
\usepackage{graphicx}
\usetheme{Frankfurt}
\usecolortheme{dolphin}
\usefonttheme{professionalfonts}

\begin{document}
	\author{Matemático Oromion}
	\title{Renombrando comandos en \LaTeX{} con}
	\cprotect\subtitle{\verb|\renewcommand{cmd}[args]{def}|}
	\logo{\LaTeX{}}
	\institute{\large Universidad del Perú}
	\date{\today}
	\subject{Fin del curso}
	\setbeamercovered{transparent}
	\setbeamertemplate{navigation symbols}{}
	\begin{frame}[plain]
	\maketitle
\end{frame}


\begin{frame}{Contenidos}{}
\tableofcontents
\end{frame}
\section{Nuevo comando}
\begin{frame}
\frametitle{\thesection.}
\begin{examples}{Comando \cmd{newcommand}}

Define un nuevo comando y comete un error si ya está definido.
\end{examples}
\end{frame}
\section{Definiendo nuevo comando}


\begin{frame}
\frametitle{\thesection.}
\begin{block}{Comando \cmd{renewcommand}}
Redefine un comando predefinido y comete un error si aún no está definido.
\end{block}
\end{frame}
\section{Comando robusto @}
\begin{frame}
\frametitle{\thesection.}
\begin{alertblock}{Comando \cmd{providecommand}}
Define un nuevo comando si aún no está definido.
\end{alertblock}
\end{frame}
\end{document}