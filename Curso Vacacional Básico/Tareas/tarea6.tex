%Tarea de clase 6
\documentclass[12pt]{beamer}
\setbeamercolor{title}{bg = magenta!50!black, fg= white}
\usepackage[utf8]{inputenc}
\usepackage[spanish]{babel}
\spanishdatedel
\uselanguage{Spanish} % Teoremas en idioma que elijas.
\languagepath{Spanish}
\input{code.tex}
\usepackage{cprotect} % Allow \cprotect for display code with \verb|code|
\usepackage{graphicx}
\usepackage{verbatim}
\usetheme{Frankfurt}
\usecolortheme{dolphin}
\usefonttheme{professionalfonts}

\begin{document}
\author{Matemático Oromion}
\title{\Large Renombrando comandos en \LaTeX{} con}
\cprotect\subtitle{\verb|\renewcommand{cmd}[args]{def}|}
\logo{\includegraphics[scale=.5]{peru.jpg}}
\institute{\large Universidad del Perú}
\date{\textcolor{magenta}{\today}}
\subject{Fin del curso}
\setbeamercovered{transparent}
\setbeamertemplate{navigation symbols}{}

\begin{frame}[plain]
	\maketitle
\end{frame}

\begin{frame}{Contenidos}{Presentación final}
\tableofcontents
\end{frame}

\section{Introducción}
\cprotect\subsection{Entiendiendo el comando: \verb!\newcommand{\newbox}[args]{def}!}

\begin{frame}[fragile]
\frametitle{\thesection.\thesubsection. Entendiendo el comando \cmd{newcommand}}
\vspace*{-2cm}
\centering
\verb|\newcommand{cmd}[args]{def}|

\begin{columns}
\column{0.5\textwidth}
Se desea nombrar a un color.

\begin{itemize}
	\item El primer argumento obligatorio recibe el nombre del \cmd{comando}.
	\item 
\end{itemize}
\column{0.5\textwidth}
\begin{examples}{Comando \cmd{newcommand}}
\begin{enumerate}
\item[$\bullet$] A.
\end{enumerate}
\end{examples}

\begin{alertblock}{¡Cuidado!}
Se comete un error si el comando ya está definido.
\end{alertblock}
\end{columns}

\end{frame}

\cprotect\subsection{Aprendiendo el comando: \verb!\renewcommand{cmd}[args]{def}!}

\begin{frame}[fragile]
\frametitle{\thesection.\thesubsection. Aprendiendo el comando \cmd{renewcommand}}
\vspace*{-2cm}
\centering
\verb|\renewcommand{cmd}[args]{def}|

\begin{block}{Comando \cmd{renewcommand}}
Redefine un comando predefinido y comete un error si aún no está definido.
\end{block}
\end{frame}

\cprotect\subsection{Extra: \verb!\providecommand{cmd}[args]{def}!}
\cprotect\subsection{Usando el comando robusto: \verb!\@!}

\section{Ejemplos}

\subsection{Definiendo el estilo del modo matemático de una ecuación en derivadas parciales}
\subsection{Creando un comando de una caja}

\begin{frame}
\frametitle{\thesection.\thesubsection. Extra: \cmd{providecommand}}

\begin{alertblock}{Comando \cmd{providecommand}}
Define un nuevo comando si aún no está definido.
\end{alertblock}
\end{frame}

\section{\thesection. Agradecimientos}
\subsection{Ubuntu Colombia}
\begin{frame}
\frametitle{Agradecimientos}
 \end{frame}
\end{document}