%
% Clase 1 LaTeX GEM.tex -- a document for writing notes with GEM.
%
% Copyright © 2017 Oromion <caznaranl@uni.pe>
%
% This program is free software: you can redistribute it and/or modify
% it under the terms of the GNU General Public License as published by
% the Free Software Foundation, either version 3 of the License, or
% (at your option) any later version.
%
% This program is distributed in the hope that it will be useful,
% but WITHOUT ANY WARRANTY; without even the implied warranty of
% MERCHANTABILITY or FITNESS FOR A PARTICULAR PURPOSE.  See the
% GNU General Public License for more details.
%
% You should have received a copy of the GNU General Public License
%s along with this program.  If not, see <http://www.gnu.org/licenses/>.
%
\documentclass[12pt,a4paper]{article}
\usepackage[utf8x]{inputenc}
\usepackage[spanish]{babel}
\usepackage{amsmath,amsfonts,amssymb,amsthm}
\usepackage{graphicx}
\usepackage[usenames,dvipsnames,x11names,table,svgnames]{xcolor}
\usepackage[colorlinks=true,urlcolor=blue,linkcolor=black,anchorcolor=black,citecolor=black]{hyperref}
\usepackage{fancyhdr}

\lhead{Curso de \LaTeX}
\chead{Apuntes de clase}
\rhead{Clase 1}
\lfoot{\textit{Grupo Estudiantil de Matemática}}
\cfoot{\thepage}
\rfoot{31 de enero del 2017} % today
\pagestyle{fancy}
\usepackage[left=2cm,right=2cm,top=2cm,bottom=3cm]{geometry}
\usepackage{lscape}
\pagecolor{GreenYellow}
\color{Sepia}

\title{Apuntes de clases de \LaTeX{} Verano 2017}
\author{Carlos Alonso Aznarán Laos}
\date{31 de enero del 2017} %\today

\setlength\parindent{0pt} % \noindent automático en todo el documento.

\begin{document}

\maketitle

\section{Introducción al curso}

\subsection{Expositor}

Estudiante de matemática: Luis Felipe Villavicencio López.

E-mail: \href{mailto:lvillavicenciol@uni.pe}{lvillavicenciol@uni.pe}

\subsection{Organizador}

Estudiante de matemática: Carlos Alonso Aznarán Laos.

E-mail: \href{mailto:caznaranl@uni.pe}{caznaranl@uni.pe}

\subsection{Horario}

Sábado 28 de febrero: 10:00 A.M. (Primera Clase).

Miércoles 1 de febrero: 4:00 P.M a 6:00 P.M. (Segunda clase).

\subsection{Lugar}

Sala del Centro de Estudiantes de la Facultad de Ciencias. (Primera clase).

Sala de cómputo N$^{\circ} 1$ Tercer piso Facultad de Ciencias (Segunda clase).

\pagebreak

\begin{center}
\section*{Primera clase}

La extensión con la cual se guarda todo archivo en \LaTeX\\ se llama
\texttt{.tex} y genera archivos \texttt{.pdf} por lo general.
\end{center}

\begin{verbatim}
    \documentclass{article} % También puede estar dentro de las llaves: book,
        letter, report, beamer.
\end{verbatim}

\flushleft{\texttt{$\backslash$usepackage[optional]\{obligatorio\}} \Bigg \} Preámbulo}\\
\flushleft{\texttt{$\backslash$begin\ldots} \Bigg \} Cuerpo}
\begin{verbatim}
\end{document}
\end{verbatim}

Para digitar el caracter ``backslash'' ($\backslash$) se puede realizar usando la combinación de teclas \texttt{Alt + 92} o sino \texttt{Alt Gr + tecla ubicada al lado derecho del 0}.

Para compilar basta con pulsar la tecla \texttt{F5}.

Para realizar un comentario, se escribe primero \texttt{\%} y se usa con el propósito de hacer el código fuente más fácil de entender con vistas a su mantenimiento o reutilización.

Estos son algunos de los paquetes más populares que se usan en \LaTeX{}:

$\backslash$usepackage\{amsmath\}$\rightarrow  \int \iint \oint \sum$\\
$\backslash$usepackage\{amssym\}$  \rightarrow \text{N}^{\circ}$\\
$\backslash$usepackage\{amsthm\}$  \rightarrow \text{Teoremas, pruebas}$.

Otros paquetes: Tikz, asymptote, graphicxs. Todos ellos paquetes de dibujos e inclusión de gráficos.

\begin{verbatim}
    \usepackage[spanish]{babel}
    \usepackage[T1]{fontenc}
    \usepackage[utf8]{inputenc}
    \noindent % Omite la indentación o sangrado en una línea.
    \setlenght\parindent{0} % Omite la indentación de todo el documento.
\end{verbatim}

Pulsar las teclas \texttt{Ctrl + e} para crear un nuevo entorno en TeXstudio 

\begin{center}
    \begin{verbatim}
        \begin{}
        \end{}    
    \end{verbatim}    
\end{center}

Para crear un documento nuevo:

\begin{center}
    \begin{verbatim}
        \documentclass[12pt,a4paper]{article}
        \usepackage[spanish]{babel}
        \usepackage[T1]{fontenc}
        \usepackage[utf8x]{inputenc}
        \institute
        \author{Yo\\Tú}
        \title{Título grandioso}
        \date{\today}
        \begin{document}
        \maketitle
        \end{document}
    \end{verbatim}    
\end{center}

\end{document}