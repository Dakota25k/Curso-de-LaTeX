% !TeX document-id = {5113de6a-bb6a-4187-b852-9c4e386e8850}
% !TeX TXS-program:compile = txs:///lualatex
\documentclass{standalone}
\usepackage{pgfplots}
\pgfplotsset{compat=1.8}
\usepackage{verbatim}
\begin{document}
\begin{comment}
Por favor compile com lualatex. Demora vários minutos, seja paciente
Exemplo do site
https://texwelt.de/wissen/fragen/6952/kann-man-bei-pgfplots-die-argumente-trigonometrischer-funktionen-von-grad-auf-radiant-umstellen
\end{comment}
\begin{tikzpicture}
  \begin{axis}[
      view = {60}{-60},
      hide axis,
      colormap = {flower}{%
        color(0cm)  = (yellow);       color(4cm)  = (orange);
        color(8cm)  = (red!40!black); color(12cm) = (red!80!black); 
        color(16cm) = (green);        color(20cm) = (green!20!black)},
    ]
    \addplot3[
      surf,
      z buffer = sort,       point meta = u,
      domain   = -13.2:13.2, domain y   = -37.4:37.4,
      samples  = 80,         samples y  = 120,
      variable = \u,         variable y = \v      
    ] 
    ( { -u + (2*0.84*cosh(0.4*u)*sinh(0.4*u))/(0.4*((sqrt(0.84)*cosh(0.4*u))^2
        + (0.4*sin(deg(sqrt(0.84)*v)))^2)) },
      { (2*sqrt(0.84)*cosh(0.4*u)*(-(sqrt(0.84)*sin(deg(v))
        * cos(deg(sqrt(0.84)*v))) + cos(deg(v))*sin(deg(sqrt(0.84)*v))))/(0.4
        * ((sqrt(0.84)*cosh(0.4*u))^2 + (0.4*sin(deg(sqrt(0.84)*v)))^2)) },
      { (2*sqrt(0.84)*cosh(0.4*u)*(-(sqrt(0.84)*cos(deg(v))
        * cos(deg(sqrt(0.84)*v))) - sin(deg(v))*sin(deg(sqrt(0.84)*v))))/(0.4
        * ((sqrt(0.84)*cosh(0.4*u))^2 + (0.4*sin(deg(sqrt(0.84)*v)))^2)) });
    \end{axis}
\end{tikzpicture}
\end{document}