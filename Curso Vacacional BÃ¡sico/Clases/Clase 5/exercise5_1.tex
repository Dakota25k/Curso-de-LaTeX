% Título de secciones
%------------------------------------------------
% \titleformat{#1}[#1]{#2}{#3}{#4}{#5}[#2]
% {#1} \part, \chapter, \section, \subsection, \subsubsection, \paragraph, \subparagraph
% [#1] shape (determine la forma en que está dado el título)
% Si se usa nodo Tikz, no es necesario usar el parámetro opcional shape.
% {#2} format
%se define el formato. Se puede dejar vacío.
% {#3} label
% puede entrar en conflicto si se usa palabras confusas, i.e, parte o part.
% {#4} sep
% {#5} before-code
% lo que va antes del título, se puede usar tikz
% [#2] after code
% lo que va desués del título

%\titleformat{command}[shape]{format}{label}{sep}{before-code}[after-code]
\RequirePackage{xcolor}
\definecolor{color2}{HTML}{ACB8F1}
\definecolor{color1}{HTML}{089880}
%\newcommand{\titleSec}[1]{
%	\tikz \node[rectangle, text width = \textwidth, align = right, draw] (tS) at (0,0) {\thesection. #1};
%}
%comentó lo de tikz.

\titleformat{\section}{\Huge\bfseries\flushright}{}{0pt}{\textcolor{color2}}{\thesection. #1}}
\titleformat{\subsection}{\Large\bfseries}{}{0pt}{\textcolor{color1}{\thesubsection. #1}}[\vspace{1cm}]

\pagestyle{fancy}

\ProcessOptions \relax
%\titleformat{\section}[box]{\Huge\bfseries}{}{0pt}{#1}{}

Tarea es modificar el \titleformat.