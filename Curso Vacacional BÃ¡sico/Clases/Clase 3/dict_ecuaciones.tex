\usepackage{amsmath}
\newcommand{\cuadratica}{
\[
	x_{1,2} = \frac{1}{2a}\Bigl(-b\pm \sqrt{b^{2}-4ac}\Bigr)
\tag{A}
\]
}
%No suma al conteo de ecuaciones
%IfeqCase es como el Swithcase
\usepackage{xstring} % Paquete necesario para cargar condicionales en \newcommand. 

\newcommand{\Fourier}[2][eq]{ % Si no recibe parámetro de entrada, entonces se numerará. Se define dos parámetros de entrada. Parámetro opcional es inLine y el parámetro obligatorio es el color.
	\IfEqCase {#1}{
		{eq}{
			\begin{equation}
				\hat{f}(\xi) = \int_{-\infty}^{\infty}f(x)e^{-2\pi ix\xi}dx
			\end{equation}
			}
		{disp}{
			\[
			\hat{f}(\xi) = \int_{-\infty}^{\infty}f(x)e^{-2\pi ix\xi}dx
			\]
		}
		{inLine}{
			\textcolor{#2}{$\hat{f}(\xi) = \int_{-\infty}^{\infty}f(x)e^{-2\pi ix\xi}dx$}
		}
	}
}

\newcommand{\Wave}{
\begin{equation}
	\frac{\partial ^{2}u}{\partial t^{2}} = c^{2}\frac{\partial^{2}u}{\partial x^{2}}
\end{equation}
}

\newcommand{\Maxwell}{
	\begin{align*}
		\nabla \cdot E &= 0 & \nabla \times E &= -\frac{1}{c}\frac{\partial H}{\partial t} \\
		\vspace{5pt} \\
		\nabla \cdot H &= 0 & \nabla \times H &= \frac{1}{c}\frac{\partial E}{\partial t}
	\end{align*}
}

\newcommand{\Schrodinger}{
	\begin{equation}
		i\hbar \frac{\partial}{\partial t}\psi = \hat{H}\psi
	\end{equation}
}