\documentclass{standalone}
\usepackage[utf8x]{inputenc}
\usepackage[spanish]{babel}
\spanishdatedel
\usepackage[T1]{fontenc}
\usepackage{libertine}
\usepackage{mathtools,amssymb,amsfonts,amsmath,amsthm,mathrsfs,bm,times,bbold}
\usepackage[usenames,dvipsnames,x11names,table,svgnames]{xcolor}
\usepackage{tikz}
\usetikzlibrary{trees,calc}
\begin{document}
\begin{tikzpicture}[grow cyclic,
level 1/.style = {level distance = 1cm, sibling angle = 45},
level 2/.style = {level distance = .5cm, sibling angle = 30},
level 4/.style = {level distance = .3cm, sibling angle = 15}
]

%via three points = {one child at (0,1) and two children at (-1,1) and (1,1)}
%Distnacia entre la raíz y el nodo destino es 1cm
	%\node (R) at (0,0) {raíz} child; % Ramificación. 2 hijos, conviene usar \foreach
	\node (r1) at (0,0) {R1} child child; % Las coordenadas apuntan al centro del nodo.
	%($(R) + (0,1.5)$)
	
	%\coordinate [rotate = 90; %]at (0,0);
	child \foreach \i in {1, 2, 3}{
		child \foreach \j in {1, 2, 3}{
		
		}
	};
	%\node 
\end{tikzpicture}
\end{document}
En el mapa mental se puede enlazar a un archivo.
Mediante el paquete ipref. Documentos dinámicos y referencias cruzadas.
Se gira en sentido horario, pero se puede definir en sentido antihorario.