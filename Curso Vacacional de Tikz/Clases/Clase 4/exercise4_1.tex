\documentclass{standalone}
\usepackage[utf8x]{inputenc}
\usepackage[spanish]{babel}
\spanishdatedel
\usepackage[T1]{fontenc}
\usepackage{libertine}
\usepackage{mathtools,amssymb,amsfonts,amsmath,amsthm,mathrsfs,bm,times,bbold}
\usepackage[usenames,dvipsnames,x11names,table,svgnames]{xcolor}
\usepackage{tikz}
\usetikzlibrary{mindmap,trees,backgrounds}
\begin{document}
\begin{tikzpicture}[
mindmap,
%concept color = black,
text = white, % Es automático
%root concept/.append style = {concept color = black},
%level 1 concept/.append style = {concept color = blue},
%level 2 concept/.append style = {concept color = green!75!black}
]

\node[concept] {concepto principal} % Concepto Raíz
	child[grow = -45] { %, concept color = blue
		node[concept] (PC) {Primer concepto} [clockwise from = -60]
			child{node[concept] (C221) {Concepto 2.2.1}}
			child{node[concept] (C222) {Concepto 2.2.2}}
	} %Nivel 1
	child[grow = 225, concept color = blue] { %concept color = green!75!black
		node[concept] (CH) {Concepto hermano} [clockwise from = -60]		child[concept color = orange]{node[concept] (C121) {Concepto 1.2.1}}
		%,concept color = orange
		child[concept color = green!]{node[concept] (C122) {Concepto 1.2.2}}
	};%Nivel 1 child ``hace cercer''

\node[annotation] at (CH.east) (Esto es un comentario);

\begin{pgfonlayer}{background}
\draw[concept connection] (PC) edge (CH);%de mindmap
%¿Qué significa edge?
\draw[concept connection] (C121) edge (C221);
\draw[concept connection] (PC) edge (C122);
\end{pgfonlayer}
\end{tikzpicture}
\end{document}
En el mapa mental se puede enlazar a un archivo.
Mediante el paquete ipref. Documentos dinámicos y referencias cruzadas.