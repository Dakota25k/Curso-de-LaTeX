\documentclass{standalone}
\usepackage[utf8x]{inputenc}
\usepackage[spanish]{babel}
\usepackage[T1]{fontenc}
\usepackage{PTSansNarrow}
\usepackage[usenames,dvipsnames,x11names,table,svgnames]{xcolor}
\usepackage{amsthm,mathtools,amsfonts}
\usepackage{tikz}
\usetikzlibrary{calc,babel,through,intersections,backgrounds}
\newtheorem{prop}{Proposición}
\setcounter{prop}{1}
\begin{document}
\begin{tikzpicture}[remark/.style = {line width = 1.125pt, opacity = .5}, scale=2]
%Paso 1: Dibujar los puntos A, B, C y D.
\coordinate[label = left:\textcolor{DarkBlue}{$\mathbf{A}$}] (A) at (0, 0);
\coordinate[label = right:$\mathbf{B}$] (B) at (1.25, .15);
\coordinate[label = above:$\mathbf{C}$] (C) at (2, 2.15);
\coordinate[label = above:{$\mathbf{D}$}] (D) at ($(A)!.5!(B)!{2*sin(60)}!90:(B)$);
\draw (A) -- (B) -- (C);

%Paso 2: Dibujar la circunferencia H.
\node[name path = H, label =135:$\mathcal{C}\colon\mathbf{H}$, draw, line width =1pt, circle through = (C)] at (B) {};

%Paso 3: Trazar los segmentos DF y DE. 
\draw[name path = D-F] (D) -- ($(D)!4!(B)$) coordinate[label = below:$\mathbf{F}$] (F);
\draw[name path = D-E] (D) -- ($(D)!3.5!(A)$) coordinate[label = below:$\mathbf{E}$] (E);

%Paso 4: Encontrar el punto de intersección G.
\path[name intersections = {of = H and D-F, by = {[label = left:$\mathbf{G}$]G}}];

%Paso 5: Dibujar la circunferencia K.
\node[name path = K, label = left:$\mathcal{C}\colon\mathbf{K}$, draw, line width =1pt, circle through = (G)] at (D) {};

%Paso 6: Encontrar el punto de intersección L.
\path[name intersections = {of = K and D-E, by = {[label = below:\textcolor{DarkBlue}{$\mathbf{L}$}]L}}];

%Paso 7: Resaltar los segmentos congruentes BC y AL.
\draw[remark, line width =1pt, DarkCyan] (B) -- (C);
\draw[remark, line width =1pt, DarkMagenta] (A) -- (L);

%Paso 8: Marcar los puntos A, B, C, D, L y G.
\foreach \point in {A, B, C, D, L, G}{
\fill[black, opacity =.5] (\point) circle[radius = 1.15pt];
}
%Paso 9: Enunciar el teorema. 
\node[below, text width = 13cm, align = justify] at (-0.5,-3.5){
	\begin{prop}[Euclides 300 A.C.]
	Para colocar una línea recta igual a una línea recta dada con un extremo en un punto dado.
	\end{prop}
	\begin{proof}[Prueba]
	Sea \textcolor{DarkBlue}{$\overline{BC}$} un segmento de recta y $\textcolor{DarkBlue}{\mathbf{A}}$ un punto exterior a ella. Llámese $\mathcal{C}\colon\mathbf{H}$ a la circunferencia de centro \textcolor{DarkBlue}{$\mathbf{B}$} y radio \textcolor{DarkBlue}{$BC$}, trácese el segmento $\overline{AB}$.\vskip.5em
	De la proposición I, halle el punto $\mathbf{D}$ (ayuda: construya dos circunferencias de radio \textcolor{DarkBlue}{$AB$} con centro en \textcolor{DarkBlue}{$\mathbf{A}$} y \textcolor{DarkBlue}{$\mathbf{B}$} respectivamente, la intersección será el punto en cuestión).\vskip.5em
	A continuación desde $\mathbf{D}$ trácese dos rectas que pasen por \textcolor{DarkBlue}{$\mathbf{B}$} y \textcolor{DarkBlue}{$\mathbf{A}$} respectivamente. El punto de intersección con la circunferencia $\mathcal{C}\colon\mathbf{H}$ será llamado $\mathbf{G}$.\vskip.5em
	Con centro en $\mathbf{D}$ trácese la circunferencia $\mathcal{C}\colon\mathbf{K}$ de radio $DG$. Luego, el punto $\mathbf{L}$ es la intersección de $\mathcal{C}\colon\mathbf{K}$ y el segmento $\overline{DE}$. Así obtenemos los segmentos $\overline{BC}\cong\overline{AL}$.
	\end{proof}
};
\end{tikzpicture}
\end{document}
Para cortar de la mayor de dos líneas rectas desiguales, una línea recta igual a la menos